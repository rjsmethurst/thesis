\documentclass[12pt,useAMS]{ociamthesis}  % default square logo 
%\documentclass[12pt,ibeltcrest]{ociamthesis} % use old belt crest logo
%\documentclass[12pt,shieldcrest]{ociamthesis} % use older shield crest logo

%load any additional packages
\usepackage{amssymb}
\usepackage{footnote, enumerate, url, amsmath, color}
\usepackage{natbib}
\usepackage{multirow}
\usepackage{tabularx}
\usepackage{hyperref}
\usepackage{amssymb}

\hypersetup{
    colorlinks,
    citecolor=black,
    filecolor=black,
    linkcolor=black,
    urlcolor=black
}

\definecolor{titlecol}{rgb}{0,0,0}
\definecolor{titlecol2}{rgb}{0,0.65,0}
\definecolor{titlecol3}{rgb}{0.99,0.4,0.}
\definecolor{minorcol}{rgb}{1,0,0}
%\definecolor{titledark}{rgb}{0,0,0.8}
%\definecolor{hilit}{rgb}{0,0,1}
%\definecolor{hilitdark}{rgb}{0,0,0.8}
%\font\sbf=cmssbx10 at 32.28pt %big font for headers

\font\nbf=cmssbx10 at 12.28pt %big font for headers

%%%%%%%%%%%%%%%%%%%%%%%%%%%%%%%
%%%% If you want to leave notes in the text feel free to define
%%%% your own colour above and a style below
%%%%%%%%%%%%%%%%%%%%%%%%%%%%%%%
\def\note		{\color{titlecol2} \nbf}
\def\noteb		{\color{titlecol} \nbf}
\def\notebsm	{\color{titlecol}}
\def\notecjl	{\color{titlecol2} \nbf}
\def\notecsm	{\color{titlecol2}}
\def\noterjs	{\color{titlecol3} }
\def\minor		{\color{minorcol}}

%%%%%%%%%%%%%%%%%%%%%%%%%%%%%%%
% For the eventual referee response
\def\changed    {\color{titlecol} }
%\def\changed    {}


%%%%%%%%%%%%%%%%%%%%%%%%%%%%%%%
%  Other stuff I use a lot
\def\oiii		{$\mathrm{\left[ O \textsc{iii}\right] }$}
\def\moiii		{\mathrm{\left[ O \textsc{iii}\right] }}
\def\nii		{$\mathrm{\left[ N \textsc{ii}\right] }$}
\def\mnii		{\mathrm{\left[ N \textsc{ii}\right] }}
\def\sii		{$\mathrm{\left[ S \textsc{ii}\right] }$}
\def\msii		{\mathrm{\left[ S \textsc{ii}\right] }}
\def\galfit     {{\tt GALFIT}}
\def\gandalf     {{\tt GANDALF}}
\def\emcee      {{\tt emcee}}

\def\msun	       {$\rm{M}_{\odot}$}
\def\mmsun	{\rm{M}_{\odot}}
\def\fnobulge    {$f_{\rm no~bulge}$}
\def\mfnobulge {f_{\rm no~bulge}}
\def\fedgeon     {$f_{\rm edge-on}$}
\def\mfedgeon  {f_{\rm edge-on}}
\def\fcnobulge    {$f_{\rm confirmed~no~bulge}$}
\def\mfcnobulge {f_{\rm confirmed~no~bulge}}
\def\ha               {H$\alpha$}
\def\mha            {{\rm H}\alpha}
\def\mbh           {$M_{\rm BH}$}
\def\mmbh        {M_{\rm BH}}

\def\kappamean {$\overline{\kappa}$}
\def\mkappamean {\overline{\kappa}}

\def\lesssim{\mathrel{\hbox{\rlap{\hbox{\lower3pt\hbox{$\sim$}}}\hbox{\raise2pt\hbox{$<$}}}}}
\def\gtrsim{\mathrel{\hbox{\rlap{\hbox{\lower3pt\hbox{$\sim$}}}\hbox{\raise2pt\hbox{$>$}}}}}


\newcommand\nodata{ ~$\cdots$~ }

%define all that you need to define
\def\starpy {\textsc{starpy}}

%input macros (i.e. write your own macros file called mymacros.tex 
%and uncomment the next line)
%\include{mymacros}

\title{The influence of morphology, AGN and environment on the quenching histories of galaxies}   %note \\[1ex] is a line break in the title

\author{Rebecca Jane Smethurst}             %your name
\college{Pembroke College}  %your college

%\renewcommand{\submittedtext}{change the default text here if needed}
\degree{Doctor of Philosophy}     %the degree
\degreedate{Michaelmas 2016}         %the degree date

%end the preamble and start the document
\begin{document}

%this baselineskip gives sufficient line spacing for an examiner to easily
%markup the thesis with comments
 
%set the number of sectioning levels that get number and appear in the contents
\setcounter{secnumdepth}{3}
\setcounter{tocdepth}{3}


\maketitle                  % create a title page from the preamble info

%\begin{dedication}
%This thesis is dedicated to\\
%all of those\\
%that dare to dream.
%\emph{``The Universe is under no obligation to make sense to you.''} \\- Neil deGrasse Tyson
\newpage
\vspace*{3cm}
\begin{center}
\emph{``All we have to decide, \\ is what to do with the time \\ that is given to us.''} \\\end{center} \begin{flushright}- J.~R.~R. Tolkein\end{flushright}

%\end{dedication}
\newpage

\begin{abstract}  

% What's your big Q?
What drives the transition of galaxies from the disc dominated, star forming blue cloud to the elliptical dominated, quiescent red sequence? What role does the morphology, central supermassive black hole and galaxy environment play in this transition? 
% How do you answer them? STARPY?
I have attempted to answer these questions by using Bayesian statistics to infer a simple star formation history (SFH) describing the time, $t_q$, and exponential rate, $\tau$, that quenching occurs in a galaxy. I use both the optical and NUV photometry of a galaxy in order to infer the posterior distribution of its SFH across the two dimensional $[t_q, \tau]$ parameter space. I then utilise the Galaxy Zoo 2 morphological classifications to obtain a morphology weighted, combined population distribution across each quenching parameter for a sample of galaxies. 
% What do you find in morph chapter?

I apply this method across the blue cloud, green valley and red sequence of a sample of $126,316$ galaxies and find a clear difference between the quenching timescales preferred by smooth and disc weighted populations, with three major routes through the green valley dominated by smooth (rapid rates, attributed to major mergers), intermediately classified (intermediate rates, attributed to galaxy interactions) and disc morphologies (slow rates, attributed to secular evolution). I hypothesise that morphological changes occur in systems which have undergone quenching with an exponential rate, $\tau < 1.5~\rm{Gyr}$, in order for the evolution of galaxies in the green valley to match the ratio of smooth to disc galaxies observed in the red sequence.
% What do you find in AGN feedback chap?

I repeat this SFH analysis for a sample of $1,244$ Type 2 AGN host galaxies and find statistical evidence for recent, rapid quenching, suggesting that this may be caused by AGN feedback. However I find that rapid quenching rates cannot account for all the quenching across the AGN host population; slow quenching rates, attributed to secular evolution, are also significant in the evolution of AGN host galaxies.
% What do you find in bulgeless AGN?

I investigate this possible secular co-evolution of galaxies and black holes further by measuring the black hole masses of a sample of $101$ bulgeless AGN host galaxies and compare them to typical black hole-galaxy scaling relations. I find that the measured black holes of the bulgeless galaxies are $\sim1-2~\rm{dex}$ more massive than they should be, given their lack of bulges. This suggests that black hole-galaxy scaling relations may arise due to mutual correlations to the overall gravitational potential of the dark matter halo of the galaxy. 
% What do you find in environment chapter?

I also considered the effect of the group environment on the time and rate that quenching occurs, with respect to the group-centric radius, for $4,629$ satellite galaxies. I find that although mergers, mass quenching and morphological quenching are all occurring in groups, environmentally driven quenching mechanisms are also prevalent. However, I find that these environmentally driven quenching processes are not correlated with the velocity of a satellite within a group, ruling out ram pressure stripping as a possible mechanism. 
% Sum up your big picture

I discuss how all of these quenching mechanisms are likely to affect a galaxy across its lifetime, acting in concert to reduce the SFR, which in turn produces the wide distribution of quenching timescales seen across the colour-magnitude diagram. I discuss ideas for future work using the method employed in this work, including applying it to forthcoming data from large integral field unit surveys. 

\end{abstract}

\begin{originalitylong}
I carried out the work presented in thesis at the Department of Astrophysics, University of Oxford between October 2013 and December 2016, under the supervision of Prof. Chris Lintott. It was funded by a Science Technology Facilities Council Studentship Grant Code ST/K502236/1. I hereby declare that no part of this thesis has been submitted in support of another degree, diploma or other qualification at the University of Oxford or other higher learning institute. Except where otherwise stated or where reference is made to the work of others, the work in this thesis is entirely my own.

The work in Sections~\ref{chap:starpy}, \ref{chap:morph} \& \ref{sec:agnfeedback} is based on the peer-reviewed, published papers \citealt{smethurst15}, \mnras, 450, 435, and \citealt{smethurst16}, \mnras, 463, 2986, for which I am the lead author {\minor and contributed $\sim85\%$ of the work}. 

The work presented in Section~\ref{sec:intbulgeless} was carried out in collaboration with Dr. Brooke Simmons and Prof. Chris Lintott and is in preparation for submission to \mnras.

The work throughout this thesis relies on observations from the Sloan Digital Sky Survey (SDSS). Funding for SDSS-III has been provided by the Alfred P. Sloan Foundation, the Participating Institutions, the National Science Foundation, and the U.S. Department of Energy Office of Science. The SDSS-III web site is \url{http://www.sdss3.org/}.

%SDSS-III is managed by the Astrophysical Research Consortium for the Participating Institutions of the SDSS-III Collaboration including the University of Arizona, the Brazilian Participation Group, Brookhaven National Laboratory, Carnegie Mellon University, University of Florida, the French Participation Group, the German Participation Group, Harvard University, the Instituto de Astrofisica de Canarias, the Michigan State/Notre Dame/JINA Participation Group, Johns Hopkins University, Lawrence Berkeley National Laboratory, Max Planck Institute for Astrophysics, Max Planck Institute for Extraterrestrial Physics, New Mexico State University, New York University, Ohio State University, Pennsylvania State University, University of Portsmouth, Princeton University, the Spanish Participation Group, University of Tokyo, University of Utah, Vanderbilt University, University of Virginia, University of Washington, and Yale University.

The work throughout this thesis is based on observations made with the NASA \emph{Galaxy Evolution Explorer}. GALEX is operated for NASA by the California Institute of Technology under NASA contract NAS5-98034.

This thesis has been made possible by the participation of more than $250,000$ users in the Galaxy Zoo project. Their contributions are individually acknowledged at \url{http://authors. galaxyzoo.org}.

This thesis made extensive use of the Tool for Operations on Catalogues And Tables \citep[TOPCAT;][]{taylor05} and the {\tt astropy} Python module \citep{astropy13}. This research has also made use of NASA's ADS service and Cornell's ArXiv.

The copyright of this thesis rests with the author. No quotation from it or information derived from it may be published without the prior consent and acknowledgement of its author. 

\begin{flushright}
Rebecca J. Smethurst
\\
(\emph{December 2016})
\end{flushright}


\end{originalitylong}

\begin{acknowledgements}


I can't quite believe that I've actually written my thesis. It still feels like something I'll have to do eventually; something that needs crossing off the master to-do list. That's not for the lack of hard work that it took to get to this point (trust me!), but more because the last three years and eight weeks have been so enjoyable. Although I had a lot to do with the whole hard work part, I can't claim credit for how much I've enjoyed my DPhil - that all goes to the people in my life that made it so. 

First and foremost I have to thank my supervisor Prof. Chris Lintott. Chris, thank you for taking a chance on me and for bringing me back into the world of astronomy that I had so sorely missed. Somehow, you just understood me from the start; that I mostly wanted to be left to my own devices and that buying me the occasional rum wouldn't go amiss. I suppose I should also mention all those hours we met to talk science - that was pretty useful too, I guess. In all seriousness though, I'm honoured to have been your first graduate student and I can't imagine having been supervised by anybody else. 

A massive thank you also has to go to Dr. Brooke Simmons. Brooke, I lost count a long time ago of how many times you made me laugh so hard it hurt. Thank you for making observing, working on a paper and going to conferences some of the most fun experiences of my life. With those two trips to visit you in San Diego I've just about forgiven you for leaving! You are literally the best friend and colleague anyone could wish for and always have the best advice for those following in your footsteps. And the best Taylor Swift GIFs, obviously. 

Thanks also have to go to the entire Galaxy Zoo Team for their support and collaboration over the past three years. In particular Dr. Karen Masters for her advice and genuine interest in my career; thank you Karen for showing me that the dream is real.

Three years of PhD life would be truly impossible without a group of friends you can count on to make you laugh, cry and get the drinks in. I am lucky enough to have many of those. To my office mates over the past three years, thank you for putting up with my oddities and for making me laugh with yours. To my fellow PhD students, thank you for sharing this experience with me and being there with some cake or a chat whenever needed. To all those at Pembroke, thank you for providing me with a home away from home and being my \emph{Cheers} bar. To anyone who has ever been part of the Zooniverse family: I've never known a more wonderful, eclectic group of people and I'll be forever grateful to have been welcomed in for all the hijinks. To all my friends from home and from Durham, thank you for reminding me that there is a world outside of academia and for helping me pop the Oxford bubble once in a while.

Special mention has to go to Laura, Ricarda, Ali, Grant, Mark, Sarah, Christina, Will, and Malte. For all of the pub trips, poker nights, formals, bops and bar nights. Only time will tell what effect you wonderful people have had on my liver.

Thanks must also go to the genius that is Gilmore Girls for bringing me much needed joy every evening when I got home after a long day writing. Seven seasons and seven thesis sections - coincidence? I think not! Music has been an exceptional source of motivation during my write up; with the entire thing written under the influence of a list of artists which would probably be longer than my thesis. As I write these acknowledgements, I'm listening to The Boss, which feels appropriate. 

I'm not sure if the fact that I'm listening to Bruce Springsteen as I type this or the fact that I've actually completed my thesis will make my parents more proud; it's a pretty close call! Mum \& Dad, I don't think there's enough trees on this planet to make enough paper for me to write down how grateful I am for everything you've ever done for me. You always told me there was nothing I couldn't do if I set my mind to it, you gave me every opportunity and you trusted me to make the right decisions about my life. I hope what I've decided to do with it makes you proud, because I am one proud daughter. I'm also one proud sister. Meg, during the past three years I've got to watch you grow from my crazy, wonderful little sister into the crazy, smart young woman who walks around Bath like she's always lived there. Thank you for being my best friend, for sharing in my crazy and for bringing so much laughter into my life. You're my person.

So I'm already weeping at my desk but there's still one person left to thank. Sam, you are the best thing in my life. I count myself lucky for every day I get to spend with you. Thank you for supporting me every step of the way through my PhD, even if it meant spending time apart. Thank you for being the other half of my crazy and for bringing so much joy to my days. I still maintain that the only person who will ever read my thesis as thoroughly as I did, is you. I'd say I can't believe you did that for me, but I can. 

Oy with the poodles already!


\begin{flushright}
Becky 
\\
(\emph{December 2016})
\end{flushright}

\end{acknowledgements}

\begin{romanpages}          % start roman page numbering
\tableofcontents            % generate and include a table of contents
\listoffigures              % generate and include a list of figures
\end{romanpages}            % end roman page numbering


\baselineskip=18pt plus1pt
\linespread{1.2}
 \setlength{\parskip}{1em}
 
%now include the files of latex for each of the chapters etc
\chapter{Introduction}\label{chap:intro}

Understanding the physical processes which have shaped our Universe is the fundamental goal of Astrophysics. A successful theory of the Universe must describe how it evolved from the primordial soup of matter present just after the Big Bang to the variety of galaxy properties we see today. 

The most widely accepted cosmological model is the $\Lambda$-CDM ($\Lambda$- cold dark matter) model which describes a flat Universe made of only $\sim5\%$ baryonic (``normal'') matter, $\sim26\%$ cold dark matter and $\sim69\%$ dark energy \citep{planck16}. In such a Universe, tiny quantum fluctuations in the early Universe grow with time, becoming overdense and laying the foundations for galaxy formation \citep{guth82, hawking82, linde82, starobinsky82}. Given the relatively small fraction of baryonic matter in the Universe, its gravitational contribution to this process is often neglected, greatly simplifying the problem. Structure in simulations is then observed to form hierarchically \citep{press74, gott75, white78, aarseth79, gott79, turner79, efstathiou81, davis85}. At the overdense regions in the early Universe, matter collapses dissipationally under its own gravity forming a dark matter `halo', which then grows through smooth accretion and mergers with other halos to produce the inhomogeneous large scale structure of galaxy filaments, clusters and voids we observe today \citep[see comprehensive review by][]{frenk12}. 

Adding localised baryonic physics into this picture, however, complicates matters. Galaxies are, unfortunately, not just simple smooth dark matter halos; they are thought to start life when baryons cool and condense at the centre of a dark matter halo. Further accretion of matter will cause a gravitational collapse \citep[if angular momentum is present as a result of tidal torques then a rotating gas disc will form][]{fall80, barnes87} and stars will form as hydrogen gas cools and coalesces. From this moment a galaxy will evolve, its shape changing depending on its encounters with other galaxies. 

Today we observe galaxies with a multitude of shapes, or morphologies, across all redshift ranges. \cite{hubble36} classified galaxies based on their shape, producing the widely adopted `tuning fork diagram', now known as the Hubble sequence and shown in Figure~\ref{fig:hubble}. Hubble noticed that galaxies could be broadly categorised as either ellipticals or discs with spiral arms and/or barred structures. He referred to these categories as early-types (placed to the left of his diagram, in keeping with time axis conventions) and late types respectively, because he thought that as galaxies evolved they developed spiral structure. However, as discussed above, cosmological studies have concluded that galaxies start life as a rotating gas disc and so instead, Hubble's diagram is often read from right to left \citep[with some debate over the placement of the S0 galaxies in this picture;][]{kormendy96}. Connecting this picture of pure gas disc galaxies in their infancy with the plethora of galaxy structures we see today, is the focus of this thesis. 

\begin{figure}
\centering{
\includegraphics[width=\textwidth]{introduction/hubble.jpeg}}
\caption[The Hubble sequence for morphological classification of galaxies]{The Hubble sequence of galaxy morphology shown on his famous `tuning fork diagram' as published in \cite{hubble36}.}
\label{fig:hubble}
\end{figure}


However, galaxy morphology alone is not enough to characterise a galaxy. The magnitude (used as a proxy for stellar mass), star formation rate (SFR), metallicity (Z) and environment are all crucial to describing a galaxy's current state. By studying these galaxy properties, insights into the processes which govern galaxy evolution can be gained.

Large scale surveys of galaxies first revealed a bimodality in the optical colour-magnitude diagram (CMD) of galaxies revealing two distinct populations (see Figure~\ref{fig:cmdbaldry}); one at relatively low mass with blue optical colours and another at relatively high mass with red optical colours \citep{Baldry04, Baldry06, Willmer06, ball08, Brammer09}. These populations were dubbed the `blue cloud' and `red sequence' respectively \citep{Chester64, bower92, Bell04, Driver06, Faber07}.  The sparsely populated colour space between these two populations was dubbed the  `green valley'.

\begin{figure}[t]
\centering{
\includegraphics[width=\textwidth]{introduction/cmd.pdf}}
\caption[Galaxy Colour Magnitude Diagram from \cite{Baldry04}]{The galaxy colour magnitude diagram as observed by \cite{Baldry04}. The figure has been adapted from Figure 1 in \citeauthor{Baldry04} and annotated to show the locations of the red sequence and blue cloud. A lower magnitude corresponds to a higher mass and a large $u-r$ value corresponds to a redder colour.}
\label{fig:cmdbaldry}
\end{figure}


The majority of {\minor late-type} galaxies were found in the blue cloud and the majority of {\minor early-types} on the red sequence, with colour often used as a proxy for morphology. The Galaxy Zoo project \citep{lintott08, Lintott11}, which produced morphological classifications for a million galaxies, helped to confirm that this colour bimodality is not entirely morphology driven \citep{Strat01, Salim07, Sch07, CHV08, Bamford09, Skibba09}, detecting larger numbers of spiral galaxies in the red sequence \citep{masters10c} and smooth galaxies in the blue cloud \citep{Sch09} than had previously been observed. A change in colour therefore doesn't always coincide with a change in morphology. 

Large galaxy surveys also revealed that star forming galaxies are observed to lie on a well defined `star forming sequence' (SFS) in the stellar mass vs. star formation rate (SFR) plane \citep{brinchmann04, Salim07, daddi07}. The majority of blue cloud galaxies are found to lie on this SFS with the majority of the red sequence lying well below it with very low SFRs. The green valley, which lies between the red sequence and blue cloud, is therefore assumed to contain galaxies which have recently undergone a suppression of star formation \citep[SF;][]{Salim07}. This suppression of SF and subsequent transition of a galaxy from blue cloud to red sequence must therefore also be intrinsically tied with a possible change in morphology from a {\minor late-type} galaxy to an {\minor early-type} galaxy. This is entirely possible in hierarchical structure formation through mergers of galaxies.

The discovery of the fundamental plane \citep{dressler87, djorgovski87}, wherein the stellar velocity dispersion is correlated with the luminosity and effective radius of an elliptical, further constrained the mechanisms responsible for galaxy formation. The fundamental plane can be reproduced in simulations of major mergers \citep{bekki98, nipoti03, boylan05, robertson06, hilz12, taranu15}, lending support to the theory of hierarchical structure formation.

Further evidence for hierarchical structure formation is that galaxies are often found huddled together in groups \citep{zwicky38, zwicky52, abell58}, all sharing one large dark matter halo (groups with 100 or more galaxies are referred to as clusters \citealt{bower04}). Conversely some galaxies are found isolated from others in less dense environments (often referred to as the field), either because they are fossil groups \citep[where all members have eventually merged][]{ponman94, jones00, jones03} or have truly been isolated for their entire lifetimes. This environmental density is found to be correlated not only with morphology \citep[][and see Figure~\ref{fig:dressler}]{dressler80, smail97, poggianti99, postman05, Bamford09}, but also colour \citep{butcher78, pimbblet02}, quenched galaxy fraction \citep{kauffmann03, Baldry06, peng12, darvish16} and SFR \citep{gomez03} suggesting that the environment may drive this transition from blue cloud to red sequence through increasing the likelihood of galaxy mergers. 

\begin{figure}[t]
\centering{
\includegraphics[width=0.8\textwidth]{introduction/dressler.png}}
\caption[Morphology Density relation from Figure 4 of \cite{dressler80}]{The morphology-density relation from Figure 4 of \cite{dressler80} showing how the fraction of ellipticals (E) increases with increasing environmental density and the fraction of spirals (S + Irr) decreases.}
\label{fig:dressler}
\end{figure}

However, within the framework of hierarchical structure formation, the observed mass-metallicity relation \citep{tremonti04} cannot be reproduced. This relationship describes how higher mass galaxies have higher metal contents. Simulations have shown that this phenomenon cannot be reproduced by the merging of galaxies alone, since mergers dilute the metallicity \citep{pipino08,rupke10, montuori10, torrey12}. This is also supported by observations showing that metallicities in galaxy pairs are suppressed by $\sim0.05~\rm{dex}$ \citep{ellison08, michel08, scudder12}. There must therefore be processes other than mergers occurring which are responsible for the colours, SFRs, masses, metallicities and morphologies of galaxies observed. 

Further insight has also recently been revealed by the number of integral field unit (IFU) surveys which have come online in the past couple of years. These include the SAURON instrument \citep{bacon01}, the completed $\rm{ATLAS}^{\rm{3D}}$ survey \citep{cappellari11} which targeted a small sample of 260 early-type galaxies, and the larger, ongoing surveys of MaNGA \citep{bundy15}, SAMI \citep{croom12} and CALIFA \citep{sanchez12}. Upon completion at the end of the decade, these surveys are expected to revolutionise the field of galaxy evolution, providing spatially resolved maps of SF indicators for over $10,000$ galaxies.

Even with the higher resolution data forthcoming, there is still value in studying galaxies which have just left the SFS, to reveal the mechanisms governing galaxy evolution, including both the suppression of SF and the possible transformation of galaxy structure. Green valley galaxies have long been thought of as the `crossroads' of galaxy evolution, a transition population between the star forming blue cloud and the quiescent red sequence. \citet{Bell04} were the first to recognise that green valley galaxies may be a transitional population between the blue cloud and red sequence. \citet{Wyder07} confirmed that galaxy bimodality also appears in NUV-optical colour space, with \citet{Schim07} investigating the morphological dependence in this parameter space and \citet{Martin07} using it to constrain the flux of galaxies transitioning through the green valley. \citet{Faber07} suggested that the build up of the red sequence has to occur via a mixture of suppression of star formation of galaxies in the blue cloud and mergers on the red sequence, which was confirmed by \citet{Mendez11} who showed that mergers alone are not responsible for the transition from the blue cloud. {\minor It has been proposed that this transition occurs} on rapid timescales, otherwise there would be an accumulation of galaxies residing in the green valley \citep{Gonc12}; however \citet{schawinski14} concluded that this was only true for smooth galaxies, with disc galaxies transitioning much more slowly. The morphological dependence and causes of this transition are therefore still the cause of much debate. 


This thesis will focus on studying this transition and the mechanisms responsible. I will refer to the suppression of a galaxy's SFR as \emph{quenching} and processes which can cause this suppression as \emph{quenching mechanisms}. 

\section{Possible quenching mechanisms}\label{sec:quenchmech}

There are many mechanisms {\minor proposed to} cause quenching. They are often referred to as either internal mechanisms (caused by the galaxy's `nature') or external mechanisms (caused by the way the galaxy is `nurtured'). The properties of a galaxy and its environment are often thought to control which mechanisms will affect a galaxy throughout its lifetime and subsequently affect the morphology. 

%There have been many previous theories for the initial triggers of these quenching mechanisms, including negative feedback from AGN \citep{diMatteo05, Martin07, Nandra07, Sch07}, mergers \citep{Darg10a, Cheung12, Barro13}, supernovae winds \citep{MFB12}, cluster interactions \citep{Coil08, Mendez11, Fang13} and secular evolution \citep{masters10c, masters11a, Mendez11}. By investigating the \emph{amount} of quenching that has occurred in the blue cloud, green valley and red sequence; and by comparing the amount across these three populations, I can apply some constraints to these theories. 
\subsection{Internal Quenching Mechanisms}\label{sec:intquench}

\subsubsection{AGN feedback as a quenching mechanism}\label{sec:agnquench}

An active galactic nucleus (AGN) is an actively growing supermassive black hole in the centre of a galaxy. There are many observed spectral classes of AGN which are explained by the viewing angle in the theory of unification \citep[see review by][]{netzer15}. The basic structure of an AGN consists of an accretion disc of {\minor hot} material which is formed around the black hole as the material inspirals and is pulled into a single orbital plane by the force of gravity. The friction built up by the rotating gas causes the accretion disc to heat up and emit large amounts of energy in the form of X-rays. The accretion disc is also surrounded by a broad line region and an absorption region (often referred to as the torus) which can both absorb and re-emit the emitted light from the accretion disc (for a diagram depicting AGN structure see \citealt{beckmann12}). Type 2 Seyfert AGN are those which are viewed approximately edge on to their accretion disc and so have the majority of their emitted light absorbed by the torus. Some light is scattered off the surrounding gas clouds, resulting in the detection of broadened spectral emission lines. Type 1 Seyfert AGN however are those which are viewed perpendicular to their accretion disk and so are unobscured by the torus; light from both the accretion disc and/or emitted radiation jets can be observed. 

% \begin{figure}[t]
% \centering{
% \includegraphics[width=0.9\textwidth]{introduction/agn_unification.pdf}}
% \caption[Illustration of the structure of an AGN from \cite{beckmann12}]{Cartoon of the structure of an AGN including the accretion disc, dusty absorber (often referred to as the obscuring torus) \& output jet, in the context of viewing angle for various spectral classes of observed AGN.}
% \label{fig:agntorus}
% \end{figure}

There are tight correlations between properties of galaxies, such as the bulge mass, total stellar mass \& stellar velocity dispersion, and the black hole mass \citep{magorrian98, marconi03, haringrix04}. This implies a co-evolution between the black hole and its host galaxy therefore suggesting that changes in the SFR and structure of a galaxy could also be tied to black hole activity. 

If we consider a `back of the envelope' calculation of the ratio between the total energy of the black hole and the binding energy of the galaxy, we can determine if output from the black hole will be able to have a significant effect on its host galaxy \citep[see p.649 of][]{mo10}. The total energy output by a black hole in its lifetime can be expressed as $E_{BH} = \overline{\epsilon}M_{BH}c^2$, where $\overline{\epsilon}$ is the mean efficiency of the black hole of mass $M_{BH}$. The galaxy binding energy can be approximated as $E_{gal} \approx M_{*}\sigma^2$, where $M_{*}$ is the stellar mass of the galaxy with stellar velocity dispersion $\sigma$. Taking the average values of $M_{BH} \sim 10^{8.0}~\rm{M}_{\odot}$,~$M_{*} \sim 10^{10.8}~\rm{M}_{\odot}$ and $\sigma \sim 200~\rm{km}~\rm{s}^{-1}$, of the 30 galaxies observed by \citet{haringrix04}, we can estimate the ratio $E_{BH}/E_{gal} \sim \overline{\epsilon} 10^3$. If the efficiency of the black hole is $\sim15\%$ \citep{elvis02}, the energy of the black hole can easily surpass the binding energy of its galaxy, suggesting that a black hole can indeed impact its host. This is thought to occur via AGN feedback where the output of energetic material and radiation from the black hole is theorised to either heat or expel the gas needed for SF in a galaxy, causing a quench. Energy from a black hole is observed to be ejected in narrow, collimated jets of material out of the plane of a galaxy \citep[see review by][]{homan12}, but for AGN feedback to be effective these jets would somehow need to impact the gas in the entire galaxy. 

\begin{figure}[t]
\centering{
\includegraphics[width=0.8\textwidth]{introduction/silk_mamon_LF.pdf}}
\caption[Illustration of the mismatch between theoretical and observed luminosity function from \cite{silk12}]{Cartoon of the role of feedback in modifying the observed luminosity function of galaxies with respect to theoretical predictions. Supernova winds are thought to be responsible at the low mass end, with AGN feedback responsible at the high mass end. Figure 1 in \cite{silk12}.}
\label{fig:lumfuncpic}
\end{figure}

AGN feedback was first suggested as a mechanism for regulating star formation due to the results of simulations wherein galaxies could grow to unrealistic stellar masses \citep{silk98, Bower06, Croton06, somerville08}. Without a prescription for the effects of AGN feedback, the shape of the galaxy luminosity function could therefore not be matched at the high luminosity end \citep{baugh98, baugh05, kauffmann99a, kauffmann99b, somerville01, kitzbichler06}. A similar problem was also encountered at the low end of the luminosity function, which was rectified by the inclusion of the effects of supernova wind feedback \citep{dekel86, powell11}. This is illustrated in Figure~\ref{fig:lumfuncpic} taken from \cite{silk12}. 

Indirect observational evidence has now been found for both positive and negative feedback in various systems (see the comprehensive review from \citealt{fabian12}). The strongest being the indirect evidence that the largest AGN fraction is found in the green valley \citep{cowie08, Hickox09, schawinski10a}, suggesting a link between AGN activity and the process which moves a galaxy from the blue cloud to the red sequence. However, concrete statistical evidence for the effect of AGN feedback on the host galaxy population has so far been elusive.


\subsubsection{Mass quenching}\label{sec:massquench}

Mass quenching is defined by \citet{peng10, peng12} as any quenching mechanism acting independently of a galaxy's environment, but not of its mass. However, there is still much debate over the exact mechanism which is the cause of such a quench. \citet{darvish16} suggest that non-AGN driven feedback mechanisms (for example supernova feedback) are responsible for the correlation observed between the mass quenching efficiency and SFR in \citet{peng10}. However, \citet{gabor15} suggest that this is driven by ``halo quenching processes'' whereby the inflow of cool gas from the galaxy halo is either cut off or hindered from cooling at $M_{halo} > 10^{12}~\rm{M}_{\odot}$ \citep{birnboim03, dekel06}. If this happens, a galaxy uses up the rest of its available gas for star formation via the Kennicutt-Schmidt law \citep{schmidt59, kennicutt98} and consequently grows in mass.

In the rest of this thesis, I refer to mass quenching as a cut off of gas inflow, resulting in a gradual consumption of gas in star formation. This definition of mass quenching is thought to be a dominant mechanism for isolated galaxies in the field \citep{kormendy04}. However, it is also thought that as a galaxy infalls in to a group or cluster over long timescales, gas reservoirs can also be depleted via a mass quenching process \citep{peng12}. 

 
\subsubsection{Morphological quenching}\label{sec:morphquench}

Morphological quenching is the process by which the internal structure of a galaxy can have a negative impact on its own SFR\footnote{Essentially shooting itself in the foot.}. This can happen in one of two ways, either by preventing star formation from occurring or by increasing the rate of consumption of gas for star formation. The former is {\minor thought} to be caused by bulges \citep{bluck14} whereby the large gravitational potential of the bulge prevents the disc from collapsing and forming stars \citep{Fang13}. 

The latter mechanism is {\minor thought} to occur in galaxies hosting bars; the bar funnels gas to the centre of the galaxy \citep{athanassoula92a} where gas is exhausted by star formation effectively quenching the galaxy \citep{zurita04, sheth05}. This process is thought to be responsible for large numbers of red spirals and supported by observations of increasing bar fraction with red colours \citep{masters11a}. Recent observational evidence from \cite{hart16} also suggests that spiral structure can also cause morphological quenching. \citeauthor{hart16} propose that many armed spiral structures can trigger galaxy wide starbursts thereby rapidly using up gas for future star formation; similarly two armed spirals are observed with redder colours, suggesting that this spiral phase is much longer lived and may funnel gas into the centre of the galaxy to be exhausted in star formation over longer timescales.  
  
\subsection{External Quenching Mechanisms}\label{sec:extquench}
  
\subsubsection{Mergers as a quenching mechanism}\label{sec:mergersquench}

Major mergers have been intrinsically linked to the formation of {\minor early-type} galaxies since \citet{toomre72} showed this was possible with a simulation of the merger of two equal mass disc galaxies. Since $\Lambda$-CDM relies on the idea of hierarchical structure formation through the merger of dark matter halos for its description of galaxy formation, it also follows that galaxy evolution should be further influenced by mergers. 

The hypothesis is as follows: when two galaxies merge, the influx of cold gas funnelled by the forces in the interaction often results in energetic starbursts \citep{mihos94, mihos96, hopkins06d, hopkins08a, hopkins08b, snyder11, hayward14, sparre16}, which can exhaust the gas required for star formation, effectively quenching the post-merger remnant. This remnant galaxy will also have formed a dynamically hot bulge through the dissipation of angular momentum in the merger \citep{toomre77, walker96, kormendy04, hopkins11c, martig12}. The mass ratio of the two galaxies merging is thought to affect the size of the bulge that is formed in the remnant \citep{cox08, hopkins09c, tonini16}, with the most massive major mergers with a 1:1 mass ratio producing fully elliptical galaxies \citep{toomre72, barnes96, mihos96, kauffmann96, pontzen16}.

Such a scenario is also intrinsically linked to the triggering of an AGN due to the influx of gas in the merger which can fuel the black hole accretion \citep{sanders88, dimatteo05, hopkins09a, treister12}. Many studies have therefore focussed on investigating the growth of black holes due to mergers \citep[e.g.][]{veilleux02, bellovary13, ellison13, medling15, gabor16}. Simulations of mergers with AGN have led many to believe that a merger which triggers both a starburst and an AGN can quench a galaxy in extremely rapid timescales \citep{springel05b, bell06}. Recent simulations have also suggested that feedback from the triggered AGN (see Section \ref{sec:agnquench}) is necessary to fully remove (or heat) all the available gas, otherwise the SFR will recover back to the SFS post-merger \citep{pontzen16, sparre16}. 

Mergers also have a clear environmental dependence, as they are more likely to occur in denser environments. However, their effects must be separated from those quenching mechanisms driven solely by the properties of the galaxy environment. 

\subsubsection{Environment driven quenching}\label{sec:envquench}

The environment of a galaxy has long been considered a key  `nurturing' aspect of galaxy evolution. Correlations of galaxy morphology \citep{dressler80, smail97, poggianti99, postman05, Bamford09}, colour \citep{butcher78, pimbblet02} and the quenched galaxy fraction \citep{kauffmann03, Baldry06, peng12, darvish16} with the environmental density all suggest that the environment is in some way responsible for the build up of the red sequence through quenching. 

The proposed quenching mechanisms under the umbrella of environmental quenching are numerous and varied. Together with the typical gravitational galaxy-galaxy interactions \citep{moore96} which are expected to be more frequent in a dense environment, environmental quenching also includes hydrodynamic interactions occurring between the cold interstellar medium (ISM) of the in-falling galaxy and the hot intergalactic medium (IGM) of the group or cluster. Such hydrodynamic interactions include ram pressure stripping \citep{gunngott72}, viscous stripping \citep{nulsen82}, and thermal evaporation \citep[a rapid rise in temperature of the ISM due to contact with the IGM;][]{cowie77}. Another such process is starvation \citep[also called strangulation;][]{larson80} which can remove the outer galaxy halo, thus cutting off the star formation gas supply to a galaxy. Preprocessing occurs when all of the above mechanisms take place in a group of galaxies which then merges with a larger group or cluster \citep{dressler04}. 

The most likely (and therefore the most studied) candidate mechanism for the cause of the environmental density-morphology and SFR relations is ram pressure stripping \citep[RPS;][]{abadi99, poggianti99}. However, there has been mounting evidence that RPS can only strip a galaxy of $40-60\%$ of its gas supply \citep{fillingham16} and so may not be as effective a quenching mechanism as first thought \citep{emerick16}. Therefore investigations of other environmentally driven quenching mechanisms, such as strangulation \citep{peng15, hahn16, maier16, paccagnella16, roberts16, vandevoort16} and harassment \citep[high speed galaxy `fly-by' gravitational interactions][]{bialas15, smith15b} are having a recent resurgence. 

\section{Using star formation histories to investigate quenching}\label{sec:invquench}

% previous work
% using SFHs
% the benefits of SPSs

In order to understand how a galaxy is quenched, the star formation history (SFH) is often modelled. This approach requires the inference of the global SFH of a galaxy from its current stellar population. While it is possible to observe resolved stellar populations for nearby galaxies with the \emph{Hubble Space Telescope} (HST) in order to pinpoint the main sequence turn off (an indicator of the age of a stellar population) on the Hertzsprung Russell diagram, this is not possible in more distant galaxies. Instead, reliance is placed upon the correlation between the integrated total light of a galaxy and its SFH \citep{searle73}. Adopting a global SFH for a galaxy is a big assumption, especially since recent IFU studies have shown that bulges and discs have vastly different SFHs \citep[see for example][]{johnston16}. However, these differences will smear out across the integrated light of a galaxy and so inferring the SFH in this way will give an estimate of the average global SFH of the galaxy. This is still useful information, especially when investigating the average quenching history of a large population of galaxies.

Many studies have employed this technique using either photometric broadband colours or spectral data as indicators of the integrated SFH \citep[for example][]{deJong96, madau98, davies01, kauffmann03, dressler04, macarthur04, Martin07, perez11, sanchez11, mcdermid15}. However, this technique is sensitive to the degeneracies between age and metallicity \citep{worthey94} as well as the effects of dust \citep{ganda09, pastrav13} on the integrated light. 

The turn of the millennium therefore saw the development of full spectral energy density (SED) fitting codes to infer the SFH (without having to make any prior assumptions on its form) such as \textsc{moped} \citep{heavens00}, \textsc{starlight} \citep{cidfernandes05}, \textsc{vespa} \citep{tojeiro07}, and more recently \textsc{firefly} \citep{wilkinson15}. In the absence of spectral data however, broadband colours are still effective at inferring the global SFHs of galaxies, especially when wavebands across the spectrum, such as ultra-violet, optical and infrared colours are used simultaneously \citep{madau98}. 

This method is achieved by modelling the observed SED of a galaxy using a combination of SEDs of simple stellar populations (SSP) at various ages. SSPs assume that stars are coeval and form with the same metallicity and comprehensive knowledge of stellar evolutionary tracks and initial mass functions \citep[IMF;][]{salpeter55, chabrier03} are therefore needed to calculate the spectrum of a SSP at a given age \citep{chen10, kriek10}. Luckily, some astronomers have made the study of these SSPs their life's work \citep[for example][]{BC03, Maraston05, vazquez05, CGW09}, therefore once an IMF and a SFH function have been assumed, the SED of a model galaxy can be predicted at any point in its history \citep{chen10}. This technique also assumes a universal IMF, which recent studies have shown may not be appropriate \citep{vandokkum08, conroy12, cappellari12, smithr15}. Whilst the choice of an IMF and metallicity of a SSP will affect the output SED \citep{CGW09, kriek10}, the choice of the functional form of the SFH will have the greatest impact. 

Many possible forms for global galaxy SFHs have been assumed in previous studies, including an exponential decline \citep{tinsley72, gavazzi02, weiner06, Martin07, noeske07, kriek10,  schawinski14, hart16}, the extended (or delayed) exponential model \citep{gavazzi02, oemler13, simha14}, a Gaussian distribution \citep{feuillet16} or a log normal distribution \citep{gladders13, abramson16}. The studies of \cite{lee10}, \cite{boquien14} and \cite{smith15} have shown that these SFHs don't accurately characterise the detailed SFH of a galaxy, as they generalise the localised bursts of star formation across a galaxy's lifetime into a global SFH. In an investigation of the SFH of a single galaxy these forms of the SFH are therefore not appropriate, however, when studying the general quenching histories of a large population of galaxies these functional forms are still appropriate. They allow for insight to be gained on the complex processes responsible for the galaxy properties observed across the population by allowing the quenching history to be described by just $2$ parameters. 

In this thesis I will assume an exponentially declining SFH, following the work of \cite{schawinski14}, for galaxies with morphological classifications from Galaxy Zoo. Employing Bayesian methods, I will use optical and NUV photometry to infer the dependence of quenching histories on the morphology across the colour magnitude diagram. I will also investigate the quenching histories of galaxies that host AGN and those in dense environments to help constrain the quenching mechanisms responsible for galaxy evolution.

\section{Data}\label{sec:data}

In the following section I describe the data sources for the optical \& NUV colours and morphologies used throughout this study.

\subsection{Sloan Digital Sky Survey}\label{sec:sdssintro}

The Sloan Digital Sky Survey (SDSS; \citealt{york00}) is an optical imaging and spectroscopic survey of $8,000$ square degrees of sky, which was completed using a $2.5 \rm{m}$ telescope at Apache Point Observatory in New Mexico, USA. SDSS Data Release 8 \citep{aihara11} provided publicly available optical magnitudes across $5$ broadband filters, $ugriz$, for over 1 million galaxies in the `main galaxy' sample. {\minor Across a redshift range of $0.005 < z < 0.25$, the physical scale varies from $\sim0.1~\rm{kpc}/''$ to $\sim3.9~\rm{kpc}/''$, therefore using a fixed aperture to acquire galaxy photometry will result in aperture bias. Here I instead} utilise the Petrosian magnitude, {\tt petroMag}, values for the $u$ ($3,543 \rm{\AA}$) and $r$ ($6,231 \rm{\AA}$) wavebands provided by the SDSS pipeline. Spectral data is available for a significant proportion of the SDSS main galaxy sample but spectral fibre is a set size {\minor and so will suffer from aperture bias}. Across a population of galaxies the fibre will cover varying radii depending on the distance and size of a galaxy. Therefore the usual spectral star formation indicators cannot be utilised in this study as they will over- or under-estimate the global average SFR of a galaxy. 

Magnitudes are corrected for galactic extinction \citep{Oh11} by applying the \citet{Cardelli89} law, giving a typical correction of $u-r \sim 0.05$. K-corrections are also adopted to $z=0.0$ and absolute magnitudes obtained from the NYU-VAGC \citep{Blanton05, padmanabhan08, blanton07}, giving a typical $u-r$ correction of $\sim 0.15$ mag. The change in the $u-r$ colour due to both corrections therefore ranges from $\Delta (u-r) \sim 0.2$ at low redshift, increasing up to $\Delta (u-r) \sim 1.0$ at $z \sim 0.25$, which is consistent with the expected k-corrections shown in Figure 15 of \citet{blanton07}. These corrections were calculated by \citet{Bamford09} for a subset of galaxies in the SDSS survey. These corrections are a crucial aspect of this work since a $\Delta (u-r) \sim 1.0$ can cause a galaxy to change whether it is classed as blue cloud, green valley or red sequence.

Star formation rates and stellar masses, where available, were obtained from the MPA-JHU catalogue \citep{kauffmann03, brinchmann04}.  I use the average values of \textsc{avg sfr} and \textsc{avg mass} from the inferred likelihood distributions for each galaxy. These SFRs are derived from emission lines using the method of \cite{charlot01}. All SFRs are corrected for aperture size by fitting to the photometry outside the fibre with stochastic models as in \cite{Salim07}. SFRs for non star forming galaxies and AGN were derived indirectly using the $4,000 \rm{\AA}$ break. For those galaxies with emission lines with low signal-to-noise ratio, the SFR was estimated indirectly using a conversion factor likelihood distribution between the luminosity of the H$\alpha$ Balmer emission line and the SFR. Masses are obtained from fits to the photometry with a large grid of SFHs produced using the \cite{BC03} models. In this thesis these values are obtained for interest to compare samples; they are never used to infer the SFHs of galaxies due to the circular nature of the modelling used to derive these values and infer the SFHs.

\subsection{Galaxy Evolution Explorer}\label{sec:galexintro}

The \emph{Galaxy Evolution Explorer} (GALEX; \citealt{Martin05}) is an ultra-violet space based telescope which images galaxies simultaneously in two broadband filters: both the far ultra-violet (FUV) with an effective wavelength of $1,516 \rm{\AA}$ and in the near ultra-violet (NUV) with an effective wavelength of $2,267 \rm{\AA}$. {\minor In this investigation the GALEX Data Release 5 catalogue \citep[][accessed via the virtual observatory using the TOPCAT application]{bianchi11} is utilised, which does not include those sources in the outer field of view radii, $0.9^\circ < R_{\rm{fov}} < 1.1^\circ$, which suffer from poor astrometry, leading to mismatches to SDSS objects \citep[see][]{?,?}.} Sources detected with GALEX were matched with a search radius of $1''$ to the SDSS data in right ascension and declination. {\minor All cross-matched sources have a GALEX field of view radius, $R_{\rm{fov}} < 0.5^\circ$ which ensures a robust cross match to SDSS}. The {\tt auto} magnitudes provided by the GALEX pipeline are used in this study (for a discussion of aperture bias between different surveys see \citealt{hill11}). All magnitudes are k-corrected and extinction corrected as described in Section~\ref{sec:sdssintro}.

\subsection{Galaxy Zoo}\label{sec:GZ}

Galaxy Zoo (GZ) is a citizen science project enlisting the help of thousands of members of the public to voluntarily classify galaxy images online\footnote{\url{http://galaxyzoo.org}}. The first version of GZ classified just under 1 million SDSS galaxy images as either smooth, spirals or mergers within approximately 6 months of launch \citep{lintott08, Lintott11}. In the second version, GZ2 \citep{GZ2}, volunteers were asked to make more detailed morphological classifications of $304, 022$ images from the SDSS DR8 (a subset of those classified in the first Galaxy Zoo; GZ1). These images were all classified by at least 17 independent volunteers, with the mean number of classifications standing at $\sim42$. GZ is now in its tenth year of classifying and its fifth incarnation, after classifying images from \emph{Hubble Space Telescope} Legacy surveys in GZ:Hubble \citep{willett16} and the CANDELS survey galaxies in GZ:CANDELS \citep{simmons16}. At the time of writing, images from the DeCALS\footnote{\url{http://legacysurvey.org/}} survey and Illustris simulation \citep{vogelsberger14, genel14} are being classified by volunteers. 

\begin{figure}
\centering{
\includegraphics[width=\textwidth]{introduction/gz2_tree.pdf}}
\caption[GZ2 classification decision tree]{Flowchart of the classification tree for GZ2, beginning at the top with Task 0. Tasks are colour-coded by their relative depths in the decision tree with tasks in green, blue and purple respectively one, two or three steps below branching points in the decision tree.}
\label{fig:gztree}
\end{figure}

From GZ2 onwards, all projects have collected classification data via a multi-step decision tree, shown in Figure~\ref{fig:gztree}.  Each individual step in a tree is a \emph{task}, which consists of a \emph{question} with a finite number of possible \emph{answers}. The selection of an answer is called the volunteer's \emph{vote}. The first task of GZ2 asks volunteers to choose whether a galaxy is mostly smooth, is featured and/or has a disc or is a star/artefact. Every volunteer who classifies a galaxy image will complete this task, therefore the most statistically robust classifications are available at this level.

% The Galaxy Zoo 2 (GZ2) project consists of $304, 022$ images from the SDSS DR8 (a subset of those classified in Galaxy Zoo 1; GZ1) all classified by \emph{at least} 17 independent users, with the mean number of classifications standing at $\sim42$. The GZ2 sample is more robust than the GZ1 sample and provides more detailed morphological classifications, including features such as bars, the number of spiral arms and the ellipticity of smooth galaxies. It is for these reasons I use the GZ2 sample, as opposed to the GZ1, allowing for further investigation of specific galaxy classes in the future. 

The classifications from volunteers produces a vote fraction for each galaxy; for example if 80 out of 100 people thought a galaxy was featured and/or had a disc, whereas 20 out of 100 people thought the same galaxy was mostly smooth (i.e. {\minor an early-type}), that galaxy would have raw vote fractions $p_{d} = 0.8$ and $p_{s} = 0.2$. In this example this galaxy would be included in the `clean' disc sample ($p_d \geq 0.8$) according to \cite{GZ2} and would be considered a late-type galaxy. Similar vote fractions can be produced at each stage in the tree, such as $\{p_{\mathrm{bar}}, p_{\mathrm{no~bar}}\}$, $\{p_{\mathrm{spiral}}, p_{\mathrm{no~spiral}}\}$ and $\{p_{\mathrm{odd}}, p_{\mathrm{not~odd}}\}$. Selecting a sample of galaxies with a specific feature using these vote fractions becomes a trade off between purity and completeness. Since not every volunteer will submit a response to a 2nd, 3rd or 4th tier question (see Figure~\ref{fig:gztree}), the number of classifiers recording a response must be considered, in order to reduce noise in cases where only a small number of people answered that task. 

For example, imagine that a galaxy is classified by $40$ people, $38$ of whom say that the galaxy is mostly smooth in answer to the first question. However, $2$ people decide that the galaxy is featured and/or has a disc, both of whom subsequently respond to Task 2 to say that there is a sign of a bar in the same galaxy. This would give a $p_{\rm{bar}}=1$, despite the fact that the $p_s = 0.95$. This is an unlikely situation, but highlights the need for not only consideration of the number of respondents for a task but also the vote fractions of previous tasks when using a threshold to identify a subset of features. Appropriate values for these thresholds given the number of respondents are shown in Table~\ref{table:votes} (reproduced from Table 3 in \citealt{GZ2}) and are adopted where relevant throughout this study. 

\begin{table}[t]
\centering
 \begin{tabular*}{\textwidth}{l@{\extracolsep{\fill}}lcc}
 \hline
\multicolumn{1}{l}{Task} &
\multicolumn{1}{l}{Previous task} &
\multicolumn{1}{c}{Vote fraction} &
\multicolumn{1}{c}{Vote fraction}
\\ 
\multicolumn{1}{l}{} &
\multicolumn{1}{l}{} &
\multicolumn{1}{c}{$N_{task}\geq10$} &
\multicolumn{1}{c}{$N_{task}\geq20$}
\\ 
\hline					
00                      & --        & --        & --        \\
01                      & 00        & 0.227     & 0.430     \\
02                      & 00,01     & 0.519     & 0.715     \\
03                      & 00,01     & 0.519     & 0.715     \\
04                      & 00,01     & 0.519     & 0.715     \\
05                      & --        & --        & --        \\
06                      & 05        & 0.263     & 0.469     \\
07                      & 00        & 0.223     & 0.420     \\
08                      & 00,01     & 0.326     & 0.602     \\
09                      & 00,01,03  & 0.402     & 0.619     \\
10                      & 00,01,03  & 0.402     & 0.619     \\
\hline
\end{tabular*}
\caption[Thresholds for selecting sub-samples of galaxies using GZ2 data]{Thresholds for determining well-sampled galaxies in GZ2 originally published in \cite{GZ2}. Thresholds depend on the number of respondents for a task, including the thresholds that should be applied to previous task(s) for both 10 and 20 respondents. As an example, to select galaxies that may or may not contain bars, cuts for $p_\mathrm{features/disc}>0.430$, $p_\mathrm{not~edgeon}>0.715$, and $N_\mathrm{not~edgeon}\geq20$ should be applied. No thresholds are given for Tasks 00 and 05, since these are answered for every classification in GZ2. The task numbers are those defined in Figure~\ref{fig:gztree}.}
\label{table:votes}
\end{table}

All previous Galaxy Zoo projects have also incorporated extensive analysis of volunteer classifications to measure classification accuracy and bias. A weighting is computed for each volunteer based on their classification history and the redshift of the galaxy in question in order to produce debiased vote fractions for each galaxy (for a detailed description of redshift debiasing and consistency-based volunteer weightings, see either Section 3 of \citealt{Lintott09} or Section 3 of \citealt{GZ2}). This produces highly accurate and robust detailed morphological classifications and is a significant statistical improvement over efforts completed using only a small number of expert classifiers \citep{schawinski07, nair10b, ann15}. These classifications can now be used as machine learning training sets \citep{dieleman15} to improve future automated classifications of galaxies. 

The debiased GZ2 $p_d$ and $p_s$ vote fractions encompass the continuous spectrum of morphological features (as shown in Figure~\ref{fig:mosaic}), rather than a simple binary classification separating smooth and disc galaxies (see Section~\ref{sec:bigpic}). These classifications allow each galaxy to be considered as a probabilistic object with both bulge and disc components. I utilise the debiased GZ2 vote fractions in this study to facilitate future work studying more detailed galaxy structures. 

\subsection{Defining the GZ2-GALEX main galaxy sample}\label{sec:defsample}

\begin{figure}
\centering{
\includegraphics[width=\textwidth]{introduction/mosaic_disc_fraction_z_0-07_0-075.pdf}}
\caption[Example SDSS images with GZ2 vote fractions]{Randomly selected SDSS $gri$ composite images showing the continuous probabilistic nature of the Galaxy Zoo sample from a redshift range $0.070 < z < 0.075$. The debiased disc vote fraction for each galaxy is shown. The scale for each image is $0.099~\rm{arcsec/pixel}$.}
\label{fig:mosaic}
\end{figure}

\begin{figure}[t]
\centering{
\includegraphics[width=\textwidth]{introduction/mag_redshift_completeness.pdf}}
\caption[GZ2-GALEX sample completeness]{Absolute $u$-band magnitude against redshift for the whole of SDSS (grey dashed lines) in comparison to the GZ2 subsample (blue solid lines). Typical Milky Way $L_*$ galaxies with $M_u \sim -20.5$ are still included in the GZ2 subsample out to the highest redshift. {\minor The numbers indicate the contour levels and the median error is shown in the top left (errors on $z$ are negligible).}}
\label{complete}
\end{figure}


I require a sample of galaxies with optical and NUV photometry from SDSS and GALEX respectively, along with morphologies from GZ2 in order to study the morphological dependence of galaxy quenching histories. The GZ2 sample consists of $304,022$ SDSS galaxies which were selected to include the brightest ($m_r < 17$), largest (radius containing $90\%$ of the Petrosian flux, $r_{90} > 3''$) and nearest ($0.005 < z < 0.25$) galaxies in order to achieve robust detailed morphological classifications. I first removed those objects considered to be stars or artefacts in Task 0 or merging pairs in Task 6 (using the thresholds defined in Table~\ref{table:votes}) from this GZ2 sample. Further to this, I required NUV photometry from the GALEX survey, within which $\sim42\%$ of the GZ2 sample galaxies were observed, giving a total sample size of $126, 316$ galaxies. This will be referred to as the \textsc{gz2-galex} sample\footnote{No attempt is made to remove unobscured Type 1 AGN from the \textsc{gz2-galex} sample. Unobscured AGN have characteristic colours in optical imaging bands, therefore the contribution of the AGN to the galaxy photometry leads to an inaccurate estimate of the SFH. However, this effect will be negligible across the \textsc{gz2-galex} sample given the expected small fraction of unobscured AGN.}. 

The \textsc{gz2-galex} sample is shown in Figure~\ref{complete} with the $u$-band absolute magnitude against redshift, compared with the SDSS data set. Despite the GZ2 selection for the brightest and largest galaxies and the cross match to GALEX (which has a higher magnitude limit than SDSS) typical Milky Way $L_*$ galaxies with $M_u \sim -20.5$ are still included in the GZ2 subsample out to the highest redshift of $z \sim 0.25$; however dwarf and lower mass galaxies are only detected at the lowest redshifts. The redshift is taken into account during the SFH modelling (see Section~\ref{qmod}). {\minor The median error on the colours of the sample are $(u-r) = 0.087$ and $(NUV-u)=0.16$ to $2$ significant figures.} 

Galaxy colours were not corrected for intrinsic dust attenuation. This is of particular consequence for disc galaxies, where attenuation increases with increasing inclination. \cite{Buat05} found the median value of the attenuation in the GALEX NUV passband to be $\sim 1$ mag. Similarly \cite{masters10a} found a total extinction from face-on to edge-on spirals of 0.7 and 0.5 mag for the SDSS $u$ and $r$ passbands and show spirals with $\log(a/b) > 0.7$ have signs of significant dust attenuation. For the \textsc{gz2-galex} sample I find $\sim10\%$ of discs (with $p_d > 0.5$) have $\log(a/b) > 0.7$, therefore we must be aware of possible biases in the results due to dust. Such biases will cause the SFH of a dusty galaxy to be inferred with less recent and faster quenching. 

{\minor We must also consider the caveat that the detected NUV flux may not be attributed to recent star formation from the emission of hot, young stars, but may also be attributed to blue horizontal branch stars; often referred to as the `UV upturn' \citep{schombert16}. This is particularly apparent in early-type galaxies which are dominated by old stellar populations. Unfortunately, the source of the NUV emission cannot be determined, therefore we must be aware of this caveat when analysing the results of this investigation.}

Galaxy stellar masses are estimated using the method outlined in \cite{Baldry06}, who  fit a relationship between the observed $u-r$ colour and the $r$-band mass-to-light ratio, $(M_*/L_r)$, of a galaxy as:
\begin{equation}\label{baldrymlur}
\log\left(\frac{M_*}{L_r}\right) =
\begin{cases}
-0.95 + 0.56(u-r) & \text{if } (u-r) < 2.1 \\
-0.16 + 0.18(u-r) & \text{if } (u-r) \geq 2.1. 
\end{cases}
\end{equation}
The $r$-band mass-to-light ratio is calculated from the $r$-band absolute magnitude, $\mathcal{M}_r$, of a galaxy as outlined in \cite{blanton01}:
\begin{equation}\label{blanton}
\log\left(\frac{M_*}{M_{\odot}}\right) = \left(\frac{\mathcal{M}_{r,\odot} - \mathcal{M}_r}{2.5}\right) + \log\left(\frac{M_*}{L_r}\right),
\end{equation}
where $\mathcal{M}_{r,\odot}$ and $M_{\odot}$ are the $r$-band absolute magnitude and stellar mass of the Sun respectively. 

I shall use the \textsc{gz2-galex} sample to probe how different quenching mechanisms cause galaxies of different morphologies and environments to transition from the SFS to quiescence. 


\section{Thesis Summary}\label{sec:thesissum}


This thesis proceeds as follows. In Chapter~\ref{chap:starpy} I describe the SFH model used to characterise the colours of quenching galaxies, along with the statistical methods used to determine the distribution of quenching histories in a population of galaxies. In Chapter~\ref{chap:morph} I apply this method across the red sequence, green valley and blue cloud and investigate the morphological dependence of quenching histories in these populations. Chapter~\ref{chap:agn} is split into two parts. In Section~\ref{sec:agnfeedback} I investigate the effect of AGN feedback on the quenching histories of a  population of AGN host galaxies. In Section~\ref{sec:intbulgeless} I investigate the proposed slow co-evolution of galaxies with their central black holes, by measuring the black hole masses of a sample of bulgeless galaxies, which have assumed merger free histories. In Chapter~\ref{chap:env} I return to investigating the quenching histories of galaxies, this time focussing on the effect of the group environment on satellite galaxies in comparison to centrals and those in the field. In Chapter~\ref{chap:discussion} I discuss how the implications of my results in the context of galaxy evolution and propose ideas for future work.

Where necessary I adopt the Planck 2015 cosmological results \citep{planck16} with $(\Omega_m, \Omega_{\lambda}, h) = (0.309 \pm 0.006, 0.691 \pm 0.006, 0.677 \pm 0.005)$. 

\chapter{STARPY: Bayesian inference of a galaxy's star formation history}

\emph{The work in the following chapter has been published in \citet{smethurst15}.}
\\

\section{Star Formation History Models}\label{qmod}

The quenched star formation history (SFH) of a galaxy can be simply modelled as an exponentially declining star formation rate (SFR) across cosmic time ($0 \leq t ~\rm{[Gyr]} \leq 13.8$) as:
\begin{equation}\label{sfh}
SFR =
\begin{cases}
I_{sfr}(t_q) & \text{if } t < t_q \\
I_{sfr}(t_q) \times exp{\left( \frac{-(t-t_{q})}{\tau}\right)} & \text{if } t > t_q 
\end{cases}
\end{equation}
where $t_{q}$ is the onset time of quenching, $\tau$ is the timescale over which the quenching occurs and $I_{sfr}$ is an initial constant star formation rate dependent on $t_q$.  A smaller $\tau$ value corresponds to a rapid quench, whereas a larger $\tau$ value corresponds to a slower quench. 

Here I assume that all galaxies formed at a time $t=0~\rm{Gyr}$ with an initial burst of star formation. The mass of this initial burst is controlled by the value of the $I_{sfr}$ which is set as the average specific SFR (sSFR) at the time of quenching $t_q$.  \citet{peng10} defined a relation (their equation 1) between the average sSFR and redshift (cosmic time, $t$) by fitting to measurements of the mean sSFR of blue star forming galaxies from SDSS, zCOSMOS and literature values at increasing redshifts \citep{Elbaz07, Daddi07}:
\begin{equation}
sSFR(m,t) = 2.5 \left( \frac{m}{10^{10} M_{\odot}} \right)^{-0.1} \left(\frac{t}{3.5 ~\rm{Gyr}}\right)^{-2.2} \rm{Gyr}^{-1}.
\end{equation}
Beyond $z \sim 2$ the characteristic SFR flattens and is roughly constant back to $z\sim6$. The cause for this change is not well understood but can be seen across similar observational data \citep{peng10, gonzalez10, bethermin12}. Motivated by these observations, the relation defined in \citet{peng10} is taken up to a cosmic time of $t=3~\rm{Gyr}~(z \sim 2.3)$ and prior to this a constant average SFR is assumed (see middle panel of Figure~\ref{sfr_mass_col}). At the point of quenching, $t_{q}$, the SFH models are defined to have an $I_{sfr}$ which lies on this relationship for the sSFR, for a galaxy with mass, $m = 10^{10.27} M_{\odot}$ (the mean mass of the \textsc{gz2-galex} sample; see left panel of Figure~\ref{sfr_mass_col}).

\begin{figure*}
\centering{
\includegraphics[width=\textwidth]{starpy/sfr_mass_colour_diagram.pdf}}
\caption[SFH models in observational planes]{Left panel: SFR-stellar mass plane for all 126,316 galaxies in the \textsc{gz2-galex} sample (shaded contours), with model galaxy trajectories shown by the coloured lines, with each point representing a time step of $0.5~\rm{Gyr}$.  The `main sequence' of star formation as defined by \citet{peng10} is shown by the solid line with $\pm1\sigma$ (dashed lines). Middle panel: The SFHs of the models are shown, where the SFR is initially constant before quenching at time $t_q$ and thereafter exponentially declining with a characteristic timescale $\tau$. The SFR at the point of quenching is set to be consistent with the typical SFR of a star-forming galaxy at the quenching time, $t_q$ (dashed curve; \citealt{peng10}). Right panel: The full range of models can reproduce the observed colour-colour properties of the sample; for clarity the figures show only 4 of the possible models explored in this study. Note that some of the model tracks produce colours redder than the apparent peak of the red sequence in the GZ2 subsample; however this is not the \emph{true} peak of the red sequence due to the necessity for NUV colours from GALEX.}
\label{sfr_mass_col}
\end{figure*}
  
Under these assumptions the average SFR of these models will result in a lower value than the relation defined in \citet{peng10} at all cosmic times as each galaxy only resides on the `main sequence' at the point of quenching. However galaxies cannot remain on the `main sequence' from early to late times throughout their entire lifetimes given the unphysical stellar masses and SFRs this would result in at the current epoch in the local Universe \citep{bethermin12, Heinis14}. If prescriptions for starbursts, mergers, AGN etc. were included in this model, the reproduction of the average SFR across cosmic time would improve; however I have chosen to first focus on the simplest possible model.

Once this evolutionary SFR is obtained, it is convolved with the \citet{BC03} population synthesis models to generate a model SED at each time step. The observed features of galaxy spectra can be modelled using simple stellar population techniques which sum the contributions of individual, coeval, equal-metallicity stars. The accuracy of these predictions depends on the completeness of the input stellar physics. Comprehensive knowledge is therefore required of (i) stellar evolutionary tracks and (ii) the initial mass function (IMF) to synthesise a stellar population accurately. 

These stellar population synthesis (SPS) models are an extremely well explored (and often debated) area of astrophysics \citep{Maraston05, Eminian08, CGW09, falkenberg09, Chen10, Kriek10, miner11, melbourne12}. In this work I have chosen to utilise the \citet{BC03} \emph{GALEXEV} SPS models, along with a Chabrier IMF \citep{chabrier03}, across a large wavelength range ($0.0091 < ~\lambda~\rm{[\mu m]}~ < 160 $) with solar metallically (m62 in the \citet{BC03} models; hereafter BC03), to allow a direct comparison with \citet{schawinski14}.


Fluxes from stars younger than $3~$Myr in the SPS model are suppressed to mimic the large optical depth of protostars embedded in dusty formation clouds (as in \citealt{schawinski14}).  Filter transmission curves are then applied to the fluxes to obtain AB magnitudes and ultimately colours.  For a particular galaxy at an observed redshift, $z$, I calculate the observed time, $t^{obs}$ for that galaxy using the standard cosmological conversion between redshift and time provided in the \textsc{astropy} {\em Python} module \citep{astropy13}. The predicted colours of the SFH models at the observed redshift of each individual galaxy can then be compared to the observed colours directly.

\begin{figure}
\centering{
\includegraphics[height=0.75\textheight]{starpy/colours.pdf}}
\caption[Predicted colours and SFRs of quenching models]{Quenching timescale $\tau$ versus quenching onset time $t_q$ in all three panels for the quenched SFH models used in \starpy. Colour shadings show model predictions of the $u-r$ optical colour (top panel), $NUV-u$ colour (middle panel), and star formation rate (lower panel), at $t^{obs} = 12.8~\rm{Gyr}$, the mean observed redshift of the GZ2 sample (see Section \ref{qmod}). The combination of optical and NUV colours is a sensitive measure of the $\theta = [t_q, \tau]$ parameter space. Note that all models with $t > 12.8$ \rm{Gyr} are effectively un-quenched. The `kink' in the bottom panel is due to the assumption that the sSFR is constant prior to $t \sim 3~\rm{Gyr}$ ($z\sim 2.2$).}
\label{pred}
\end{figure}

Figure~\ref{pred} shows these predicted optical and NUV colours at a time of $t^{obs} = 12.8 ~\rm{Gyr}$ (the average observed time of the \textsc{gz2-galex} sample, $z \sim 0.076$) for the exponential SFH model. These predicted colours will be referred to as $d_{c,p}(t_{q}, \tau, t^{obs})$, where $c$=\{opt,NUV\} and $p$ = predicted. The SFR at a time of $t^{obs}=12.8~\rm{Gyr}$ is also shown in Figure~\ref{pred} to compare how this correlates with the predicted colours. The $u-r$ predicted colour shows an immediate correlation with the SFR, however the $NUV-u$ colour is more sensitive to the value of $\tau$ and so is ideal for tracing any recent star formation in a population . At small $\tau$ (rapid quenching timescales) the $NUV-u$ colour is insensitive to $t_{q}$, whereas at large $\tau$ (slow quenching timescales) the colour is very sensitive to $t_{q}$. Together the two colours are ideal for tracing the effects of $t_{q}$ and $\tau$ in a population. 

This model is not a fully hydrodynamical simulation, it is a simple model built in order to test our understanding of the evolution of galaxy populations. These models are therefore not expected to accurately determine the SFH of every galaxy in the \textsc{gz2-galex} sample, in particular galaxies which have not undergone any quenching. In this case the models described above can only attribute a constant star formation rate to these  unquenched galaxies. In reality, there are many possible forms of SFH that a galaxy can take, a few of which have been investigated in previous literature; starbursts \citep{Canalizo01}, a power law \citep{Glazebrook03}, single stellar populations \citep{Trager00, Sanchez06, Vazdekis10}, log-normal distributions \citep{abramson16} and metallicity enrichment \citep{deLucia14}. Incorporating these different SFHs along with prescriptions for mergers and a possible reinvigoration of star formation post quench (e.g. see recent work by \citealt{pontzen16}) into the SFH models is a possible future extension to this work once the results of this study are well enough understood to permit additional complexity to be added.

\section{Probabilistic Fitting Methods}\label{stats}

In order to achieve robust conclusions I conducted a Bayesian analysis \citep{Sivia, mackay03} of the predicted colours from the SFH models in comparison to the observed colours of the \textsc{gz2-galex} sample. This approach requires consideration of all possible combinations of $\theta \equiv (t_{q}, \tau)$. Assuming that all galaxies formed at $t=0~\rm{Gyr}$ with an initial burst of star formation, we can assume that the `age' of each galaxy in the GZ2 sample is equivalent to an observed time, $t^{obs}_{k}$. I then used this  `age' to calculate the predicted model colours at this cosmic time for a given combination of $\theta$: $d_{c,p}(\theta_k, t^{obs}_{k})$ for both optical and NUV $(c={opt,NUV})$ colours. The predicted model colours can now directly be compared with the observed \textsc{gz2-galex} sample colours, so that for a single galaxy $k$ with optical ($u-r$) colour, $d_{opt, k}$ and NUV ($NUV-u$) colour, $d_{NUV,k}$, the likelihood of a given model $P(d_{k}|\theta_k, t^{obs}_{k})$ is:


\begin{equation}\label{like}
\begin{split}
P(d_{k}|\theta_k, t^{obs}_{k}) = \frac{1}{\sqrt{2\pi\sigma_{opt, k}^2}}\frac{1}{\sqrt{2\pi\sigma_{NUV, k}^2}} \exp{\left[ - \frac{(d_{opt, k} - d_{opt, p}(\theta_k, t_{k}^{obs}))^2}{\sigma_{opt, k}^2} \right]} \\ \exp{\left[ - \frac{(d_{NUV, k} - d_{NUV, p}(\theta_k, t_{k}^{obs}))^2}{\sigma_{NUV, k}^2} \right]}.
\end{split}
\end{equation}


Here I have assumed that $P(d_{opt}|\theta_k, t^{obs}_{k})$ and $P(d_{NUV}|\theta_k, t^{obs}_{k})$ are independent of each other and that the errors on the observed colours are also independent. To obtain the probability of a combination of $\theta$ values \underline{given} the GZ2 data: $P(\theta_k|d_k, t^{obs})$, i.e. how likely is a single SFH model given the observed colours of a single GZ2 galaxy, I utilise Bayes' theorem:
 \begin{equation}\label{big}
P(\theta_k|d_k, t^{obs}) = \frac{P(d_k|\theta_k, t^{obs})P(\theta_k)}{\int P(d_k |\theta_k, t^{obs})P(\theta_k) d\theta_k}.
\end{equation}
I assume a flat prior on the model parameters so that:
\begin{equation}\label{prior}
P(\theta_k) =
\begin{cases}
1 & \text{if } 0 \leq t_q ~\rm{[Gyr]}~ \leq 13.8 ~  \text{ and } ~ 0 \leq \tau  ~\rm{[Gyr]}~ \leq 4\\
0 & \text{otherwise.} \\
\end{cases}
\end{equation}

\begin{figure}
\centering{
\includegraphics[width=0.9\textwidth]
{starpy/triangle_t_tau_red_s_1237655504035185152_40000_14_16_06_08_14.pdf}
\caption[Example \starpy ~output]{Example output from \starpy ~for a galaxy within the red sequence. The contours show the positions of the `walkers' in the Markov Chain (which are analogous to the areas of high probability) for the quenching models described by $\theta = [t_q, \tau]$. The histograms show the 1D projection along each axis. Solid (dashed) blue lines show the best fit parameters (with $\pm 1\sigma$) to the data. The postage stamp image from SDSS is shown in the top right along with the debiased vote fractions for smooth ($p_s$) and disc ($p_d$) from Galaxy Zoo 2.} }
\label{one_example}
\end{figure}

As the denominator of Equation~\ref{big} is a normalisation factor, comparison between likelihoods for two different SFH models (i.e., two different combinations of $\theta_k = [t_q, \tau]$) is equivalent to a comparison of the numerators. Markov Chain Monte Carlo (MCMC; \citealt{mackay03, emcee13, GW10}) provides a robust comparison of the likelihoods between $\theta$ values; here I choose \emph{emcee},\footnote{\url{emcee13.iel.fm/emcee/}} a Python implementation of an affine invariant ensemble sampler by \cite{emcee13}.

This method allows for a more efficient exploration of the parameter space by avoiding those areas with low likelihood. A large number of `walkers' are started at an initial position where the likelihood is calculated; from there they individually `jump' to a new area of parameter space. If the likelihood in this new area is greater (less) than the original position then the `walkers' accept (reject) this change in position. Any new position then influences the direction of the  `jumps' of other walkers. This is repeated for the defined number of steps after an initial `burn-in' phase. \emph{emcee} returns the positions of these `walkers', which are analogous to the regions of high probability in the model parameter space. 

The routine outlined above has been coded using the \emph{Python} programming language into a package named \starpy ~which has been made freely available to download\footnote{\url{github.com/zooniverse/starpy}}. An example output from this module for a single galaxy from the \textsc{gz2-galex} sample in the red sequence is shown in Figure~\ref{one_example}. 

\section{Testing STARPY}

\begin{figure}
\centering{
\includegraphics[width=\textwidth]{starpy/mosaic_test.jpg}}
\caption[Testing \starpy]{Results from \starpy ~for an array of synthesised galaxies with known, i.e. \underline{true}, $t_q$ and $\tau$ values (marked by the red lines) using the complete function to calculate the predicted colour of a proposed set of $\theta$ values in each MCMC iteration, assuming an error on the calculated known colours of $\sigma_{u-r} = 0.124$ and $\sigma_{NUV-u} = 0.215$ (the average errors on the GZ sample colours). I also assume that each synthesised galaxy has been observed at a redshift of $z=0$. In each case \starpy ~succeeds (50th percentile best fit parameters are shown by the blue lines) in locating the true parameter values within the degeneracies of the star formation history model.}
\label{test_mosaic}
\end{figure}

In order to test that \starpy ~can find the correct quenching model for a given observed colour, 25 synthesised galaxies were created with known SFHs (i.e. known values of $\theta = [t_q, \tau]$) from which optical and NUV colours were generated using the BC03 SPS models. These were input into \starpy ~ to test whether the known values of $\theta$ were reproduced, within error, for each of the 25 synthesised galaxies. Figure~\ref{test_mosaic} shows the results for each of these synthesised galaxies, with the known values of $\theta$ shown by the red lines. In some cases this red line does not coincide with the inferred best fit $\theta$ values shown by the blue lines, however in all cases the intersection of the red lines is within the sample contours; therefore \starpy succeeds in locating the true parameter values within the degeneracies of the SFH model. 

\section{Speeding up STARPY}\label{lookuptable}

\begin{figure*}
\centering{
\includegraphics[width=0.49\textwidth]{starpy/corner_test_starfpy_full_sfh_function_0.pdf}
\includegraphics[width=0.49\textwidth]{starpy/corner_test_starfpy_lookup_0.pdf}}
\caption[Comparing complete and look-up table versions of \starpy]{Left panel: Results from \starpy ~for \underline{true} $t_q$ and $\tau$ values (red lines) using the complete function to calculate the predicted colour of a proposed set of $\theta$ values in each MCMC iteration. The median walker position (the 50th percentile of the Bayesian probability distribution) is shown by the solid blue line with the dashed lines encompassing $68\% (\pm 1\sigma)$ of the samples (the 16th and 84th percentile positions). The time taken to run for a single galaxy using this method is approximately 2 hours. Right panel: Results from \starpy ~for \underline{true} $t_q$ and $\tau$ values using a look up table generated from the complete function to calculate the predicted colour of a proposed set of $\theta$ values in each MCMC iteration. The time taken to run for a single galaxy using this method is approximately 2 minutes.}
\label{lookup}
\end{figure*}

\begin{table}
\centering{
\caption{Median walker positions (the 50th percentile; as shown by the blue solid lines in Figure~\ref{lookup}) found by \starpy ~ for a single galaxy, using the complete star formation history function and a look up table to speed up the run time. The errors quoted define the region in which $68\%$ of the samples are located, shown by the dashed blue lines in Figure~\ref{lookup}. The known true values are also quoted, as shown by the red lines in Figure~\ref{lookup}. All values are quoted to three significant figures.}
\begin{tabular*}{0.65\textwidth}{r @{\extracolsep{\fill}}ccc}
\multicolumn{1}{l}{} & \multicolumn{3}{c}{}                                          \\ \hline
                     & $t_q$                       & $\tau$                       &  \\ \hline
True                 & $4.37$                        & $2.12$                         &  \\
Complete             & $3.893 \pm^{3.014}_{2.622}$ & $2.215 \pm^{0.395}_{0.652}$ &  \\
Look up table        & $3.850 \pm^{2.988}_{2.619}$ & $2.218 \pm^{0.399}_{0.649}$ & \\ \hline
\end{tabular*}}
\label{median_lu}
\end{table}

I wish to consider the SFH model parameters for a large populations of galaxies across the colour magnitude diagram, however for each combination of $\theta$ values which \emph{emcee} proposes for a single galaxy, a new SFH must be built, prior to convolving it with the BC03 SPS models at the observed age and then predicted colours calculated from the resultant SED. For a single galaxy this takes up to 2 hours on a typical desktop machine for long Markov Chains. A 3-dimensional look-up table was therefore generated at $50 ~t^{obs}$, $100 ~t_{quench}$ and $100 ~\tau$ values; this was then interpolated over for a given observed galaxy's age and proposed $\theta$ values at each step in the Markov Chain. This ensured that a single galaxy takes approximately 2 minutes to run on a typical desktop machine. 

Figure~\ref{lookup} shows an example of how using the look up table in place of the full function does not affect the results to a significant level. Table~\ref{median_lu} quotes the median walker positions (the 50th percentile of the Bayesian probability distribution) along with their $\pm 1\sigma$ ranges for both methods in comparison to the true values specified to test \starpy. The uncertainties incorporated into the quoted values by using the look up table are therefore minimal with a maximum $\Delta = 0.043$.

Using this lookup table, each of the $126,316$ total galaxies in the \textsc{gz2-galex} sample was run through \starpy ~on multiple cores of a computer cluster to obtain the Markov Chain positions (analogous to $P(\theta_k|d_k)$) for each galaxy, $k$ (see Figure~\ref{one_example}). In each case the Markov Chain consisted of $100$ `walkers' which took $400$ steps in the `burn-in' phase and $400$ steps thereafter, at which point the MCMC acceptance fraction was checked to be within the range $0.25 < f_{acc} < 0.5$ (which was true in all cases). Due to the Bayesian nature of this method, a statistical test on the results is not possible; the output is probabilistic in nature across the entirety of the parameter space.

\section{POPSTARPY: studying populations of galaxies with STARPY}\label{popstarpy}

To study the SFH of a large population of galaxies, the individual galaxy walker positions output by \starpy ~(analogous to the posterior probability distribution) are combined across $[t, \tau]$ space. The Markov Chain walker positions are binned and weighted by their corresponding logarithmic posterior probability $\log [P(\theta_k|d_k)]$, provided by the \emph{emcee} package, in order to emphasise the features and differences between various populations. This weighting by $\log [P(\theta_k|d_k)]$ is to minimise the contribution of galaxies poorly fit by this exponentially declining SFH. This is no longer inference but merely a method to visualise the results across a population of galaxies.

I also discard those walker positions with a corresponding normalised posterior probability of $P(\theta_k|d_k) < 0.2$ in order to exclude galaxies which are not well fit by the quenching model, therefore galaxies in each sample which reside on the main sequence will not contribute to the final population distribution of quenching parameters. This raises the issue of whether I exclude a significant fraction of the \textsc{gz2-galex} sample and whether those galaxies reside in a specific location of the colour-magnitude. The fraction of galaxies which had all or more than half of their walker positions discarded due to low probability are shown in Table \ref{discardnum}. Using the $P(\theta_k|d_k) < 0.2$ constraint, $2.4\%$, $7.0\%$ and $5.4\%$ of green, red and blue galaxies respectively had \emph{all} of their walker positions discarded. 

This is not a significant fraction of either population, therefore the \starpy~ module is effective in fitting the majority of galaxies and this method of discarding walker positions ensures that poorly fit galaxies are removed from the analysis of the results. Figure \ref{discarded} shows that these galaxies with discarded walker positions are also scattered across the optical-NUV colour-colour diagram and therefore \starpy ~is also effective in fitting galaxies across this entire plane. 

\begin{table*}
\centering{
\caption{The number of galaxies in each population which had walker positions discarded due to low posterior probability values in order to exclude those galaxies from the analysis which were poorly fit by the SFH quenching model.}
\begin{tabular*}{0.95\textwidth}{p{4cm} @{\extracolsep{\fill}} ccc}
                                          & \textbf{Red Sequence}                                   & \textbf{Green Valley}                                  & \textbf{Blue Cloud}                                      \\ \hline
All walkers discarded                     & \begin{tabular}[c]{@{}c@{}}1420\\ (7.00\%)\end{tabular} & \begin{tabular}[c]{@{}c@{}}437\\ (2.41\%)\end{tabular} & \begin{tabular}[c]{@{}c@{}}3109\\ (5.37\%)\end{tabular}  \\
More than half walker positions discarded & \begin{tabular}[c]{@{}c@{}}2010\\ (9.92\%)\end{tabular} & \begin{tabular}[c]{@{}c@{}}779\\ (4.30\%)\end{tabular} & \begin{tabular}[c]{@{}c@{}}6669\\ (11.52\%)\end{tabular} \\ \hline
\end{tabular*}}
\label{discardnum}
\end{table*}

\begin{figure}
\includegraphics[width=0.9\textwidth]{starpy/discarded_galaxy_colour_colour.pdf}
\caption[Colours of discarded galaxies]{Contours show the full GZ2 subsample optical-NUV colour-colour diagram. The points show the positions of the galaxies which had all (top panels) or more than half (bottom panel) of their walker positions discarded due to their low probability for the red sequence (left), green valley (middle) and blue cloud (right).}
\label{discarded}
\end{figure}

Figure ~\ref{test_mosaic} shows how peaks in the histograms are found across all areas of the parameter space in both dimensions $[t, \tau]$, ensuring that any conclusions drawn from combined population distributions are due to a superposition of extended probability distributions, as opposed to a bimodal distribution of probability distributions across all galaxies.

The classifications from Galaxy Zoo 2 provide a uniquely powerful continuous measurements of a galaxy's morphology, therefore I utilise the debiased user vote fractions to obtain separate population density distributions for both smooth and disc galaxies. This is obtained by also weighting by the morphology vote fraction when the binned walker positions are combined. This ensures that the entirety of the population is used, with galaxies with a higher $p_d$ contributing more to the disc weighted than the smooth weighted population distribution. This negates the need for a threshold on the GZ2 vote fractions \citep[e.g., $p_d > 0.8$ as used in][]{schawinski14}. These distributions will be referred to as the population densities.

For example, the galaxy shown in Figure~\ref{one_example} would contribute almost evenly to both the smooth and disc parameters due to the GZ2 vote fractions. Since galaxies with similar vote fractions contain both a bulge and disc component, this method is effective in incorporating intermediate galaxies which are thought to be crucial to the morphological changes between early- and late-type galaxies. It was the consideration of these intermediate galaxies which was excluded from the investigation by \citet{schawinski14}.

\subsection{Alternative Hierarchical Bayesian approach}\label{althyper}

The approach presented above relies upon a visualisation of the SFHs across each population, with no inference involved beyond the use of \textsc{starpy} to derive the individual galaxy SFHs. An alternative approach to this problem would be to use a hierarchical Bayesian method to determine the `hyper-parameters' that describe the distribution of the parent population $\theta' = [t_q', \tau']$ that each individual galaxy's SFH is drawn from. 

The posterior PDF for $\vec{\theta}'$ to describe such a galaxy population:
\begin{equation}\label{hyper}
P(\vec{\theta}'|\vec{d}) = \frac{P(\vec{d}|\vec{\theta}')P(\vec{\theta}')}{P(\vec{d})}, 
\end{equation}
where $\vec{d}$ represents all of the optical and NUV colour data in a population $\{\vec{d}_k\}$. For one galaxy, $k$, the marginalised likelihood is:
\begin{equation}\label{one}
P(d_k|\vec{\theta}') = \iint \! P(d_k|t_k, \tau_k) P(t_k, \tau_k|\vec{\theta}') \ \mathrm{d}t_k ~ \mathrm{d}\tau_k
\end{equation}
and for all galaxies, $N$, therefore: 
\begin{equation}
P(\vec{d}|\vec{\theta}') = \prod_k^N P(d_k|\vec{\theta}').
\end{equation}

Using \textsc{starpy}~ for an individual galaxy, $k$ the output is the `interim' posterior $P(t_k, \tau_k|d_k)$ which I can relate to $P(d_k|t_k, \tau_k)$  so that:

\begin{equation}\label{marg}
P(d_k|\vec{\theta}') = \iint  \! P(t_k, \tau_k|d_k) . P(d_k) . \frac{P(t_k, \tau_k|\vec{\theta}')}{P(t_k, \tau_k)} \ \mathrm{d}t_k ~ \mathrm{d}\tau_k.
\end{equation}
In order to calculate this I draw $N_s$ random samples, $r$, from each interim posterior, $P(t_k, \tau_k|d_k)$ so that Equation \ref{marg} can be expressed as a sum over a number of random samples, $N_s$ (as with the calculation of an expected mean):
\begin{equation}\label{imp}
P(d_k|\vec{\theta}') = \frac{P(d_k)}{N_s} \sum_r^{N_s} \frac{P(t_{k,r}, \tau_{k,r}|\vec{\theta}')}{P(t_k, \tau_k)},
\end{equation}
for the $r^{th}$ sample of $N_s$ total samples taken from one galaxy's, $k$,  interim posterior PDF. This fraction is known as the `importance weight', $w_r$, in importance sampling. 

However, I also have two morphological vote fractions that I can weight by to determine separate hyper-parameters, $\vec{\theta}' = [\vec{\theta}'_d, \vec{\theta}'_s]$, for both disc, $d$, and smooth, $s$, galaxies. Therefore:

\begin{equation}\label{morphimp}
w_r = \frac{P(t_{k,r}, \tau_{k,r}|\vec{\theta}')}{P(t_k, \tau_k)} =  \frac{p_{d,k} P(t_{k,r}, \tau_{k,r}|\vec{\theta}'_d) + p_{s,k} P(t_{k,r}, \tau_{k,r}|\vec{\theta}'_s)}{P(t_k, \tau_k)}
\end{equation} 

If we substitute equation \ref{imp} into equation \ref{hyper} we find that the $P(d_k)$ terms cancel and we are left with:
\begin{equation}
P(\vec{\theta}'|\vec{d}) = P(\vec{\theta}')~\prod_k^N \frac{1}{N_{s,k}} \sum_r^{N_s} w_r ,
\end{equation}
where $P(\vec{\theta}')$ is the assumed prior on the hyper-parameters, which is assumed to be uniform.

\begin{figure}
\begin{centering}
\includegraphics[width=0.8\textwidth]{starpy/figc2b.pdf}
\caption[Replica colour-colour distributions using a hierarchical method]{Optical-NUV colour-colour diagrams for the \textsc{inactive} galaxies shown by the black contours, split into low mass (top), medium mass (middle) and high mass (bottom) galaxies weighted by $p_d$ (left) and $p_s$ (right). Kernel smoothing has been applied to the overlaid replica datasets, which are created by sampling from the \textbf{inferred 2 component Gaussian mixture model hierarchical parent distributions}. Gaussian random noise is also added to the inferred colours, with a mean and standard deviation of the errors on the observed colours of the respective sample. Contours are shown for samples taken from the disc (blue) and smooth weighted (red) inferred hierarchical distributions.}
\label{replica}
\end{centering}
\end{figure}

\begin{figure}
\begin{centering}
\includegraphics[width=0.8\textwidth]{starpy/figc3b.pdf}
\caption[Replica colour-colour distributions using the \textsc{popstarpy} method]{Optical-NUV colour-colour diagrams for the \textsc{inactive} galaxies shown by the black contours, split into low mass (top), medium mass (middle) and high mass (bottom) galaxies weighted by $p_d$ (left) and $p_s$ (right). Kernel smoothing has been applied to the overlaid replica datasets, which are created by sampling from the \textbf{\textsc{popstarpy} population density distributions described in Section~\ref{popstarpy}}. Gaussian random noise is also added to the inferred colours, with a mean and standard deviation of the errors on the observed colours of the respective sample. Contours are shown for samples taken from the disc (blue) and smooth weighted (red) inferred hierarchical distributions.}
\label{replicapop}
\end{centering}
\end{figure}

This approach is heavily dependent on what shape is assumed for the hyper-distribution; a decision which is not trivial. It is often common for this function to take the form of a multi-component Gaussian mixture model \citep{mackay03, lahav00}. For example a two component Gaussian mixture model in $[t, \tau]$ space is described by eight hyper-parameters for a single morphology, $\vec{\theta}' = [\mu_{t,1}, \sigma_{t,1}, \mu_{\tau,1}, \sigma_{\tau,1}, \mu_{t,2}, \sigma_{t,2}, \mu_{\tau,2}, \sigma_{\tau,2}]$. This approach assumes no covariance between hyper-parameters for simplicity. The equations outlined above, combined with MCMC methods can be used to infer these $8 \vec{\theta}'$ parameters from which the hierarchical population distribution can be determined. 

%I used this assumption of a two component Gaussian mixture model, to infer the population parameters for both the \textsc{agn-host} and \textsc{inactive} populations and the results are shown in Figure~\ref{method3}. These results were produced by drawing $N_s = 100$ random samples from each galaxy, $k$, in each mass bin. I plot the distributions for a given morphology by taking the median value of the posterior distribution for each of the 8 parameters describing the two component Gaussian mixture. I can see in Figure~\ref{method3} that this hierarchical method produces similar distributions for the \textsc{agn-host} and \textsc{inactive} samples. This finding is not expected given the differences between the two samples in colour-colour space seen in Figure~\ref{colcol}. 

%\begin{figure}
%\includegraphics[width=0.48\textwidth]{figc1a.pdf}
%\includegraphics[width=0.48\textwidth]{figc1b.pdf}
%\caption[8pt]{Hierarchical-posterior PDF of the quenching time ($t_q'$, top) and rate ($\tau'$, bottom) population parameters, normalised so that the areas under the curves are equal. \textsc{agn-host} (left) and \textsc{inactive} (right) galaxies are split into low (top), medium (middle) and high (bottom) mass, weighted for smooth (red dashed) and disc (blue solid) galaxies. A low (high) value of $t_q'$ corresponds to the early (recent) Universe. A small (large) value of $\tau'$ corresponds to a rapid (slow) quench.}
%\label{method3}
%\end{figure}

In order to test whether this assumption of a multi-component Gaussian mixture model is appropriate, I sampled the inferred hierarchical distributions to produce replica datasets in optical-NUV colour space. These are shown here in Figure~\ref{replica}  in comparison to the observed colour-colour distributions of the \textsc{inactive} sample (a subset of $\sim6,000$ galaxies from the \textsc{gz2-galex} sample, see Section~\ref{agnfeedback}). For all masses and morphologies the replicated $u-r$ and $NUV-u$ colours do not accurately match the observed data. 


I also varied the value of $N_s$ and found that increasing the number of samples drawn did not improve this fit for the \textsc{inactive} population. Similarly increasing the number of components in the Gaussian mixture model did not immediately improve the accuracy of the fit.  I therefore concluded that this functional form of the population distribution was unsatisfactory. %An extensive exploration of a wide variety of functional forms is necessary to ensure the correct conclusions are drawn from the data. Such an investigation is beyond the scope of this paper. 

The \textsc{popstarpy} approach described in section \ref{popstarpy} was motivated by the investigation increasing the number of samples, $N_s$ drawn from the posterior of each galaxy, k, until the point where all the samples were drawn. Instead of attempting to infer parameters to describe this distribution, as above, I presented the distribution itself (as described in Section \ref{popstarpy}).  The distributions produced by this visualisation method reveal the complexity that the parent distribution must describe which, as concluded earlier, cannot be effectively modelled.

I also tested whether the \textsc{popstarpy} method is reasonable by producing replica datasets in optical-NUV colour space, as before, by drawing $1000$ $[t, \tau]$ values from the population density distributions derived for the \textsc{inactive} sample (see Section~\ref{agnfeedback}). These replica datasets are shown here in Figure~\ref{replicapop} in comparison to the observed colour-colour distributions of the \textsc{inactive} sample. Comparing these replica colours in Figure~\ref{replicapop}, with those produced by drawing from the inferred hierarchical distributions, shown in Figure~\ref{replica}, they can be seen to produce a more accurate match to the observed data for the majority of masses and morphologies. 

Considering these issues with assuming a functional form for the hierarchical parent distribution, an expansion on this approach would be to perform `heat map optimization', similar to image reconstruction, to determine the parent distribution for a given population. Each pixel would need a prior (e.g. a basic entropic prior) and the heat map would sum to unity. This is a significant expansion upon the work presented here and is something the author wishes to investigate in future work.

For the results presented in the following chapters, I therefore use the \textsc{popstarpy} method to visualise the population distribution, rather than quoting inferred values to describe it.



\chapter{The morphological dependance of quenching}\label{morph}

\emph{The work in the following chapter has been published in \citet{smethurst15}.}


By studying the galaxies which  have just left the mass-SFR relation (see top panel of Figure \ref{sfr_mass_col}),  the quenching mechanisms by which this occurs and its morphological dependance can be probed. By investigating the \emph{amount} of quenching that has occurred in the blue cloud, green valley and red sequence; and by comparing that amount across the three populations, constraints can be applied to the many possible quenching theories outlined in Chapter \ref{intro}. 

I have been motivated by a recent result suggesting there are two contrasting evolutionary pathways through the green valley by different morphological types (\citealt{schawinski14}, hereafter S14). Specifically that late-type galaxies quench very slowly and form a nearly static disc population in the green valley, whereas early-type galaxies quench very rapidly, transitioning through the green valley and onto the red sequence in $\sim 1$~Gyr \citep{Wong12}. That study used a toy model to examine quenching across the green valley but did not use statistics to support their conclusions. Here I use the same toy model but implement \starpy ~in order to statistically study the star formation histories of galaxies across the colour magnitude diagram.


\section{Defining the Green Valley}\label{defGV}

To define which of the $126, 316$ galaxies of the \textsc{gz2-galex} sample are in the green valley, I looked to previous definitions in the literature defining the separation between the red sequence and blue cloud. For example, \citet{Baldry04} used a large sample of $66,846$ local galaxies ($0.004 < z < 0.08$) from the SDSS to trace this bimodality by fitting double Gaussians to the colour magnitude diagram without cuts in morphology. Their relation between the $u-r$ colour, $C'_{ur}$, and r-band magnitude, $M_r$, to define the colour cut between the blue and red galaxy populations is defined in their Equation 11 as:
\begin{equation}\label{eqgv}
C'_{ur}(M_{r}) = 2.06 - 0.244 \tanh \left( \frac{M_r + 20.07}{1.09}\right).
\end{equation}

Due to the necessity for NUV photometry in this study, matching to GALEX removed typical `red and dead' galaxies from the \textsc{gz2-galex} sample. It is therefore not appropriate to define the green valley by a visual fit to the  colour magnitude diagram as in S14. For example, the optical $u-r$ colour histograms shown in Figure \ref{fig:cmgvsplit}, split both a complete SDSS sample and the \textsc{gz2-galex} sample in bins of their absolute r-band magnitude and in each case show the position of the green valley at that $M_r$ as defined by \citet{Baldry04}. For the \textsc{gz2-galex} sample at brighter r-band magnitudes (i.e. larger mass), this definition of the green valley seems to intersect with the observed peak at red colours. However, for the larger SDSS sample (from the MPA-JHU catalog; \citealt{kauffmann03, brinchmann04}) this green valley definition does not intersect with the peak at red colours, as this sample is complete, containing the high mass typical red sequence galaxies. Using a visual fit to the green valley of the \textsc{gz2-galex} sample colour-magnitude diagram would therefore cause green valley galaxies to be misclassified as red sequence.


\begin{figure}
\centering{
\includegraphics[width=0.49\textwidth]{morphology/sdss_hist_slice.pdf}
\includegraphics[width=0.49\textwidth]{morphology/galzoo_hist_slice.pdf}}
\caption[Optical $u-r$ colour histograms in absolute r-band magnitude slices of the \textsc{gz2-galex} and Baldry et al. (2004) complete SDSS samples]{Optical $u-r$ colour histograms, sliced in absolute r-band magnitude for a complete SDSS sample (MPA-JHU catalog; left) and for the \textsc{gz2-galex} sample (right). In each panel the definition between the blue cloud and the red sequence from \citet{Baldry04} is shown by the dashed line (as defined in Equation~\ref{eqgv}); the solid lines show $\pm 1\sigma$ either side of this definition.}
\label{fig:cmgvsplit}
\end{figure}

I therefore adopt the \citet{Baldry04} green valley definition for this study which is shown in Figure~\ref{fig:CMGV} by the dashed line in comparison to both the \textsc{gz2-galex} sample (left) and the SDSS data used for the fit by \cite[][right]{Baldry04}. Any galaxy within $\pm 1\sigma$ of this relationship, shown by the solid lines in Figure~\ref{fig:CMGV}, is therefore considered a green valley galaxy. Only $47\%$ of the red sequence galaxies present in the entire Galaxy Zoo 2 sample are present in the \textsc{gz2-galex} sample, as opposed to $72\%$ of the blue cloud and $53\%$ of the green valley galaxies. 

However, although the galaxies identified as residing on the red sequence within the \textsc{gz2-galex} sample have NUV detections, this does not mean they are not representative of a typical red sequence `red and dead' galaxy. \cite{ko13} show that in a sample of quiescent red-sequence galaxies without $\mathrm{H}\alpha$ emission (i.e. without spectral indication of recent star formation), $26\%$ show NUV excess emission and that the fraction with recent star formation is $39\%$. Of the $48\%$ of \textsc{gz2-galex} galaxies classified as below the main sequence using the definition from \citet[][see Section~\ref{qmod}]{peng10}, $44\%$ of these galaxies lie on the red sequence ($94\%$ of all the red sequence galaxies; see Table ~\ref{table:qsubs}). I am therefore confident that the galaxies within the \textsc{gz2-galex} sample will be representative of galaxies across the colour magnitude diagram.

\begin{figure*}
\centering{
\includegraphics[width=\textwidth]{morphology/col_mag_GV_Baldry_data.pdf}}
\caption[Colour-magnitude diagram showing the location of the Baldry et al. (2004) green valley definition]{Colour-magnitude diagram for the \textsc{gz2-galex} sample (left) and the SDSS sample from \citet[][right]{Baldry04}. In both panels the definition between the blue cloud and the red sequence from \citet{Baldry04} is shown by the dashed line, as defined in Equation~\ref{eqgv}. The solid lines show $\pm 1\sigma$ either side of this definition; any galaxy within the boundary of these two solid lines is considered a green valley galaxy. The lack of red sequence galaxies due to the necessity for NUV GALEX colours skews the apparent location of the green valley in the \textsc{gz2-galex} sample, therefore a literature definition of the green valley is used to ensure galaxies are correctly classified.}
\label{fig:CMGV}
\end{figure*}

The decomposition of the \textsc{gz2-galex} sample into red sequence, green valley and blue cloud galaxies is shown in Tables~\ref{table:subs} and \ref{table:qsubs} along with further subsections by galaxy type and SFR (where available for the \textsc{gz2-galex} sample from the MPA-JHU catalog) respectively. The tables also list the definitions I adopt henceforth for early-type ($p_s~ \geq~0.8$), late-type ($p_d~ \geq~0.8$), smooth-like ($p_s~ >~0.5$), disc-like ($p_d~ >~0.5$), quenched ($\rm{SFR}$ $ < P - 5\sigma$), quenching ($P - 5\sigma < \rm{SFR}$ $< P - \sigma$) and star forming  ($\rm{SFR}$ $> P -\sigma$) galaxies, where $P$ is the SFR as defined by \citet{peng10} for a given stellar mass and observed time (see Equation \ref{eq:peng}). 

\begin{table}
\caption{Table showing the decomposition of the \textsc{gz2-galex} sample by galaxy type into the subsets of the colour-magnitude diagram.}
\begin{tabular*}{\textwidth}{l @{\extracolsep{\fill}}cccc}
\hline
\begin{tabular}[c]{@{}c@{}} {\color{white} -} \\ {\color{white} -}  \end{tabular} & All                                                      & Red Sequence                                              & Green Valley                                              & Blue Cloud \\  \hline 
Smooth-like ($p_s > 0.5$)        & \begin{tabular}[c]{@{}c@{}}42453\\ (33.6\%)\end{tabular} & \begin{tabular}[c]{@{}c@{}}17424\\ (61.9\%)\end{tabular}  & \begin{tabular}[c]{@{}c@{}}10687\\ (44.6\%)\end{tabular}   & \begin{tabular}[c]{@{}c@{}}14342\\ (19.3\%)\end{tabular}  \\ 
Disc-like ($p_d > 0.5$)          & \begin{tabular}[c]{@{}c@{}}83863\\ (80.7\%)\end{tabular} & \begin{tabular}[c]{@{}c@{}}10722\\ (38.1\%)\end{tabular}   & \begin{tabular}[c]{@{}c@{}}13257\\ (55.4\%)\end{tabular}  & \begin{tabular}[c]{@{}c@{}}59884\\ (47.4\%)\end{tabular}  \\
Early-type ($p_s \geq 0.8$) & \begin{tabular}[c]{@{}c@{}}10517\\ (8.3\%)\end{tabular}  & \begin{tabular}[c]{@{}c@{}}5337\\ (18.9\%)\end{tabular}    & \begin{tabular}[c]{@{}c@{}}2496\\ (10.4\%)\end{tabular}    & \begin{tabular}[c]{@{}c@{}}2684\\ (3.6\%)\end{tabular}    \\
Late-type ($p_s \geq 0.8$)  & \begin{tabular}[c]{@{}c@{}}51470\\ (40.9\%)\end{tabular} & \begin{tabular}[c]{@{}c@{}}4493\\ (15.9\%)\end{tabular}    & \begin{tabular}[c]{@{}c@{}}6817\\ (28.5\%)\end{tabular}    & \begin{tabular}[c]{@{}c@{}}40430\\ (54.4\%)\end{tabular}  \\ \hline
\textbf{Total}                       & \begin{tabular}[c]{@{}c@{}}\textbf{126316} \\ (100.0\%)\end{tabular}                                                & \begin{tabular}[c]{@{}c@{}}28146 \\ (22.3\%)\end{tabular} & \begin{tabular}[c]{@{}c@{}}23944 \\ (18.9\%)\end{tabular} & \begin{tabular}[c]{@{}c@{}}74226 \\ (58.7\%)\end{tabular} \\\hline
\end{tabular*}
\label{table:subs}
\end{table}


\begin{table}
\caption{Table showing the decomposition of the \textsc{gz2-galex} sample by their star formation rate in the subsets of the colour-magnitude diagram.}
\begin{tabular*}{\textwidth}{l @{\extracolsep{\fill}}cccc}
\hline
\begin{tabular}[c]{@{}c@{}} {\color{white} -} \\ {\color{white} -}  \end{tabular} 		& All                                                      						& Red Sequence                                              			& Green Valley                                             			 & Blue Cloud \\  \hline 
\begin{tabular}[l]{@{}l@{}}Quenched\\ ($\rm{SFR} < P - 5\sigma$) \end{tabular}	& \begin{tabular}[c]{@{}c@{}}24278\\ (19.7\%)\end{tabular} 			& \begin{tabular}[c]{@{}c@{}}17018\\ (60.9\%)\end{tabular}    & \begin{tabular}[c]{@{}c@{}}6440\\ (27.5\%)\end{tabular}    & \begin{tabular}[c]{@{}c@{}}820\\ (1.1\%)\end{tabular}  \\ 
\begin{tabular}[l]{@{}l@{}}Quenching\\ ($P - 5\sigma < \rm{SFR} < P - \sigma$) \end{tabular}	 & \begin{tabular}[c]{@{}c@{}}34743\\ (28.2\%)\end{tabular}			 & \begin{tabular}[c]{@{}c@{}}9277\\ (33.1\%)\end{tabular}    & \begin{tabular}[c]{@{}c@{}}12181\\ (51.9\%)\end{tabular}    & \begin{tabular}[c]{@{}c@{}}13285\\ (18.6\%)\end{tabular}  \\ 
\begin{tabular}[l]{@{}l@{}}Star Forming  \\ ($\rm{SFR} > P -\sigma$) \end{tabular} & \begin{tabular}[c]{@{}c@{}}63957\\ (52.0\%)\end{tabular} 			& \begin{tabular}[c]{@{}c@{}}1665 \\ (5.9\%)\end{tabular}    & \begin{tabular}[c]{@{}c@{}}4828\\ (20.6\%)\end{tabular}    & \begin{tabular}[c]{@{}c@{}}57464\\ (80.3\%)\end{tabular}  \\ \hline
\textbf{Total}                       		& \begin{tabular}[c]{@{}c@{}}\textbf{122,978} \\ (100.0\%)\end{tabular} & \begin{tabular}[c]{@{}c@{}}27960 \\ (22.7\%)\end{tabular} & \begin{tabular}[c]{@{}c@{}}23449 \\ (19.1\%)\end{tabular} & \begin{tabular}[c]{@{}c@{}}71569 \\ (58.2\%)\end{tabular} \\\hline
\end{tabular*}
\label{table:qsubs}
\end{table}

Figure~\ref{sfr_mass_sub} displays the information shown in Tables \ref{table:subs} \& \ref{table:qsubs} with the SFR against the stellar mass for the \textsc{gz2-galex} sample (where available from the MPA-JHU catalog) by splitting it into blue cloud, green valley red sequence, late- and early-type populations. This figure confirms that the green valley galaxies in the \textsc{gz2-galex} sample are indeed a population which have either left, or begun to leave, the star forming sequence or have some residual star formation still occurring. Figure~\ref{sfr_mass_sub} also reveals that the early-type galaxies of the \textsc{gz2-galex} sample seem to dominate the low and high mass ends of the star forming sequence (shown by the blue solid lines in each panel).  


\begin{figure*}
\centering{
\includegraphics[width=\textwidth]{morphology/sfr_mass_subsets.pdf}}
\caption[SFR-stellar mass plane split by morphology and colour]{Star formation rate against stellar mass for the different populations of galaxies (top row, left to right: all galaxies, late-type galaxies, early-type galaxies; bottom row, left to right: blue cloud, green valley and red sequence galaxies) and how they contribute to the star forming sequence (from \citet{peng10}, shown by the solid blue line with 0.3 dex scatter by the dashed lines). Based on positions in these diagrams, the green valley does appear to be a transitional population between the blue cloud and the red sequence. Detailed analysis of star formation histories can elucidate the nature of the different populations' pathways through the green valley.}
\label{sfr_mass_sub}
\end{figure*}


\section{Results}

The \starpy ~package was run on all galaxies in the \textsc{gz2-galex} sample and the \textsc{popstarpy} method outlined in Section~\ref{popstarpy} was used  to produce Figures~\ref{red_s},~\ref{green_v} \&~\ref{blue_c} for both smooth and disc weighted populations in the red sequence, green valley and blue cloud. The percentages shown in Figures~\ref{red_s},~\ref{green_v} \&~\ref{blue_c} are calculated as the fractions of the population densities located in each region of parameter space for a given population. 

Since the sample contains such a large number of galaxies, these fractions are interpreted as broadly equivalent to the percentage of galaxies in a given population undergoing quenching within the stated timescale range. Although this is not quantitatively exact, it is nevertheless a useful framework for interpreting the population densities.

Also shown in Figure~\ref{fig:bestfit} is the distribution of the median walker positions (the 50th percentile of the posterior probability distribution) of each individual galaxy, split into red, green and blue disc-like ($p_d > 0.5$) and smooth-like ($p_s > 0.5$) populations. These positions were calculated without discarding any walker positions due to low probability and without weighting by the GZ2 morphological vote fractions; therefore this may be more intuitive to understand than Figures~\ref{red_s},~\ref{green_v} \&~\ref{blue_c}.

Although the quenching timescales are continuous in nature, in this Section I refer to rapid, intermediate and slow quenching timescales which correspond to ranges of  $\tau ~\rm{[Gyr]} < 1.0$, $1.0 < \tau ~\rm{[Gyr]} < 2.0$ and $\tau ~\rm{[Gyr]} > 2.0$ respectively for ease of discussion.



\subsection{The Red Sample}\label{rs}

\begin{figure*}
\centering{
\includegraphics[width=0.55\textwidth]{morphology/red_smooth.pdf}\\
\includegraphics[width=0.55\textwidth]{morphology/red_disc.pdf}}
\caption[Population densities of red smooth and disc galaxies]{Contour plots showing the population densities for red galaxies of the \textsc{gz2-galex} sample, weighted by the morphological vote fractions from GZ2 to give both bulge (top) and disc (bottom) dominated distributions. The histograms show the projection into one dimension for each parameter. The dashed lines show the separation between rapid ($\tau ~\rm{[Gyr]} < 1.0$), intermediate ($1.0 < \tau ~\rm{[Gyr]} < 2.0$) and slow ($\tau ~\rm{[Gyr]} > 2.0$) quenching timescales with the fraction of the combined posterior probability distribution in each region shown (see Section~\ref{stats}).}
\label{red_s}
\end{figure*}

The top panel of Figure~\ref{red_s} reveals that smooth galaxies with red optical colours show a preference $(49.5\%$; see Figure~\ref{red_s}) for rapid quenching timescales across all cosmic time resulting in a very low current SFR. For these smooth red galaxies at early times only, a preference for slow and intermediate timescales in the top panel of Figure~\ref{red_s} is seen. Perhaps this is the influence of intermediate galaxies (with $p_s \sim p_d \sim 0.5$), hence why similar high probability areas exist for both the smooth weighted and disc weighted populations in the top and bottom panels of Figure~\ref{red_s}. This is especially apparent considering there are far more of these intermediate galaxies than those that are definitively early- or late-types (see Table~\ref{table:subs}). These galaxies are those whose morphology cannot be easily distinguished either because they are at a large distance or because they are an S0 galaxy whose morphology can be interpreted by different GZ2 users in different ways. \citet{GZ2} find that S0 galaxies expertly classified by \citet{nair10} are more commonly classified as ellipticals by GZ2 users, but have a significant tail to high disc vote fractions, giving a possible explanation as to the origin of this area of probability.

The bottom panel of Figure~\ref{red_s} reveals that red disc galaxies show similar preferences for rapid $(31.3\%)$ and slow $(44.1\%)$ quenching timescales. The preference for \emph{very} slow ($\tau > 3.0 ~\rm{Gyr}$) quenching timescales (which are not seen in either the green valley or blue cloud, see Figures~\ref{green_v} and~\ref{blue_c}) suggests that these  galaxies have only just reached the red sequence after a very slow evolution across the colour-magnitude diagram. Considering their limited number and the requirement for NUV emission, it is likely that these galaxies are currently on the edge of the red sequence having recently (and finally) moved out of the green valley. Table~\ref{table:subs} shows that $3.9\%$ of the sample are red sequence late-type galaxies, i.e. red late-type spirals. This is, within uncertainties, in agreement with the findings of \citet{masters10c}, who find $\sim6\%$ of late-type spirals are red when defined by a cut in the $g-r$ optical colour (rather than with $u-r$ as used in this investigation) and are at the `blue end of the red sequence'. 

Despite the dominance of slow quenching timescales, the red disc weighted population also show some preference for rapid quenching timescales ($31.3\%$), similar to the red smooth weighted population but with a lower likelihood. Perhaps these rapid quenching timescales can also be attributed to a morphological change, suggesting that the quenching has occurred more rapidly than the morphological change to a bulge dominated system.

Comparing the resultant SFRs for both the smooth and disc weighted populations in Figure~\ref{red_s} by noticing where the areas of high probability lie with respect to the bottom panel of Figure~\ref{pred} (which shows the predicted SFR at an observation time of $t\sim12.8~\rm{Gyr}$, the average `observed' time of the \textsc{gz2-galex} sample) reveals that red disc weighted population with a preference for slow quenching still have some residual star formation occurring, SFR$~\sim0.105 M_{\odot}yr^{-1}$, whereas the smooth galaxies with a dominant preference for rapid quenching have a resultant SFR$~\sim0.0075 M_{\odot}yr^{-1}$. This is approximately 14 times less than the residual SFR still occurring in the red disc weighted population. Within error, this is in agreement with the findings of \citet{tojeiro13} who, by using the VErsatile SPectral Analyses spectral fitting code (VESPA; \citealt{tojeiro07}), found that red late-type spirals show 17 times more recent star formation than red elliptical galaxies.

These results for the red galaxies investigated here with NUV emission, have many implications for green valley galaxies, as all of these systems must have passed through the green valley on their way to the red sequence. 

\subsection{Green Valley Galaxies}\label{gv}

\begin{figure*}
\centering{
\includegraphics[width=0.55\textwidth]{morphology/green_smooth.pdf}\\
\includegraphics[width=0.55\textwidth]{morphology/green_disc.pdf}}
\caption[Population densities of green smooth and disc galaxies]{Contour plots showing the population densities for green valley galaxies in the \textsc{gz2-galex} sample weighted by the morphological vote fractions from GZ2 to give both bulge (top) and disc (bottom) dominated distributions. The histograms show the projection into one dimension for each parameter. The dashed lines show the separation between rapid ($\tau ~\rm{[Gyr]} < 1.0$), intermediate ($1.0 < \tau ~\rm{[Gyr]} < 2.0$) and slow ($\tau ~\rm{[Gyr]} > 2.0$) quenching timescales with the fraction of the combined posterior probability distribution in each region shown (see Section~\ref{stats}).}
\label{green_v}
\end{figure*}

In Figure~\ref{green_v} similar comparisons for the green valley galaxies can be made to those discussed previously for the red galaxies studied. For the red galaxies, an argument can be made for two possible tracks across the green valley, shown by the bimodal nature of both distributions in $\tau$, with a common area in the intermediate timescale region where the rapid and slow timescales peaked distributions intersect. However in the green valley this intermediate quenching timescale region becomes more significant  (in agreement with the conclusions of \citealt{Gonc12}), particularly for the smooth weighted population (see the top panel of Figure~\ref{green_v}).

The smooth weighted population densities favour these intermediate quenching timescales ($40.6\%$) with some preference for slow quenching at  early times ($z > 1$). The preference for rapid quenching in the smooth population has dropped by over a half compared to the red population, however this will be influenced by the observability of galaxies undergoing such a rapid quench which will spend significantly less time in the transitional population of the green valley. To quantify this, I tested the time spent in the green valley across the $[t, \tau]$ parameter space, which is shown in Figure~\ref{fig:timeingv}. Those galaxies with such a rapid decline in star formation pass so quickly through the green valley and so will be detected at a lower number than those galaxies which have stalled in the green valley with intermediate quenching timescales.  This accounts for the observed number of intermediate galaxies which are present in the green valley and the dominance of rapid timescales detected for the red populations of both morphologies.

\begin{figure*}
\centering{
\includegraphics[width=0.9\textwidth]{morphology/green_valley_time.pdf}}
\caption[Time spent in the green valley across parameter space]{Plot showing the time spent in the green valley across the SFH model parameter space. This affects the observability of those galaxies which have quenched rapidly and recently and have passed too quickly through the green valley to be detected. The white region denotes those models with colours that do not enter the green valley by the present cosmic time.}
\label{fig:timeingv}
\end{figure*}

The green valley disc weighted population now overwhelmingly prefer slow quenching timescales ($47.4\%$) with a similar amount of intermediate quenching compared to the smooth weighted parameters ($37.6\%$; see Figure~\ref{green_v}). There is still some preference for galaxies with a star formation history which results in a high current SFR, suggesting there are also some late-type galaxies that have just progressed from the blue cloud into the green valley. 

If Figure~\ref{green_v} to Figure~\ref{red_s} are compared, quenching is seen to occur at later (more recent) cosmic times in the green valley than for red galaxies of both morphological types. Therefore both morphologies are tracing the evolution of the red sequence, confirming that the green valley is indeed a transitional population between blue cloud and red sequence regardless of morphology. Currently as the green valley is observed, its main constituents are very slowly evolving disc-like galaxies along with intermediate- and smooth-like galaxies which pass across it with intermediate timescales within $\sim 1.0-1.5~\rm{Gyr}$.

Given enough time ($t\sim4 - 5~\rm{Gyr}$), the disc galaxies will eventually fully pass through the green valley and make it out to the red sequence (the bottom panel of Figure~\ref{sfr_mass_col} shows galaxies with $\tau > 1.0~\rm{Gyr}$ do not approach the red sequence within $3~\rm{Gyr}$ post quench). This is most likely the origin of the `red spirals'.

Considering that the green valley is a transitional population, the ratio of smooth:disc galaxies that is currently observed in the green valley is expected to evolve into the ratio observed for the red galaxies with NUV emission investigated. Table~\ref{table:subs} shows the ratio of smooth-like : disc-like galaxies in the observed red sequence of the \textsc{gz2-galex} sample is $62:38$ whereas in the green valley it is $45:55$. Making the very simple assumptions that this ratio does not change with redshift and that quenching is the only mechanism which causes a morphological transformation, then $31.2\%$ of the disc-dominated galaxies currently residing in the green valley would have to undergo a morphological change to a bulge-dominated galaxy. The fraction of the density of the green valley disc weighted population occupying the parameter space $\tau < 1.5 ~\rm{Gyr}$ is $29.4\%$, this suggests that quenching mechanisms with these timescales are capable of destroying the disc-dominated structure of galaxies. However this is most likely an overestimate of the timescale that can cause a morphological change because of the observability of those galaxies which undergo such a rapid quench; \citet[][and see Figure~\ref{fig:timing}]{Martin07} showed that after considering the time spent in the green valley, the fraction of galaxies undergoing a rapid quench quadruples.


All of this evidence suggests that there are not just two routes for galaxies through the green valley as concluded by S14, but a continuum of quenching timescales which can be divided into three general regimes: rapid ($\tau < 1.0 ~\rm{Gyr}$), intermediate ($1.0 < \tau < 2.0~\rm{Gyr}$) and slow ($\tau > 2.0~\rm{Gyr}$). The intermediate quenching timescales reside in the space between the extremes sampled by the UV/optical diagrams of S14; the inclusion of the intermediate galaxies in this investigation (unlike in S14) and the more precise Bayesian analysis, quantifies this range of $\tau$ and specifically ties the intermediate timescales to all variations of galaxy morphology.

Instead of concluding that the green valley is a \emph{`red herring'} as in S14, one might instead conclude that the \emph{`grass is always redder on the other side'}.


\subsection{Blue Cloud Galaxies}\label{bc}

\begin{figure*}
\centering{
\includegraphics[width=0.55\textwidth]{morphology/blue_smooth.pdf}\\
\includegraphics[width=0.55\textwidth]{morphology/blue_disc.pdf}}
\caption[Population densities of blue smooth and disc galaxies]{Contour plots showing the population densities blue cloud galaxies in the \textsc{gz2-galex} sample, weighted by the morphological vote fractions from GZ2 to give both bulge (top) and disc (bottom) dominated distributions. The histograms show the projection into one dimension for each parameter. The dashed lines show the separation between rapid ($\tau ~\rm{[Gyr]} < 1.0$), intermediate ($1.0 < \tau ~\rm{[Gyr]} < 2.0$) and slow ($\tau ~\rm{[Gyr]} > 2.0$) quenching timescales with the fraction of the combined posterior probability distribution in each region shown (see Section~\ref{stats}). Positions with probabilities less than 0.2 are discarded as poorly fit models, therefore unsurprisingly blue cloud galaxies are not well described by a quenching star formation model. }
\label{blue_c}
\end{figure*}

\begin{figure*}
\includegraphics[width=0.95\textwidth]{morphology/contour_t_tau_mcmc_bestfit.pdf}
\caption[Best fit contours for red, green and blue clean galaxies]{Contours showing the positions in the $[t, \tau]$ parameter space of the median walker position (the 50th percentile; as shown by the intersection of the solid blue lines in Figure~\ref{one_example}) for each galaxy for all (top), disc ($p_d > 0.5$; middle), and smooth ($p_s > 0.5$; bottom) red sequence, green valley and blue cloud galaxies in the left, middle and bottom panels respectively. The error bars on each panel shows the average $68\%$ confidence on the median positions (calculated from the 16th and 84th percentile, as shown by the blue dashed lines in Figure~\ref{one_example}). These positions were calculated without discarding any walker positions due to low probability and without weighting by vote fractions, therefore this plot may be more intuitive than Figures~\ref{red_s},~\ref{green_v} \&~\ref{blue_c}. The differences between the smooth and disc populations and between the red, green and blue populations remain clearly apparent.}
\label{fig:bestfit}
\end{figure*}

Since the blue cloud is considered to be primarily made of star forming galaxies \starpy~ is expected to have some difficulty in determining the most likely quenching model to describe them, as confirmed by Figure~\ref{blue_c}. The attempt to characterise a star forming galaxy with a quenched SFH model leads \starpy~ to attribute the extremely blue colours of the majority of these galaxies with a constant SFR for as long as possible leading to fast quenching at the observed redshift (i.e. the colour has not had enough time to change from blue post quench; see the bottom panel of Figure~\ref{blue_c} in comparison with the bottom panel of Figure~\ref{pred}).

This is particularly apparent for the blue disc weighted population. Perhaps even galaxies which are currently quenching slowly across the blue cloud cannot be well fit by the quenching models implemented, as they still have high SFRs despite some quenching (by definition although a galaxy is undergoing quenching, star formation can still be occurring in a galaxy, just at a slower rate than at earlier times, described by $\tau$).


There is a very small preference for the blue smooth weighted population to undergo slow quenching which began prior to $z \sim 0.5 $. These populations have been blue for a considerable period of time, slowly using up their gas for star formation by the Kennicutt$-$Schmidt law \citep{Schmidt59, Kennicutt97}. However the major preference is for rapid quenching at recent times in the blue cloud; this therefore provides some support to the theories for blue ellipticals as either merger-driven ($\sim76\%$; like those identified as recently quenched ellipticals with properties consistent with a merger origin by \citealt{McIntosh14}) or gas inflow-driven reinvigorated star formation that is now slowly decreasing ($\sim24\%$; such as the population of blue spheroidal galaxies studied by \citealt{Kaviraj13}). However, remember that the quenching models used in this work do not provide an adequate fit to the blue cloud population.

The blue cloud is therefore primarily composed of both star forming galaxies of all morphologies and a smooth population which are undergoing a rapid quench, presumably after a previous event triggered star formation and turned them blue.


\section{Discussion}\label{morph:discussion}

In the previous section I presented the results of using \starpy ~to derive the most likely quenching histories for galaxies across the colour magnitude diagram and found differences between the SFHs of smooth- and disc-weighted populations of the red sequence, green valley and blue cloud. In this section I will speculate on the following question: what are the possible mechanisms driving these differences? 

\subsection{Rapid Quenching Mechanisms}\label{rapid}

Rapid quenching is much more prevalent in smooth galaxies than disc galaxies, and red galaxies are also much more likely to be characterised by a rapid quenching model than green valley galaxies (disregarding blue cloud galaxies due to their apparent poor fit by the quenching models, see Figure~\ref{blue_c}). In the green valley there is also a distinct lack of preference for rapid quenching timescales with $\tau < 0.5~\rm{Gyr}$; however the observability of a rapid quenching history declining with decreasing $\tau$ must be taken into account. Rapid mechanisms may be more common in the green valley than seen in Figure \ref{green_v}, however this observability should not depend on morphology so the conclusion that rapid quenching mechanisms are detected more for smooth rather than disc galaxies still holds. 

This suggests that rapid quenching mechanisms can cause a change in morphology from a disc- to a smooth dominated galaxy as it quickly traverses the colour-magnitude diagram to the red sequence. This is supported by the number of disc galaxies that would need to undergo a morphological change in order for the disc : smooth ratio of galaxies in the green valley to match that of the red galaxies in the \textsc{gz2-galex} sample (see Section~\ref{gv}). From this indirect evidence I suggest that this observed rapid quenching mechanism is caused by major mergers.

Inspection of the galaxies contributing to this area of population density reveals that this does not arise due to \emph{currently} merging pairs missed by GZ users which were not excluded from the \textsc{gz2-galex} sample (see Section \ref{class}), but by typical smooth galaxies with both red optical and NUV colours that \starpy~ attributes to rapid quenching at early times. Although a prescription for modelling a merger in the SFH is not included in this work the after effects can still be detected (see Section \ref{future} for future work planned with \starpy).

One simulation of interest by \citet{springel05b} showed that feedback from black hole activity is a necessary component of destructive major mergers to produce such rapid quenching timescales. Powerful quasar outflows can remove much of the gas from the inner regions of the galaxy, terminating star formation on extremely short timescales. \citet{Bell06}, using data from the COMBO-17 redshift survey ($0.4 < z < 0.8$), estimate a merger timescale from being classified as a close galaxy pair to recognisably disturbed as $\sim 0.4~\rm{Gyr}$. \citet{springel05b} consequently find using hydrodynamical simulations that after $\sim1~\rm{Gyr}$ the merger remnant has reddened to $u-r \sim 2.0$. This is in agreement with the simple exponential quenching models used here which show (Figure~\ref{sfr_mass_col}) that the models with a SFH with $\tau < 0.4~\rm{Gyr}$ have reached the red sequence, with $u-r ~\gtrsim 2.2$, within $\sim1~\rm{Gyr}$. This could explain the preference for red disc galaxies with rapid quenching timescales ($31.3\%$), as they may have undergone a major merger recently but are still undergoing a morphological change from disc, to disturbed, to an eventual smooth galaxy (see also \citealt{vdW09}) or have retained their disc structure in a merger in line with recent simulations from \citet{pontzen16}. This possible connection between AGN feedback and rapid quenching timescales is explored further in Chapter~\ref{agnfeedback}. 

This rapid quenching mechanism occurs much more rarely in green valley galaxies of both morphologies than for the subset of red sequence galaxies studied, however does not fully characterise all the galaxies in either the red sequence or green valley. Dry major mergers therefore do not fully account for the formation of any galaxy type at any redshift, supporting the observational conclusions made by \citet{Bell07,Bundy07, kaviraj14a} and simulations by \citet{Genel08}. 

\subsection{Intermediate Quenching Mechanisms}\label{int}

Intermediate quenching timescales are found to be equally prevalent across  all populations of both smooth and disc galaxies across cosmic time,  particularly in the green valley. Intermediate timescales are the prevalent mechanism for quenching smooth green valley galaxies, unlike the rapid quenching prevalent for red galaxies. This suggests that this intermediate quenching route must therefore be possible with routes that both preserve and transform morphology. It is this result of another route through the green valley that is in contradiction with the findings of S14. 

Once again considering the simulations of \citet{springel05b}, this time without any feedback from black holes, they suggest that if even a small fraction of gas is not consumed in the starburst following a merger (either because the mass ratio is not large enough or from the lack of strong black hole activity) the remnant can sustain star formation for periods of several Gyrs. The remnants from these simulations take $\sim5.5~\rm{Gyr}$ to reach red optical colours of $u-r \sim 2.1$. In Figure~\ref{sfr_mass_col} it can be seen that the models with intermediate quenching timescales of $1.0 \lesssim ~\tau~\rm{[Gyr]} ~\lesssim 2.0$ take approximately $2.5-5.5~\rm{Gyr}$ to reach these red colours. Similarly the recent simulations of \citet{pontzen16} show that without feedback the SFR of a disc galaxy can recover post merger and remain a star forming disc. 

I propose that the intermediate quenching timescales are caused by gas rich major mergers, major mergers without black hole feedback and from minor mergers, the latter of which being the dominant mechanism. This is supported by the findings of \citet{lotz08b} who find that the detectability timescales for equal mass gas rich mergers with large initial separations range from $\sim 1.1-1.9~\rm{Gyr}$, and of \citet{Lotz11}, who find in further simulations that as the baryonic gas fraction in a merger with mass ratios of 1:1-1:4 increases, so does the timescale of the merger from $\sim0.2~\rm{Gyr}$ (with little gas, as above for major mergers causing rapid quenching timescales) up to $\sim1.5~\rm{Gyr}$ (with large gas fractions). 

Here the very basic assumption is that the morphologically detectable timescale of a merger is roughly the same order as the quenching timescale. However, the existence of a substantial population of blue ellipticals \citep{Sch09} must be considered, which are thought to be post-merger systems with no detectable morphological signatures of a merger but with the merger-induced starburst still detectable in the photometry. This photometry is an indicator for the SFH and therefore should present with longer timescales for the photometric effects of a merger than found in the simulations by \citet{lotz08b} and \citet{Lotz11}. Observing this link between the timescale for the morphological observability of a merger and the timescales for the star formation induced by a merger is problematic, as evidenced by the lack of literature on the subject.

\citet{lotz08b} also show that the remnants of these simulated equal mass gas rich disc mergers (wet disc mergers) are observable for $\gtrsim1~\rm{Gyr}$ post merger and state that they appear ``disc-like and dusty" in the simulations, which is consistent with an ``early-type spiral morphology".  Such galaxies are often observed to have spiral features with a dominant bulge, suggesting that such galaxies may divide the votes of the GZ2 users, producing vote fractions of $p_s \sim p_d \sim 0.5$. This may be why the intermediate quenching timescales are equally dominant for both smooth and disc populations in Figures~\ref{red_s} and~\ref{green_v}. 

Other simulations (e.g. such as \citet{robertson06} and \citet{Barnes02}) support the conclusion that both gas rich major mergers and minor mergers can produce disc-like remnants. Observationally, \citet{Darg10a} showed an increase in the spiral to elliptical ratio for merging galaxies ($0.005 < z < 0.1$) by a factor of two compared to the general population. They attribute this to the much longer timescales during which mergers of spirals are observable compared to mergers with elliptical galaxies, confirming the hypothesis that the quenching timescales $\tau < 1.5 ~\rm{Gyr}$ preferred by disc galaxies may be undergoing mergers which will eventually lead to a morphological change. Similarly, \citet{Casteels13} observe that galaxies ($0.01 < z < 0.09$) which are interacting often retain their spiral structures and that a spiral galaxy which has been classified as having `loose winding arms' by the GZ2 users are often entering the early stages of mergers and interactions.

$40.6\%$ of the probability for smooth galaxies in the green valley arises due to intermediate quenching timescales (see Figure~\ref{green_v}); this is in agreement with work done by \citet{kaviraj14a, kaviraj14b} who by studying SDSS photometry ($z<0.07$) state that approximately half of the star formation in galaxies is driven by minor mergers at $0.5 < z < 0.7$ therefore exhausting available gas for star formation and consequently causing a gradual decline in the star formation rate. This supports earlier work by \cite{kaviraj11} who, using multi wavelength photometry of galaxies in COSMOS \citep{Scoville07}, found that $70\%$ of early-type galaxies appear morphologically disturbed, suggesting either a minor or major merger in their history. This is in agreement with the total percentage of the population density with $\tau < 2.0 ~\rm[Gyr]$; $73.9\%$ and $59.3\%$, for the smooth red and green galaxies in Figures~\ref{red_s} and~\ref{green_v} respectively. Note that the star formation model used here is a basic one and has no prescription for reignition of star formation post-quench which can also cause morphological disturbance of a galaxy, like those detected by \cite{kaviraj11} and seen in simulations by \cite{pontzen16}.

\citet{Darg10a} show in their Figure 6 that that beyond a merger ratio of $1:10$ (up to $\sim 1:100$), green is the dominant average galaxy colour of the visually identified merging pair in GZ. These mergers are also dominated by spiral-spiral mergers as opposed to elliptical-elliptical and elliptical-spiral. This supports the hypothesis that these intermediate timescales dominating in the green valley are caused in part by minor mergers. However this is contradictory to the findings of \citet{Mendez11} who find the merger fraction in the green valley is much lower than in the blue cloud, however they use an analytical light decomposition indicator to identify their mergers ($Gini/M_{20}$; see \citealt{lotz08b}), which tends to detect major mergers more easily than minor mergers. I have discussed the lower likelihood of a green valley galaxy to undergo a rapid quench, which I have hypothesised are attributed to major mergers (see Section \ref{rapid}), despite the caveat of the observability and believe that this may have been the phenomenon that \citet{Mendez11} detected.

The resultant intermediate quenching timescales occur initially due to one interaction mechanism, unlike the rapid quenching, which occurs due to a major merger combined with AGN feedback, and decreases the SFR over a short period of time. Therefore any external event which can cause either a burst of star formation (depleting the gas available) or directly strip a galaxy of its gas, for example galaxy harassment, interactions, ram pressure stripping, strangulation and interactions internal to clusters, would cause quenching on an intermediate timescale. Such mechanisms would be the dominant cause of quenching in dense environments; considering that the majority of galaxies reside in groups or clusters (\citealt{Coil08} find that green valley galaxies are just as clustered as red sequence galaxies). It is not surprising therefore that the majority of the \textsc{gz2-galex} galaxies are considered intermediate in morphology (see Table~\ref{table:subs}) and therefore are undergoing or have undergone such an interaction. This obvious dependancy of the quenching parameters on the galaxy environment will be investigated further in Chapter~\ref{chap:env}.


\subsection{Slow Quenching Timescales}\label{slow}
Although intermediate and rapid quenching timescales are the dominant mechanisms across the colour-magnitude diagram, together they cannot completely account for the quenching of disc galaxies. S14 concluded that slow quenching timescales were the most dominant mechanism for disc galaxies. However I show that: (i) intermediate quenching timescales are equally important in the green valley and (ii) rapid quenching timescales are equally important for the red galaxies with NUV emission. There is also a significantly lower preference for smooth galaxies to undergo such slow quenching timescales; suggesting that the evolution (or indeed creation) of typical smooth galaxies is dominated by processes external to the galaxy. This is excepting galaxies in the blue cloud where a small amount of slow evolution of blue ellipticals is occurring, presumably after a reinvigoration of star formation which is slowly depleting the gas available according to the Kennicutt$-$Schmidt law.

\citet{Bamford09} using GZ1 vote fractions of galaxies in the SDSS, found a significant fraction of high stellar mass red spiral galaxies in the field. As these galaxies are isolated from the effects of interactions from other galaxies, the slow quenching mechanisms present in their preferred star formation histories are most likely due to secular processes (i.e. mechanisms internal to the galaxy, in the absence of sudden accretion or merger events; \citealt{kormendy04, Sheth12}). Bar formation in a disc galaxy is such a mechanism, whereby gas is funnelled to the centre of the galaxy by the bar over long timescales where it is used for star formation \citep{masters12a, saintonge12, Cheung13}, consequently forming a `pseudo-bulge' \citep{Kormendy10, Simmons13}.

If these slow quenching timescales are due to secular evolution processes, this is to be expected since these processes do not change the disc dominated nature of a galaxy. 

\section{Conclusions}\label{morph:conc}

I have used morphological classifications from the Galaxy Zoo 2 project to determine the morphology-dependent star formation histories of galaxies via a Bayesian analysis of an exponentially declining star formation quenching model. The most likely parameters were determined for the quenching onset time, $t_q$ and quenching timescale $\tau$ in this model for galaxies across the blue cloud, green valley and red sequence to trace the morphological dependance of galactic evolution across the colour-magnitude diagram. The green valley is indeed found to be a transitional population for all morphological types (in agreement with \citet{schawinski14}), however this transition proceeds slowly for the majority of disc dominated galaxies and occurs rapidly for the majority of smooth dominated galaxies in the red sequence. However, in addition to \citet{schawinski14}, this Bayesian approach has revealed a more nuanced result, specifically that the prevailing mechanism across all morphologies and populations is quenching with intermediate timescales. The main findings are summarised as follows:
\begin{enumerate}[(i)]
\item The subset of red sequence galaxies with NUV emission studied in this investigation are found to have similar preferences for quenching timescales compared to green valley galaxies, but the quenching occurs at earlier quenching times (i.e. higher redshift) regardless of morphology (see Figures~\ref{red_s} and~\ref{green_v}). Therefore the quenching mechanisms currently occurring in the green valley were also active in creating the `blue end of of the red sequence' at earlier times; confirming that the green valley is indeed a transitional population, regardless of morphology.

\item The typical red galaxy with NUV emission studied, is elliptical in morphology and has undergone a rapid to intermediate quench at some point in cosmic time, resulting in a very low current SFR (see Section~\ref{rs}.

\item The green valley as it is currently observed is dominated by very slowly evolving disc dominated galaxies along with intermediate- and smooth dominated galaxies which pass across it with intermediate timescales within $\sim 1.0-1.5~\rm{Gyr}$ (see Section~\ref{gv}).

\item There are many different mechanisms responsible for quenching, all causing a galaxy to progress through the green valley, which are dependant on galaxy type, with the smooth and disc dominated galaxies each having different dominant star formation histories across the colour-magnitude diagram. These timescales can be roughly split into three main regimes; rapid ($\tau < 1.0~$Gyr), intermediate ($1.0 < \tau~$[Gyr]~$< 2.0$) and slow ($\tau > 2.0~$ Gyr) quenching.

\item Blue cloud galaxies are not well fit by a quenching model of star formation due to the continuous high star formation rates occurring (see Figure~\ref{blue_c}).

\item Rapid quenching timescales are detected with a lower probability for green valley galaxies than the red sequence galaxies studied. I speculate that this quenching mechanism is caused by major mergers with black hole feedback, which are able to expel the remaining gas not initially exhausted in the merger-induced starburst and which can cause a change in morphology from disc- to bulge-dominated. The colour-change timescales from previous simulations of such events agree with the derived timescales (see Section~\ref{rapid}). These rapid timescales are instrumental in forming red galaxies, however galaxies at the current epoch passing through the green valley do so at more intermediate timescales (see Figure~\ref{green_v}).

\item Intermediate quenching timescales ($1.0 < ~\tau~\rm{[Gyr]}~ < 2.0 $) are found with constant density across red and green galaxies for both smooth- and disc-weighted populations, the timescales for which agree with observed and simulated minor merger timescales (see Section~\ref{int}). I hypothesise that such timescales can be caused by a number of external processes, including gas rich major mergers, mergers without black hole feedback, galaxy harassment, interactions and ram pressure stripping. The timescales and observed morphologies from previous studies agree with the results, including that this is the dominant mechanisms for intermediate galaxies such as early-type spiral galaxies with spiral features but a dominant bulge, which split the GZ2 vote fractions (see Section~\ref{int}). 

\item Slow quenching timescales are the most dominant mechanism in the disc galaxy populations across the colour-magnitude diagram. Disc galaxies are often found in the field, therefore I hypothesise that such slow quenching timescales are caused by secular evolution and processes internal to the galaxy (see Section \ref{bc}). A small amount of slow quenching timescales is also detected for blue elliptical galaxies which is attributed to a reinvigoration of star formation, the peak of which has passed and has started to decline by slowly depleting the gas available (see Section~\ref{bc}). 
\end{enumerate}


\chapter{Black hole-galaxy co-evolution in the context of quenching}\label{chap:agn}

The following chapter is split into two parts; in Section \ref{sec:agnfeedback} I investigate the connection between quenching parameters and the presence of an AGN and then follwing up on these results in Section \ref{sec:intbulgeless} I investigate how black holes grow in disc galaxies with merger free evolutionary histories. 

\section{Rapid, recent quenching within a population of Type 2 AGN host galaxies}\label{sec:agnfeedback}

\emph{The work in the following chapter has been published in \citet{smethurst16}.}

In Chapter~\ref{chap:morph}, rapid quenching rates were dominant across the smooth weighted population in Figures~\ref{green_v} \& \ref{red_s}. In Section~\ref{rapid} I discussed how simulations suggest that such rapid quenching rates can only be achieved if AGN feedback is present in a major merger scenario. I therefore decided to investigate this possible connection between AGN feedback and rapid quenching rates further. I shall do so by analysing the SFHs of a population of AGN host galaxies with \textsc{popstarpy} in comparison to an inactive galaxy control sample. I  aim to determine the following: (i) Have galaxies which currently host an AGN undergone quenching? (ii) If so, when and at what rate does this quenching occur? (iii) Is this quenching occurring at different times and rates compared to a control sample of inactive galaxies? This investigation builds on the work of \citet{Martin07}, who also investigate the quenching rates of AGN with spectral SF indicators but improves significantly on their statistical techniques.

\subsection{AGN Sample}\label{agnsample}

\begin{figure*}
\includegraphics[width=\textwidth]{agn/fig2.pdf}
\caption[BPT diagram used to select AGN host galaxies]{BPT diagrams for galaxies in the \textsc{gz2-galex} sample (black crosses) with S/N $> 3$ for each emission line. Inequalities defined in: \protect\cite{kewley01} to separate SF galaxies from AGN (dashed lines), \protect\cite{kauffmann03b} to separate SF from composite SF-AGN galaxies (solid line) and \protect\cite{kewley06} to separate LINERS and Seyferts (dotted lines).  Galaxies are included in the \textsc{agn-host} sample (red circles) if they satisfy all the inequalities to be classified as Seyferts. LINERs are excluded to ensure a pure sample of AGN.}
\label{bpt}
\end{figure*}

\begin{figure*}
\includegraphics[width=\textwidth]{agn/agn-host_distributions_loiii_edd_ratio_mbh.pdf}
\caption[Distribution of measured galaxy parameters in the \textsc{agn-host} sample]{Distribution of the [OIII] luminosity (left), Eddington ratio (middle) and black hole masses (right) in the \textsc{agn-host} sample.}
\label{fig:agndistributions}
\end{figure*}


Obscured type 2 AGN were selected from the \textsc{gz2-galex} sample using a BPT diagram \citep*{bpt} using line and continuum strengths for [OIII], [NII], [SII] and [OII] obtained from the MPA-JHU catalogue \citep{kauffmann03, brinchmann04}. A BPT diagram uses emission line ratio diagnostics to determine whether a galaxy is a star forming galaxy, a Seyfert (i.e. hosting an AGN) or a LINER (a low-ionization nuclear emission-line region galaxy). This is possible because lines such as [NII] are forbidden transitions, and therefore only excited by the highest energy photons. Therefore an AGN, which is giving off higher energy photons than even the most massive stars in a galaxy, will produce a higher [NII]/H$\alpha$ ratio. 

In order to select a sample of AGN using a BPT diagram I use the following constraints. The signal-to-noise ratio of each emission line of the \textsc{gz2-galex} sample galaxies was required to be S/N $> 3$ as in \cite{schawinski10a}. Those galaxies which satisfied all of the inequalities (derived from theoretical spectral modelling of star forming galaxies and AGN) defined in \citet[][to separate SF galaxies from AGN]{kewley01} and \citet[][to separate SF galaxies from composite SF-AGN galaxies]{kauffmann03b} were selected as Type 2 AGN, giving $1,299$ host galaxies ($\sim10\%$ of the \textsc{gz2-galex} sample; in agreement with estimates of the local AGN fraction by \citealt{kauffmann04, pimbblet13}).

\cite{Sarzi10}, \cite{yan12} and \cite{Singh13} have all demonstrated that LINERs are not primarily powered by AGN, therefore to ensure a pure sample of AGN, these galaxies were excluded from the sample using the definition from \cite[][$55$ galaxies total]{kewley06} with no significant change to the results. The remaining $1,244$ galaxies will hereafter be referred to as the \textsc{agn-host} sample; Figure~\ref{bpt} shows the entire \textsc{agn-host} and \textsc{gz2-galex} samples with the selection criteria used on a BPT diagram.

I do not use Type 1 AGN in this investigation due to concerns about contamination of the observed galaxy colours, used in the SFH analysis, from potentially strong NUV emission by unobscured active nuclei. The obscuration of Type 2 AGN is highly efficient, considerably more so in the NUV than the optical \citep{Simmons11}. Residual NUV flux from a Type 2 AGN can therefore be neglected in comparison to that of the galaxy. However, I did investigate the possibility of contamination of optical galaxy colours from unobscured AGN emission in the \textsc{agn-host} sample and found that subtracting measured nuclear magnitudes (SDSS {\tt psfMag}) from the total galaxy magnitude (SDSS {\tt modelMag}) produces a negligible change in host galaxy colour ($\Delta(u-r) \sim 0.09$). I therefore use the total galaxy magnitudes (with extinction corrections as described in Section~\ref{sec:defsample}) to avoid unnecessary complexity and minimise the propagation of uncertainty from the observed colours through to the inferred SFHs. However, I note that using these corrected colours in the analysis does not change the results.

\begin{figure*}
\includegraphics[width=\textwidth]{agn/fig1.pdf}
\caption[SDSS images of galaxies in the \textsc{agn-host} sample]{Randomly selected SDSS \emph{gri} band composite images from the sample of $1,244$ Type 2 AGN in a redshift range $0.04 < z < 0.05$.  The galaxies are ordered from least to most featured according to their debiased `disc or featured' vote fraction, $p_d$ (see \citealt{GZ2}). The scale for each image is $0.099~\rm{arcsec/pixel}$.}
\label{mosaic}
\end{figure*}

Since this investigation is focussed on whether an AGN can have an impact on the star formation of its host galaxy, possible selection effects must be considered. The extent to which star formation could obscure AGN emission was addressed by \cite{schawinski10a}. They showed, by simulating the addition of AGN emission of varying luminosities to a star-forming galaxy spectra, that BPT-based selection of AGN produces a complete sample at luminosities of $L[OIII] > 10^{40}~\rm{erg~s}^{-1}$. $\sim98\%$ of the \textsc{agn-host} sample have an $L[OIII]$ above this limit, therefore I therefore assume I have selected a complete sample of Type 2 AGN independent of host galaxy SFR. 

Black hole masses of the \textsc{agn-host} sample are derived from the $M_{BH}-\sigma$ relationship whereby black hole masses are tightly correlated with the galaxy's stellar velocity dispersion \citep{magorrian98, marconi03, haringrix04}. \citet{mcconnell11}, using archival measurements along with two more accurate measurements of black hole masses derived using reverberation mapping techniques, define the $M_{BH}-\sigma$ relationship as:
\begin{equation}
\log_{10}\left(\frac{M_{BH}}{M_{\odot}}\right) = 8.29 + 5.12 ~\log_{10}\left(\frac{\sigma}{200 ~\rm{km} ~s^{-1}}\right). 
\end{equation}
where the velocity dispersion, $\sigma$, is measured from the Balmer lines and is provided in the MPA-JHU catalog \citep{kauffmann03, brinchmann04} for the \textsc{agn-host} sample.

The Eddington ratio, $\lambda_{Edd}$, describes the accretion rate of the black hole and is calculated with the proxy $\lambda_{Edd} = L_{bol}/L_{Edd}$, where $L_{Edd}$ is the Eddington luminosity and $L_{bol}$ is the bolometric luminosity. An accreting black hole is Eddington limited when the gravitional force, $F_{\rm{g}}~=~GM_{BH}m/R^2$, balances the radiation pressure force, $F_{\rm{rad}}~=~\kappa~m\cdot\frac{L}{c}\cdot\frac{1}{4\pi R^2}$ (assuming spherical symmetry). Since the black hole is most likely accreting ionized hydrogen we can estimate $\kappa\approx \sigma_T/m_p$ where $\sigma_T$ is the Thomson scattering cross section. Therefore the Eddington luminosity can be calculated as:
\begin{equation}
L_{Edd} = \frac{4\pi G M_{BH} c m_p}{\sigma_T},
\end{equation}
as outlined in \citet{binneymerrifield}. This equation then reduces to:
\begin{equation}
L_{Edd} = 3\times10^4 \left(\frac{M_{BH}}{M_{\odot}}\right) L_{\odot}.
\end{equation}
To determine $\lambda_{Edd} = L_{bol}/L_{Edd}$, we still need to be able to calculate the bolometric luminosity, $L_{bol}$. For obscured (Type 2) AGN the bolometric luminosity cannot be directly measured and so is inferred from the luminosity of the [OIII] emission line, as derived by \citet{heckman04}:
\begin{equation}
\log_{10}L_{bol} = 3.54 + \log_{10}L[OIII]. 
\end{equation}

The distributions of L[OIII], $M_{BH}$ and $\lambda_{Edd}$ of the \textsc{agn-host} sample are shown in Figure~\ref{fig:agndistributions}. SDSS images for 10 randomly selected galaxies from the \textsc{agn-host} sample are shown in Figure~\ref{mosaic}. The decomposition of the \textsc{agn-host} sample into red sequence, green valley and blue cloud galaxies is shown in Tables~\ref{table:agnsubs} and \ref{table:agnqsubs} along with further division by galaxy type and SFR (where available for the \textsc{agn-host} sample from the MPA-JHU catalog) respectively. 


\begin{table}
\caption{Table showing the decomposition of the \textsc{agn-host} sample by galaxy type into the subsets of the colour-magnitude diagram.}
\begin{tabular*}{\textwidth}{l @{\extracolsep{\fill}}cccc}
\hline
\begin{tabular}[c]{@{}c@{}} {\color{white} -} \\ {\color{white} -}  \end{tabular} & All                                                      & Red Sequence                                              & Green Valley                                              & Blue Cloud \\  \hline 
Smooth-like ($p_s > 0.5$)        & \begin{tabular}[c]{@{}c@{}}340\\ (27.3\%)\end{tabular} & \begin{tabular}[c]{@{}c@{}}21\\ (25.0\%)\end{tabular}  & \begin{tabular}[c]{@{}c@{}}105\\ (41.2\%)\end{tabular}   & \begin{tabular}[c]{@{}c@{}}213\\ (23.5\%)\end{tabular}  \\ 
Disc-like ($p_d > 0.5$)          & \begin{tabular}[c]{@{}c@{}}871\\ (70.0\%)\end{tabular} & \begin{tabular}[c]{@{}c@{}}63\\ (75.0\%)\end{tabular}   & \begin{tabular}[c]{@{}c@{}}148\\ (58.0\%)\end{tabular}  & \begin{tabular}[c]{@{}c@{}}660\\ (72.9\%)\end{tabular}  \\
Early-type ($p_s \geq 0.8$) & \begin{tabular}[c]{@{}c@{}}66\\ (5.3\%)\end{tabular}  & \begin{tabular}[c]{@{}c@{}}1\\ (1.2\%)\end{tabular}    & \begin{tabular}[c]{@{}c@{}}14\\ (5.5\%)\end{tabular}    & \begin{tabular}[c]{@{}c@{}}51\\ (5.6\%)\end{tabular}    \\
Late-type ($p_s \geq 0.8$)  & \begin{tabular}[c]{@{}c@{}}569\\ (45.7\%)\end{tabular} & \begin{tabular}[c]{@{}c@{}}39\\ (46.4\%)\end{tabular}    & \begin{tabular}[c]{@{}c@{}}74\\ (29.0\%)\end{tabular}    & \begin{tabular}[c]{@{}c@{}}456\\ (50.4\%)\end{tabular}  \\ \hline
\textbf{Total}                       & \begin{tabular}[c]{@{}c@{}}\textbf{1244} \\ (100.0\%)\end{tabular}                                                & \begin{tabular}[c]{@{}c@{}}84 \\ (6.7\%)\end{tabular} & \begin{tabular}[c]{@{}c@{}}255 \\ (20.5\%)\end{tabular} & \begin{tabular}[c]{@{}c@{}}905 \\ (72.7\%)\end{tabular} \\\hline
\end{tabular*}
\label{table:agnsubs}
\end{table}


\begin{table}
\caption{Table showing the decomposition of the \textsc{agn-host} sample galaxies by their star formation rate in the subsets of the colour-magnitude diagram.}
\begin{tabular*}{\textwidth}{l @{\extracolsep{\fill}}cccc}
\hline
\begin{tabular}[c]{@{}c@{}} {\color{white} -} \\ {\color{white} -}  \end{tabular} 		& All                                                      						& Red Sequence                                              			& Green Valley                                             			 & Blue Cloud \\  \hline 
\begin{tabular}[l]{@{}l@{}}Quenched\\ ($\rm{SFR} < P - 5\sigma$) \end{tabular}				& \begin{tabular}[c]{@{}c@{}}14\\ (1.3\%)\end{tabular} 			& \begin{tabular}[c]{@{}c@{}}9\\ (12.5\%)\end{tabular}    & \begin{tabular}[c]{@{}c@{}}4\\ (1.7\%)\end{tabular}    & \begin{tabular}[c]{@{}c@{}}1\\ (0.1\%)\end{tabular}  \\ 
\begin{tabular}[l]{@{}l@{}}Quenching\\ ($P - 5\sigma < \rm{SFR} < P - \sigma$) \end{tabular}	 & \begin{tabular}[c]{@{}c@{}}335\\ (30.6\%)\end{tabular}			 & \begin{tabular}[c]{@{}c@{}}45\\ (64.3\%)\end{tabular}    & \begin{tabular}[c]{@{}c@{}}139\\ (59.9\%)\end{tabular}    & \begin{tabular}[c]{@{}c@{}}151\\ (19.1\%)\end{tabular}  \\ 
\begin{tabular}[l]{@{}l@{}}Star Forming  \\ ($\rm{SFR} > P -\sigma$) \end{tabular} 			& \begin{tabular}[c]{@{}c@{}}744\\ (68.0\%)\end{tabular} 			& \begin{tabular}[c]{@{}c@{}}16 \\ (22.9\%)\end{tabular}    & \begin{tabular}[c]{@{}c@{}}89\\ (38.4\%)\end{tabular}    & \begin{tabular}[c]{@{}c@{}}639\\ (80.7\%)\end{tabular}  \\ \hline
\textbf{Total}                       														& \begin{tabular}[c]{@{}c@{}}\textbf{1093} \\ (100.0\%)\end{tabular} & \begin{tabular}[c]{@{}c@{}}70 \\ (6.4\%)\end{tabular} & \begin{tabular}[c]{@{}c@{}}232 \\ (21.2\%)\end{tabular} & \begin{tabular}[c]{@{}c@{}}791 \\ (72.4\%)\end{tabular} \\\hline
\end{tabular*}
\label{table:agnqsubs}
\end{table}


\subsection{Defining a control sample}

\begin{figure*}
\centering
\includegraphics[width=\textwidth]{agn/agn-host_inactive_pd_mass_z_distributions.pdf}
\caption[Morphology, stellar mass and redshift distributions of the \textsc{agn-host} and \textsc{inactive} samples]{Distribution of the GZ2 disc vote fractions ($p_d$; left), stellar masses (middle) and redshift (left) in the \textsc{agn-host} sample (solid lines) in comparison to the matched control \textsc{inactive} sample (dashed lines).}
\label{fig:zmdistmatch}
\end{figure*}

\begin{figure}
\centering
\includegraphics[height=0.75\textheight]{agn/fig3.pdf}
\caption[Colour-magnitude and SFR-mass diagram for \textsc{agn-host} galaxies]{(a) Optical colour-magnitude diagram showing the SDSS DR7 (grey filled contours), the \textsc{agn-host} sample (red circles) and \textsc{inactive} sample (blue contours). The definition of the green valley from \citet{Baldry06} (solid line) with $\pm 1\sigma$ (dashed lines) is shown. (b) SFR-stellar mass diagram showing the MPA-JHU measurements of SFR and $M_*$ of SDSS DR7 galaxies (\citealt{kauffmann03, brinchmann04}; black contours), the \textsc{agn-host} sample (red circles) and \textsc{inactive} sample (blue contours). The star forming `main sequence' from \citet{peng10} is shown by the solid line for $t = 12.8~\rm{Gyr}$, the average observed age of the \textsc{gz2-galex} sample, with $\pm1\sigma$ (dashed lines).}
\label{cmdsfms}
\end{figure}


A control sample of inactive galaxies was constructed by removing from the \textsc{gz2-galex} sample all galaxies with line strengths indicative of potential AGN activity \citep{kauffmann03b}, as well as sources identified as Type 1 AGN by the presence of broad emission lines \citep{Oh15}. This is to ensure a pure sample of galaxies with no AGN activity. I selected a mass- and morphology-matched inactive sample by identifying up to 5 inactive galaxies for each \textsc{agn-host} galaxy with the same stellar mass (to within $\pm5\%$) and GZ2 smooth, $p_s$, and disc $p_d$,  vote fractions (to within $\pm 0.1$) giving $6107$ galaxies. This sample will be referred to as the \textsc{inactive} sample. 


Figure~\ref{fig:zmdistmatch} shows the GZ2 disc vote fraction ($p_d$; left), stellar mass ($M_*$; middle) and redshift (right) distributions of the \textsc{agn-host} sample in comparison to the matched \textsc{inactive} sample. A Kolmogorov-Smirnov test revealed that the redshift distributions of the \textsc{inactive} and \textsc{agn-host} samples are statistically indistinguishable ($D \sim 0.16$, $p \sim 0.88$). 

The \textsc{agn-host} and \textsc{inactive}  samples are also shown on both an optical colour-magnitude diagram and in the SFR-stellar mass plane in Figure~\ref{cmdsfms} in comparison to the distribution of SDSS DR7 galaxies. SFRs and stellar masses are obtained from the MPA JHU catalog, where available, which follow the prescriptions outlined in \cite{brinchmann04} and \cite{Salim07} for calculating the total aperture corrected galaxy SFR in the presence of an AGN. 

The majority of the \textsc{agn-host} sample would be defined as residing in the blue cloud ($\sim73\%$) on the optical colour-magnitude diagram despite the fact that a significant proportion of the sample ($32\%$) lie more than $1\sigma$ ($0.3$ $\rm{dex}$) below the SFS as defined in Section~\ref{qmod} \citep[][see Figure \ref{cmdsfms} and Table~\ref{table:agnqsubs}]{peng10}.

\subsection{Results}\label{results}

\begin{figure}
\centering{
\includegraphics[width=\textwidth]{agn/fig6.pdf}}
\caption[Quenching time and rate population density distributions for the \textsc{agn-host} sample split by Eddington ratio]{Population density distributions for the quenching time ($t_q$; left) and rate ($\tau$; right), normalised so that the areas under the curves are equal. The \textsc{agn-host} sample is split into low (top), medium (middle) and high (bottom) Eddington ratio, $\lambda_{Edd}$, for smooth (dashed) and disc (solid) galaxies. Uncertainties from bootstrapping are shown by the shaded regions for the smooth (grey striped) and disc (grey solid) population densities. A small (large) value of $\tau$ corresponds to a rapid (slow) quench.}
\label{eddratiosplit}
\end{figure}

\begin{figure*}
\centering{
\includegraphics[width=\textwidth]{agn/fig4.pdf}}
\caption[Quenching time population density distributions for the \textsc{agn-host} and \textsc{inactive} samples] {Population density distributions for the quenching time ($t_q$) parameter, normalised so that the areas under the curves are equal. \textsc{agn-host} (left) and \textsc{inactive} (right) galaxies are split into low (top), medium (middle) and high (bottom) mass for smooth (dashed) and disc (solid) galaxies. Uncertainties from bootstrapping are shown by the shaded regions for the smooth (grey striped) and disc (grey solid) population densities. A low (high) value of $t_q$ corresponds to the early (recent) Universe.}
\label{time}
\end{figure*}


\begin{figure*}
\centering{
\includegraphics[width=\textwidth]{agn/fig5.pdf}}
\caption[Quenching rate population density distributions for the \textsc{agn-host} and \textsc{inactive} samples]{Population density distributions for the quenching rate ($\tau$), normalised so that the areas under the curves are equal. \textsc{agn-host} (left) host and \textsc{inactive} (right) galaxies are split into low (top), medium (middle) and high (bottom) mass for smooth (dashed) and disc (solid) galaxies. Uncertainties from bootstrapping are shown by the shaded regions for the smooth (grey striped) and disc (grey solid) population densities. A small (large) value of $\tau$ corresponds to a rapid (slow) quench.}
\label{rate}
\end{figure*}


%Figures~\ref{eddratiosplit}, \ref{time} \& \ref{rate} show the population density distributions for the quenching time, $t_q$ and exponential quenching rate, $\tau$, with shaded regions to show the uncertainties for both smooth and disc weighted populations. 

I use the \textsc{popstarpy} method (see Section~\ref{popstarpy}) to produce the distributions shown in Figure \ref{eddratiosplit} for the \textsc{agn-host} smooth and disc weighted populations which are split across three bins in Eddington ratio, $\lambda_{Edd}$, to investigate any trends with the accretion rate of the black hole. The bin boundaries were chosen to give equal numbers of \textsc{agn-host} galaxies in each bin. 

Similarly, in Figures~\ref{time} \& \ref{rate} I use the \textsc{popstarpy} method to determine the \textsc{agn-host} and \textsc{inactive} smooth and disc weighted populations for the quenching time and rate, split between three mass bins to investigate any trends with mass. The bin boundaries were chosen to give roughly equal numbers of inactive galaxies in each bin, before mass matching to the \textsc{agn-host} sample. This decision was made to ensure the mass bins were representative of `typical' galaxies rather than being biased by the mass distribution of the \textsc{agn-host} sample which tend to occupy higher mass galaxies. 

Table~\ref{massbins} contains the percentages of the population densities shown in Figures~\ref{time} \& \ref{rate} in the quenching regimes originally defined in Chapter~\ref{chap:morph} for rapid ($\tau < 1$ Gyr), intermediate ($1 < \tau ~\rm{[Gyr]} < 2$) and slow ($\tau > 2$ Gyr) quenching rates in each of the three mass bins. Uncertainties in the population densities (shown by the shaded regions) are determined from the maximum and minimum values spanned by $N = 1000$ bootstrap iterations, each sampling $90\%$ of the galaxy population. $1\sigma$ uncertainties are quoted for the percentages in Table~\ref{massbins}, calculated from the bootstrapped distributions.


\begin{table*}
{\tiny
\centering{
\caption{Table showing the number of galaxies in each of the three mass bins for both the \textsc{agn-hosts} and \textsc{inactive} galaxy samples and the percentage of the distribution across each morphologically weighted population found in the rapid, intermediate and slow quenching regimes.}
\label{massbins}
\begin{tabular}{c|c|c|c|c|c|c}
\hline
\textsc{sample}                     & \textsc{mass bin}                                        & \textsc{weighting}                  & $\tau < 1 ~\rm{[Gyr]}$                             & $1 < \tau ~\rm{[Gyr]} < 2 $          & $\tau > 2 ~\rm{[Gyr]}$                               & \textsc{number}                                        \\ \hline \hline
\multirow{6}{*}{AGN-HOSTS} & \multirow{2}{*}{$\log [M_*/M_{\odot}] < 10.25 $}                       & $p_d$     & $60\pm_{5}^{23}$                    & $13\pm_{9}^{9}$                    & $28\pm_{19}^{6}$       & \multirow{2}{*}{$165 (13.3\%)$}                      \\
                           &                                                 & $p_s$     & $69\pm_{6}^{14}$                    & $17\pm_{14}^{6}$                   & $14\pm_{7}^{3}$        &                                                      \\ \cline{2-7} 
                           & \multirow{2}{*}{$10.25 < \log [M_*/M_{\odot}] < 10.75$}                    & $p_d$     & $33\pm_{5}^{3}$                     & $15\pm_{4}^{4}$                    & $51\pm_{7}^{4}$        & \multirow{2}{*}{$630 (50.6\%)$}                      \\
                           &                                                 & $p_s$     & $69\pm_{5}^{4}$                     & $7\pm_{4}^{4}$                     & $26\pm_{9}^{5}$        &                                                      \\ \cline{2-7} 
                           & \multirow{2}{*}{$\log [M_*/M_{\odot}] > 10.75$}                      & $p_d$     & $20\pm_{4}^{5}$ & $25\pm_{5}^{7}$                    & $56\pm_{12}^{8}$       & \multirow{2}{*}{$449 (36.1\%)$}                      \\
                           &                                                 & $p_s$     & $24\pm_{3}^{4}$                     & $26\pm_{6}^{5}$                    & $50\pm_{7}^{7}$        &                                                      \\ \hline \hline
\multirow{6}{*}{INACTIVE}  & \multirow{2}{*}{$\log [M_*/M_{\odot}] < 10.25 $}                       & $p_d$     & $37\pm_{14}^{8}$                    & $39\pm_{6}^{8}$                    & $24\pm_{6}^{8}$        & \multirow{2}{*}{$807 (13.2\%)$}                      \\
                           &                                                 & $p_s$     & $47\pm_{11}^{5}$                    & $36\pm_{5}^{9}$                    & $17\pm_{5}^{4}$        &                                                      \\ \cline{2-7} 
                           & \multirow{2}{*}{$10.25 < \log [M_*/M_{\odot}] < 10.75$}                    & $p_d$     &          $30\pm_{3}^{4}$                          &       $18\pm_{3}^{2}$                            &    $51\pm_{4}^{4}$                   & \multirow{2}{*}{$3094 (50.7\%)$}                     \\
                           &                                                 & $p_s$     & $42\pm_{2}^{2}$            & $29\pm_{3}^{3}$   & $30\pm_{4}^{3}$ &                                                      \\ \cline{2-7} 
                           & {\multirow{2}{*}{$\log [M_*/M_{\odot}] > 10.75$}} & $p_d$     & $36\pm_{3}^{3}$            & $24\pm_{4}^{3}$         & $41\pm_{3}^{4}$ & \multicolumn{1}{l}{\multirow{2}{*}{$2206 (36.1\%)$}} \\
                           & \multicolumn{1}{l|}{}                           & $p_s$      & $38\pm_{2}^{2}$              & $28\pm_{4}^{3}$            & $34\pm_{3}^{3}$ & \multicolumn{1}{l}{}                                 \\ \hline                       
\end{tabular}}}
\end{table*}

At all masses, the population density for galaxies within the \textsc{agn-host} population across the quenching time, $t_q$ (left panels of Figure~\ref{time}), is different from that of the inactive galaxies (right panels of Figure~\ref{time}). Recent quenching ($t > 11$ Gyr) is the dominant history for low and medium mass \textsc{agn-host} galaxies, particularly for the smooth weighted population hosting an AGN (solid lines, left panels Figure~\ref{time}). However, this effect is less dominant in higher mass galaxies where quenching at earlier times also has high density (bottom left panel of Figure \ref{time}).


The population densities for the quenching rate, $\tau$, in Figure~\ref{rate} and Table~\ref{massbins} show the dominance of rapid quenching ($\tau < 1$ Gyr) within the \textsc{agn-host} population, particularly for smooth galaxies (solid lines, left panels Figure~\ref{rate}). With increasing mass the dominant quenching rate becomes slow ($\tau > 2$ Gyr). Similar trends in the population density are observed for the \textsc{inactive} population (right panels Figure~\ref{rate}) but the overall distribution is very different. 

In Figure~\ref{eddratiosplit} there is no \textsc{inactive} control sample to act as a comparison, since an Eddington ratio cannot be measured for a black hole that is not accreting. The population densities for the \textsc{agn-host} samples however, show the dominance of rapid and recent quenching as the Eddington ratio increases (i.e. higher black hole accretion rates). A bimodal distribution can be seen for the low Eddington ratio \textsc{agn-host} population (top right panel of Figure~\ref{eddratiosplit}) between both rapid and slow quenching rates. 

The population densities for the \textsc{agn-host} galaxies therefore all show evidence for the dominance of rapid, recent quenching. This result implies the importance of AGN feedback for the evolution within this population.

\subsection{Discussion}\label{sec:agndis}

The differences between the population density distributions of the \textsc{agn-host} and \textsc{inactive} populations in Figures~\ref{time} \& \ref{rate} reveal that an AGN can have a significant effect on the SFH of its host galaxy. Both recent, rapid quenching and early, slow quenching are observed in the population densities of the \textsc{agn-host} population. 

There are however minimal differences between the smooth and disc weighted distributions of the quenching parameters within the \textsc{agn-host} population (comparing solid and dashed lines in the left panels of Figures~\ref{eddratiosplit} - \ref{rate}). This is in agreement with the conclusions of \citet{kauffmann03b} who found that the structural properties of AGN hosts depend very little on AGN power. Quenching caused by AGN feedback is therefore morphologically independent; this is unlike the mechanisms discussed in Chapter~\ref{chap:morph}. This suggests that the morphological dependances observed across the quenching histories in Chapter~\ref{chap:morph} are not due to the effects of AGN feedback. 

Quenching at early times is observed within the \textsc{inactive} population (see right panels of Figure~\ref{time}), where the population density is roughly constant until recent times where the distribution drops off.  This drop-off occurs at earlier times with increasing mass, with a significant lack of quenching occurring at early times for low mass \textsc{inactive} galaxies (top right panel Figure~\ref{time}). This is evidence of downsizing within the \textsc{inactive} galaxy population whereby stars in massive galaxies form first and quench early \citep{Cowie96, Thomas10}. 

The population densities for smooth weighted higher mass (bottom left panels Figures~\ref{time} \& \ref{rate}) and lower Eddington ratio (top panels Figure \ref{eddratiosplit}) \textsc{agn-host} galaxies are also dominated by slow, early quenching. This implies that AGN feedback is not responsible for the cessation of star formation within a proportion of these galaxies, as this quenching has occurred prior to the triggering of the current AGN. I speculate that this is also due to the effects of downsizing rather than being caused by the current AGN, as the shape of both the \textsc{agn-host} and \textsc{inactive} high mass population densities are very similar at these early times. Since the lower Eddington ratio \textsc{agn-host} population is also dominated by this early quenching at slow rates (top panels Figure~\ref{eddratiosplit}) this supports the idea that the AGN is not strong enough to be the cause of this quenching.  This earlier quenching would slowly form a `dead' galaxy typical of massive elliptical galaxies which could then have a recent infall of gas either through a minor merger, galaxy interaction or environmental change, triggering further star formation and feeding the central black hole, triggering an AGN \citep{kaviraj14b}. In turn this AGN can then quench the recent boost in star formation. This track is similar to the evolution history proposed for blue ellipticals \citep[][and detected in the top panel of Figure~\ref{blue_c}]{Kaviraj13, McIntosh14, Haines15}. This SFH would then give rise to the distribution seen within the high mass smooth weighted \textsc{agn-host} population for both time and rate parameters (bottom panels of Figures~\ref{time} \& \ref{rate}).

Alternatively in the high mass disc weighted \textsc{agn-host} population, in which slow, early quenching is also observed (dashed lines bottom left panels of Figures~\ref{time} \& \ref{rate}) this evolution could be due to initially isolated discs evolving slowly by the Kennicutt-Schmidt law \citep{schmidt59, kennicutt97}. These can then undergo an interaction or merger to reinvigorate star formation, feed the central black hole and trigger an AGN \citep{Varela04, emsellem15}. These galaxies would need a large enough gas reservoir to fuel both SF throughout their lifetimes and the recent AGN. These high mass galaxies also play host to the most luminous AGN (mean $\log (L[OIII] ~[\rm{erg}~s^{-1}]) \sim 41.6$) and so this SFH challenges the typical merger driven co-evolution of luminous black holes and their host galaxies (see Section~\ref{sec:intbulgeless} for an investigation into an alternative merger free black hole-galaxy co-evolution). 

In both these high mass disc and smooth galaxies, the recently triggered, low accretion rate AGN do not have have the ability to impact the SF across the entirety of a high mass galaxy in a deep gravitational potential \citep{ishibashi12, Zinn13}. This leads to the lower peak for recent, rapid quenching within the high mass, low Eddington ratio \textsc{agn-host} population for both morphologies (left bottom panels of Figures~\ref{time} \& \ref{rate} and top panels of Figure~\ref{eddratiosplit}). 

Conversely if we now consider the low mass (left top panel Figure~\ref{rate}) and high Eddington ratio (bottom left panel of Figure~\ref{eddratiosplit}) \textsc{agn-host} population, rapid quenching, possibly caused by the AGN itself through negative feedback, is the most dominant quenching history. Within the medium mass \textsc{agn-host} population a bimodal distribution between these two quenching histories is seen (left middle panel Figure~\ref{rate}), highlighting the strength of this method which is capable of detecting such variation in the SFHs within a population of galaxies. These lower and medium mass galaxies have lower gravitational potentials from which gas may be more readily expelled or heated \citep{tortora09} by jets launched by the strongly accreting AGN, suggesting that the AGN may be the cause of the quenching observed in the population densities.

\cite{tortora09} model the effects of this jet-induced AGN feedback on a typical early-type galaxy and observe a drastic suppression of star formation on a timescale of $\sim 3 ~\rm{Myr}$. Comparing their synthetic colours with observed colours of SDSS elliptical galaxies, they find that the time between the current galaxy age, $t_\mathrm{gal}$, and the time that the feedback began, $t_\mathrm{AGN}$, peaks at $t_\mathrm{gal} - t_\mathrm{AGN} \sim 0.85 ~\rm{Gyr}$. In the top panel of Figure~\ref{time}, the population density for low mass \textsc{agn-host} galaxies has a difference between the peak of the distribution and the average age of the population (galaxy age is calculated as the age of the Universe at the observed redshift, by assuming all galaxies form at $t=0$) of $\sim0.83 ~\rm{Gyr}$. This is in agreement with the timescales for AGN feedback driven quenching derived in the simulations by \citet{tortora09}. The dominance of rapid quenching across the smooth \textsc{agn-host} population (solid lines in left panels of Figure~\ref{rate}) supports the hypothesis discussed in Chapter~\ref{chap:morph} that a merger, having caused a morphological transformation to a smooth galaxy, can also trigger an AGN, causing feedback and cessation of star formation (\citealt{sanders88, pontzen16}).

%Rapid quenching is particularly dominant for low-to-medium mass smooth galaxies as seen in the level panels of Figure~\ref{rate}. In Chapter~\ref{morph} I discussed how these incredibly rapid quenching rates could be attributed to mergers of galaxies in conjunction with AGN feedback, which are thought to be responsible for creating the most massive smooth galaxies \citep{conselice03b, springel05b, hopkins08b}. This dominance of rapid quenching across the smooth \textsc{agn-host} population (solid lines in left panels of Figure~\ref{rate}) supports this hypothesis that a merger, having caused a morphological transformation to a smooth galaxy, can also trigger an AGN, causing feedback and cessation of star formation (\citealt{Sanders88, pontzen16}).

Simulations by \cite{sparre16} show that major merger remnants are only quenched when the prescription used for AGN feedback in the model is stronger than their initial fiducial level. They increase the strength of their AGN feedback by decreasing the metallicity, $Z$, of the gas accreted by the black hole; this impacts on the cooling function of the gas $\Lambda(T, Z)$ \citep[see Section 2.1 of][]{sparre16}. This suggest that AGN feedback will therefore have a larger effect on galaxies with lower metallicity. Considering the mass-metallicity relation \citep{tremonti04}, it follows from this argument that AGN feedback will have a greater effect on galaxies with lower mass. This provides more support to the hypothesis that the dominance of rapid, recent quenching across the low and medium mass \textsc{agn-host} population (left panels Figures~\ref{time} \& \ref{rate}) is caused directly by the AGN.

 \begin{figure*}
\centering{
\includegraphics[width=\textwidth]{agn/fig7.pdf}}
\caption[Galaxy luminosity function from observations and simulations: Figure 1 of \cite{benson03}]{Figure 1 from \cite{benson03} showing the K-band luminosity function of galaxies, with $h=0.65$. The points show the observational determinations of \cite[][circles]{cole01}, \cite[][squares]{kochanek01}, and \cite[][$z < 0.1$, stars]{huang03}. Lines show model results investigated by \cite{benson03} to determine what shapes this luminosity function; they concluded that including prescriptions for AGN feedback (supernova feedback winds) can help match the simulations to the data at the bright (faint) end of the luminosity function. The knee of the function occurs at $M_K -5\log_{10} h \sim 23.5$, which is an approximate stellar mass of $\log_{10}[M_*/M_{\odot}] \sim 10.3$, assuming a mass-to-light ratio of $(M/L)_K = 0.8$ \citep{brinchmann00}.}
\label{lumfunc}
\end{figure*}

The mechanism of AGN feedback was originally proposed to regulate the number galaxies at the bright (or high mass) end of the luminosity function in cosmological simulations (see Chapter~\ref{chap:intro}). The shape of the observed K-band luminosity function can be seen in Figure~\ref{lumfunc} \citep[Figure 1 from][]{benson03}, falling away from model estimates below a K-band magnitude of $M_K -5\log_{10} h \sim 23.5$, or above an approximate stellar mass of $\log_{10}[M_*/M_{\odot}] \gtrsim 10.3$, assuming a mass-to-light ratio of $(M/L)_K = 0.8$ \citep{brinchmann00}. However, it is the low and medium mass \textsc{agn-host} populations where this rapid recent quenching is dominant (see left panels Figures~\ref{time} \& \ref{rate}). At first this seems contradictory to the arguments posed to constrain the shape of the luminosity function, but with some thought the two results can be reconciled. 

The knee of the luminosity function is the point at which AGN feedback starts to impact the masses of galaxies; this occurs at $\log_{10}[M_*/M_{\odot}] \gtrsim 10.3$, which lies at the lower edge of the medium mass \textsc{agn-host} population studied here. The quenching observed in the low and medium mass \textsc{agn-host} populations will stop the stellar masses of these galaxies from growing any larger in the future. In turn these now quenched galaxies will contribute to dry mergers, which would otherwise have had high gas fractions; limiting the stellar mass of the merger remnants. The combination of these two effects, caused initially by the quench of a lower mass galaxy by negative AGN feedback, will reduce the number of galaxies which will be able to grow to populate the high mass end of the luminosity function. 

However, there still remains the possibility that the AGN is not the cause of the quenching observed, but merely a \emph{consequence} of an alternative quenching mechanism. This idea is supported by simulations showing that the exhaustion of gas by a merger fuelled starburst could cause such a rapid quench in star formation and in turn also trigger an AGN \citep{Croton06, Wild09, snyder11, hayward14}. \citet{Yesuf14} also showed that AGN are more commonly hosted by post starburst galaxies, with the peak AGN activity appearing $\geq 200 \pm 100 ~\rm{Myr}$ after the starburst. Such a SFH is not accounted for in the models presented here (see Section ~\ref{sec:future} for a discussion), however this scenario is still consistent with the results presented in this paper; that AGN which are \emph{currently} active have been detected in host galaxies $\sim 1~\rm{Gyr}$ after the onset of quenching. Solving this issue of \emph{cause} vs. \emph{consequence} will not be possible with the currently available SDSS photometry. The advent of Integral Field Unit (IFU) surveys with many aperture fibres per galaxy per observation (such as the MaNGA \citep{bundy15}, SAMI \citep{croom12} and CALIFA \citep{sanchez12} surveys) will allow this problem to be studied by observing the change in quenching parameters with increasing distance from the galaxy nucleus (see Section \ref{sec:IFU} for a more detailed discussion). 
 

Not all galaxies in the \textsc{agn-host} and \textsc{inactive} samples are quenching ($\sim50\%$, as seen in Figure \ref{cmdsfms}) with a significant proportion of both the \textsc{agn-host} and \textsc{inactive} samples lying on the SFS. A galaxy can therefore still maintain star formation whilst hosting an AGN. The results presented in Section \ref{results} only reflect the trends for galaxies that have undergone or are currently undergoing quenching within the \textsc{agn-host} population and can therefore be accurately fit by an exponentially declining SFH. This prevalence of star forming AGN host galaxies, combined with the results above allows us to consider that either: (i)  the AGN are the cause of the rapid quenching observed but only in gas-poor host galaxies where they can have a large impact, (ii) the AGN are a consequence of another quenching mechanism but can also be triggered by other means which do not cause quenching, or (iii) the SFR of a galaxy can recover post-quench and return to the star forming sequence after a few Gyr (see recent simulations by \citealt{pontzen16} and \citealt{sparre16}). Further investigation will therefore be required to determine the nature of this quenching (see Section \ref{sec:IFU} for a discussion of proposed future work with IFU survey data).


%%%%%%%%%%%%%%%%%%%%%%%%%%%%%%%%%%%%%%%%%%%%%%%%%%%%%%%%%%%%%%%%%%%%%%%%%
%%%%%%%%%%%%%%%%%%%%%%%%%%%%%%%%%%%%%%%%%%%%%%%%%%%%%%%%%%%%%%%%%%%%%%%%%
%%%%%%%%%%%%%%%%%%%%%%%%%%%%%%%%%%%%%%%%%%%%%%%%%%%%%%%%%%%%%%%%%%%%%%%%%
%%%%%%%%%%%%%%%%%%%%%%%%%%%%%%%%%%%%%%%%%%%%%%%%%%%%%%%%%%%%%%%%%%%%%%%%%
%%%%%%%%%%%%%%%%%%%%%%%%%%%%%%%%%%%%%%%%%%%%%%%%%%%%%%%%%%%%%%%%%%%%%%%%%
%%%%%%%%%%%%%%%%%%%%%%%%%%%%%%%%%%%%%%%%%%%%%%%%%%%%%%%%%%%%%%%%%%%%%%%%%
%%%%%%%%%%%%%%%%%%%%%%%%%%%%%%%%%%%%%%%%%%%%%%%%%%%%%%%%%%%%%%%%%%%%%%%%%
%%%%%%%%%%%%%%%%%%%%%%%%%%%%%%%%%%%%%%%%%%%%%%%%%%%%%%%%%%%%%%%%%%%%%%%%%
%%%%%%%%%%%%%%%%%%%%%%%%%%%%%%%%%%%%%%%%%%%%%%%%%%%%%%%%%%%%%%%%%%%%%%%%%
%%%%%%%%%%%%%%%%%%%%%%%%%%%%%%%%%%%%%%%%%%%%%%%%%%%%%%%%%%%%%%%%%%%%%%%%%
 

\newpage

\section{Bulgeless galaxies hosting growing black holes}\label{sec:intbulgeless}

\emph{The work in the following chapter is in preparation for submission to MNRAS in Simmons, Smethurst \& Lintott (in prep.). I assisted in the planning and taking of observational spectral data, along with being responsible for the subsequent data reduction and statistical analysis. I was a major contributor to the interpretation of the results.}

%INSERT PARAGRAPH LEADING ON FROM PREVIOUS SECTION

Although the study of large populations of galaxies provides crucial information to constrain the processes governing galaxy evolution, valuable insight can still be discerned from detailed observations of a smaller sample of rare objects. 

The strong correlations that exist between black hole mass and velocity dispersion \citep{magorrian98, merritt01, hu08, kormendy11a, mcconnell11}, bulge stellar mass \citep{marconi03, haringrix04}, and total stellar mass \citep{cisternas11, Simmons13} suggest that galaxies co-evolve with their central super massive black holes (SMBH). Since mergers can grow both bulges and black holes these correlations have been interpreted as the result of a few mergers within a Hubble time \citep{peng07, hopkins08a, jahnke11}. A growing black hole must be accreting matter and is therefore observed as an AGN during this time period. Understanding the triggering mechanisms of AGN, which kick-start this process of simultaneous galaxy and black hole evolution and possible subsequent quenching caused by the AGN, is therefore important. However, in Section \ref{sec:agnfeedback}, I presented the argument that disc galaxies currently hosting an AGN undergo quenching with very slow rates, unlike the rapid rates found in simulations of mergers. This suggests a secular driven co-evolution of galaxies and their black holes. 

A secular co-evolution of galaxy and black hole has been proposed by previous works \citep{greene10b, jiang11b, cisternas11, Simmons11,schawinski11, kocevski12} and was investigated by \citet{Simmons13}. \citeauthor{Simmons13}  studied 13 AGN residing in bulgeless host galaxies, whose accretion histories are assumed to be merger free, since mergers grow bulges \citep{toomre77, hopkins11c, tonini16}. They identified two broad line objects in this sample from which they calculated black hole masses, and placed lower limits on the black hole masses of the other 11 objects. They found that the black hole masses were larger than predicted by established stellar bulge mass-black hole mass relations, but consistent with total stellar mass-black hole mass relations. However, these conclusions were drawn based on only two direct measurements of black hole mass and so they were limited by small number statistics.

In the following work, I follow up the investigation by \citeauthor{Simmons13} by examining a larger sample of disc galaxies, visually identified as bulgeless AGN hosts and investigate the locations of these galaxies on typical galaxy-black hole scaling relations. Since the disc galaxies in this sample will have different dynamical histories to bulge dominated galaxies, a correlation between their black holes and bulge stellar masses is not expected, if different dynamical histories lead to different mechanisms for black hole growth. 

%The most famous of these scaling relations is the $M-\sigma$ relation. This is a well studied relationship \citep{magorrian98, merrit01, kormendy01, tremaine01, marconi01, graham07, graham08, greene10, mcconnell11, beifoiri12, mcconnell13} and is hypothesised to arise due to the merger driven co-evolution increasing the mass of the black hole at the same time as increasing the velocity dispersion of the galaxy \citep{peng07, hopkins08a, jahnke11}. The black hole mass has similarly been found to correlate with the stellar mass of the galaxy bulge \citep{marconi03, haringrix04} and the total galaxy stellar mass \citep{cisternas11}.


%%%%%%%%%%%%%%%%%%%%%%%%%%%%%%%%%%%%%%%%%%%%%%
%
%
\subsection{Observational Data}\label{sec:data}
%
%
%%%%%%%%%%%%%%%%%%%%%%%%%%%%%%%%%%%%%%%%%%%%%%

The goal of this study is to investigate black hole growth in galaxies whose growth histories have been dominated by relatively calm, slow processes. A sample of growing (i.e. active) black holes hosted in disc-dominated galaxies is therefore required. Optimally, the AGN should have broad emission lines to facilitate measurement of black hole masses via well-established relations between line flux and width and black hole mass. \citeauthor{Simmons13} usde GZ2 to select their sample of bulgeless galaxies which selected against very massive black holes with unobscured emission. This is due to AGN bias, where a bright central point source can mimic a bulge, significantly affecting the bulge classification in Task 4 of the GZ2 decision tree (see Figure~\ref{fig:gztree}). In an investigation of the addition of a synthetic bright point source to an image of a bulgeless inactive galaxy, \citeauthor{Simmons13} found that a synthetic AGN with a luminosity of just $1/50\rm{th}$ of the host galaxy can reduce the $p_{\rm{no}~\rm{bulge}}$ classification by at least $50\%$. The GZ2 classifications are not utilised in this study for this reason, with the aim of selecting AGN hosted in bulgeless galaxies at all masses with broad emission lines that can be used to calculate black hole masses through virial assumptions (see Section~\ref{sec:bhmass}). Below the methods used to select both a disc dominated AGN sample along with a control sample of typical AGN host galaxies are described.

%%%%%%%%%%%%%%%%%%%%%%%%%%%%%%%%%%%%%%%%%%%%%%
\subsubsection{Selecting disc-dominated AGN host galaxies}\label{sec:selectAGN}% including where we get AGN luminosities from
%%%%%%%%%%%%%%%%%%%%%%%%%%%%%%%%%%%%%%%%%%%%%%

A sample of unobscured AGN with broad emission lines must first be selected. Black hole masses can then be measured from well-established correlations between emission line properties, such as the FWHM of the broadened $H\alpha$ emission line, and black hole masses \citep[e.g.,][]{gh07a, jiang11a,xiao11, peterson14}. 

These unobscured AGN have characteristic colours in multi-wavelength imaging, particularly in X-ray, optical and infrared bands \citep{kauffmann03b, dstern05, goulding09, kauffmann09, aird12, mendez13, azadi16, cowley16, harrison16}. Given the existence of all-sky surveys at many of the wavelengths relevant to the selection of unobscured AGN, it is possible to construct a large sample of sources identified as unobscured AGN with high likelihood.

An initial sample of AGN was selected using the W2R sample of \citet{edelson12}, comprised of $4,316$ sources identified using multi-wavelength data from the \emph{Wide-field Infrared Survey Explorer} \citep[\emph{WISE};][]{wright10}, Two Micron All-Sky Survey \citep[2MASS;][]{skrutskie06}, and \emph{ROSAT} all-sky survey \citep[RASS;][]{voges99}. This multi-wavelength photometric all-sky selection selects unobscured AGN at $>95$ per cent confidence \citep{edelson12}.


\begin{figure*}
\centering
\includegraphics[height=0.9\textheight]{agn/mosaic_all_diskdom_zsort.pdf}
\caption[SDSS images of DISKDOM sample]{Postage stamp SDSS images of the $96$ galaxies within the \textsc{bulgeless} sample for which SDSS spectra were available, sorted from lowest redshift ($z=0.03$; top left) to highest redshift ($z=0.24$; bottom right). The scale for each image is shown by the $5$'' ruler in the top left panel.}
\label{fig:exampleimages}
\end{figure*}


\begin{figure*}
\centering
\includegraphics[width=\textwidth]{agn/mosaic_INT_gal_only.pdf}
\caption[SDSS images of 5 galaxies observed with the IDS on the INT]{Postage stamp SDSS images of the $5$ galaxies observed with the IDS on the INT within the \textsc{bulgeless} sample. The scale for each image is shown by the $5$'' ruler in the left panel.}
\label{fig:INTimages}
\end{figure*}


Following this selection of $4,316$ sources, galaxies imaged by the Sloan Digital Sky Survey are then further sub-selected. There are $1,844$ W2R sources with positional matches having reported coordinates within $3^{\prime \prime}$ of a source in the SDSS \citep{york00} Data Release 8, a fraction consistent with the fractional area of the SDSS versus an all-sky catalog. %76 per cent of this sub-sample have measured redshifts, with a peak redshift distribution of $z \approx 0.12$.  90 percent of sources with redshifts have $z < 0.6$; the distribution has a long tail to $z_{\rm max} = 2.35$.

Each of the $1, 844$ SDSS ugriz colour images were then examined by two of my collaborators (Simmons \& Linttot) to identify bulgeless disc galaxies on the basis of clearly identifiable spiral arms, bars or obvious edge-on discs. $101$ galaxies were identified\footnote{$137$ bulgeless AGN were originally identified; 96 had spectra available from SDSS, 13 were observable in the 2014A semester with the Intermediate Dispersion spectrograph on the Isaac Newton Telscope, La Palma, Canary Islands, for which observing time was granted, with 5 $H\alpha$ detections made. 28 were observable in the 2014B semester for which observing time on the INT was rejected. Observations of these remaining galaxies currently without spectra, are ongoing at the Kast Double Spectrograph on the Shane $3\rm{m}$ telescope at the Lick Obervatory, California, USA.} which I shall refer to as the \textsc{bulgeless} sample. Figures~\ref{fig:exampleimages} \& \ref{fig:INTimages} collectively show the SDSS postage stamps for all galaxies in the sample, with all images showing the expected bright nebular emission of the unobscured AGN. 

%%%%%%%%%%%%%%%%%%%%%%%%%%%%%%%%%%%%%%%%%%%%%%
\subsubsection{Spectra}\label{sec:spectra}
%%%%%%%%%%%%%%%%%%%%%%%%%%%%%%%%%%%%%%%%%%%%%%

Of the 101 disc-dominated AGN host galaxies with SDSS imaging, 96 have spectra from SDSS Data Release 9 \citep{ahn12}, 23 of which were first identified as AGN by \cite{shen08} and \cite{edelson12}. Example spectra centred around the broad $H\alpha$ emission at $6562\AA$ for 5 of these SDSS galaxies are shown in Figure~\ref{fig:SDSSspectra}.

\begin{figure*}
\centering
\includegraphics[height=0.8\textheight]{agn/sample_sdss_spectra.pdf}
\caption[Optical SDSS spectra of 5 galaxies in the DISKDOM sample]{5 example SDSS spectra from with the corresponding measured redshift values shown. Each panel shows the same rest-frame wavelength range (bottom axis of each panel); observed wavelengths are shown on the top axis of each panel. All spectra show broadened  $H\alpha$ emission, confirming that the multi-wavelength AGN selection employed here efficiently selects unobscured AGN.}
\label{fig:SDSSspectra}
\end{figure*}

The broad $H\alpha$ emission for  {\notebsm 5 additional sources} was measured using long-slit spectra\footnote{Long slit spectra  allows for the measurement of the galaxy properties, such as stellar velocity dispersion and SFR, from the extracted galaxy spectrum. This is not possible using the fibre spectra provided by SDSS. A study of the $M_{\rm{BH}}$-$\sigma$ relation will be possible when more long-slit spectra of bulgeless AGN hosts are obtained in the future using the Kast Double Spectrograph.} techniques with the Intermediate Dispersion Spectrograph (IDS) on the Isaac Newton Telescope (INT) from 21st-23rd May 2014. I reduced these spectra using the standard reduction pipeline of Massey, Valdes \& Barnes (1992) using IRAF modules to debiase, dark subtract, flat field, calibrate, sky subtract, flux calibrate and finally extract spectra for the central regions of each galaxy. The redshift of these sources was also measured from the reduced spectra, using the peak of the broadened $H\alpha$ emission to measure $\lambda_{obs}$. These reduced spectra, centred around the broad $H\alpha$ emission at $6563\AA$, are shown in Figure \ref{fig:INTspectra} with calculated redshifts for the 5 galaxies observed.


\begin{figure*}
\centering
\includegraphics[height=0.8\textheight]{agn/int_spectra.pdf}
\caption[Optical spectra of 5 bulgeless galaxies observed on the INT with the IDS]{Reduced spectra from the IDS on the INT for the 5 galaxies observed. Each panel shows the same rest-frame wavelength range (bottom axis of each panel); observed wavelengths are shown on the top axis of each panel, with redshifts in the top right of each panel. All spectra one again show broadened $H\alpha$ emission, confirming that the multi-wavelength AGN selection employed here efficiently selects unobscured AGN.}
\label{fig:INTspectra}
\end{figure*}

% maybe some details about the spectral resolution?

Figure \ref{fig:redshifts} shows the redshift distribution of all 101 sources for which we have spectra; the mean redshift of the sample is $\left< z \right> = 0.129$, with the highest-redshift source having $z = 0.244$. 


\begin{figure}
\centering
\includegraphics[width=0.9\textwidth]{agn/z_distribution_101_sdss.pdf}
\caption[Redshift distribution of bulgeless galaxies]{Normalised redshift distribution for all 101 sources (solid) for which we have spectra, either from SDSS or measurements with the IDS on the INT. Also shown is the overall redshift distribution of SDSS DR7 in the relevant redshift range of our sources (dashed).
}
\label{fig:redshifts}
\end{figure}

\subsubsection{Selecting a Control Sample}

Since the majority of the galaxies in the \textsc{bulgeless} sample have been observed using SDSS, I constructed a control sample from the SDSS quasar catalog of \citet{shen11}. Using this sample I compiled a sample of 191 galaxies which were redshift matched to within $\pm5\%$ of the \textsc{bulgeless} sample. 7 of these galaxies were already part of the \textsc{bulgeless} galaxy sample and so these were removed.  I shall refer to the remaining 184 galaxies as the \textsc{qsocontrol} sample. 124 of the \textsc{qsocontrol} sample also had measured $(B/T)_r$ ratios from \citet[][matched with a $3''$ search radius, see Section~\ref{sec:galmass}; see also the middle panel of Figure~\ref{fig:discdomdist} for a comparison of $(B/T)_r$ in the \textsc{bulgeless} and \textsc{qsocontrol} samples]{simard11}. 

This provides a control sample of `typical' AGN host galaxies representative of the population in the redshift range probed in this study. 

%%%%%%%%%%%%%%%%%%%%%%%%%%%%%%%%%%%%%%%%%%%%%%
%
%
\subsection{Galaxy and Black Hole Properties}\label{sec:masses}
%
%
%%%%%%%%%%%%%%%%%%%%%%%%%%%%%%%%%%%%%%%%%%%%%%

In order to study the relation between the galaxies and their SMBHs in these disc dominated systems, their properties shall be compared to well-tested black hole-galaxy scaling relations. In the following section I therefore describe how the black hole masses, bolometric luminosities, Eddington ratios, photometry and total \& bulge stellar masses were derived for each galaxy in the \textsc{bulgeless} sample. 

%%%%%%%%%%%%%%%%%%%%%%%%%%%%%%%%%%%%%%%%%%%%%%
\subsubsection{Black hole mass estimates}\label{sec:bhmass}
%%%%%%%%%%%%%%%%%%%%%%%%%%%%%%%%%%%%%%%%%%%%%%

My goal is to estimate black hole masses independently of galaxy properties, such as $\sigma$, in order to compare them with established black hole-galaxy scaling relations. Thus, I instead utilise the established correlation between the black hole mass and the FWHM and luminosity in the broad \ha\ line of \citet{gh07a}: 
\begin{equation}\label{greeneho}
M_{BH} = (3.0^{+0.6}_{-0.5}) \times 10^6 \left ( \frac{L_{H\alpha}}{10^{42} ~\rm{erg}~\rm{s}^{-1}} \right )^{0.45\pm0.03} \left ( \frac{\rm{FWHM}_{H\alpha}}{10^{3}~\rm{km}~\rm{s}^{-1}} \right )^{2.06\pm0.06} M_{\odot},
\end{equation}
derived using a virial assumption. Using this virial assumption for the estimation of black hole masses is facilitated by the selection of unobscured AGN. Unobscured AGN have broad emission lines originating from within the black hole sphere of influence; this photoionized broad line region (BLR) can be used as a dynamical tracer of the black hole mass. The virial black hole mass \citep{peterson14} can be expressed simply as:
\begin{equation}\label{eq:virial}
M_{BH} = f \frac{R\Delta v^2}{G},
\end{equation}
where $\Delta v$ is the velocity dispersion of the emitting BLR, which is assumed to be spherical with radius $R$. The factor, $f = 0.75$ \citep{netzer90} then corrects for this simplifying assumption. The velocity dispersion of the BLR can be inferred from the FWHM of a broad line, such as $H\alpha$ or $H\beta$, and the radius inferred from the luminosity of the same broad line. This radius-luminosity relationship is calibrated using the more precise black hole mass measurement technique of reverberation mapping \citep{blandford82, peterson01, barth15} in which the radii are measured based on the observed delay between variations in the AGN continuum and the BLR emission \citep{kaspi05, bentz06}. Black hole masses derived with the virial method, under these simplifying geometric assumptions, have been shown to be accurate to within a factor of $\sim 3$ when compared to masses derived using the $M_{BH}$-$\sigma$ method \citep[][and see Section~\ref{agnsample}]{ferrarese01, nelson04, onken04}.

To obtain an estimate of the FWHM of the broadened $H\alpha$ lines, I performed spectral fitting on each of the SDSS and INT spectra described in Section \ref{sec:spectra} to recover narrow- and broad-line strengths and widths of the $\mha\ 6563$ \AA\ line, by using \gandalf\ \citep{sarzi06} {\notebsm to fit multiple simultaneous lines as well as the continuum of the spectra}. \gandalf\ is optimised for use with SDSS spectra and so using the program with the INT spectra required minimal data re-formatting; I logarithmically re-binned and de-redshifted the spectra. {\notebsm From the continuum-subtracted best fit provided by \gandalf, I determine the FWHM and line flux of the broad and narrow components of the \ha\ line simultaneously, once again employing \emcee\footnote{\url{dan.iel.fm/emcee/}}, the Python MCMC ensemble sampler by \cite{emcee13}, described in Chapter~\ref{chap:starpy}. 

The uncertainties reported by \emcee\ include an estimate of the uncertainty due to the separation of the narrow and broad line components in the \ha\ emission. The reported uncertainties on black hole masses include this source of uncertainty as well as the reported uncertainties in the black hole-broad line relation \citep{gh07a}. There are other sources of uncertainties, such as those involved in implicitly assuming the fixed geometric correction factor, $f=0.75$ \citep{netzer90} for each SMBH, the spectral noise, and the error introduced by assuming a Gaussian line profile for all measured broad lines. Determining uncertainties for the last two is outside the scope of this study; based on visual inspection of the line fits, the first is very small compared to the other uncertainties. These fits to the broad and narrow line \ha \ components in the INT spectra are shown in Figure \ref{fig:zoomspectra}.

\begin{figure}
\centering
\includegraphics[height=0.9 \textheight]{agn/int_zoom_spectra.pdf}
\caption[Zoom in on $H\alpha$ region of the spectra of 5 galaxies observed with the IDS on the INT]{Reduced spectra from the IDS on the INT for the 5 galaxies observed with the corresponding measured redshift values shown. Spectra are aligned with the broad $H\alpha$ emission line, the gaussian fits to which are shown by the dashed red line.  }
\label{fig:zoomspectra}
\end{figure}

The black hole masses for the 101 galaxies of the \textsc{bulgeless} sample range from {\notebsm $10^6 \mmsun \leq \mmbh \leq 2 \times 10^9 \mmsun$} and the distribution is shown in the left panel of Figure~\ref{fig:discdomdist} in comparison to those from the \textsc{qsocontrol} sample. 

\subsubsection{Bolometric Luminosities}\label{sec:eddratios}

Bolometric luminosities are calculated from the wavelength-dependent bolometric corrections shown in Figure 12 of \citet{richards06} using the conversion from the $12\mu m$ infrared luminosities, $ L_{12\mu m}$:
\begin{equation}
L_{bol} \approx 8 \times L_{12\mu m}.
\end{equation}
The infrared luminosity,  $L_{12\mu m}$, is calculated from the WISE W3 magnitudes, $M_{W3}$, for which all of the \textsc{bulgeless} sources have a detection, as follows:
\begin{equation}
L_{12\mu m} = \left(\frac{4\pi d^2}{10^{-2}~\rm{m}^2}\right)\left( \frac{c}{\lambda}\right) \left(\frac{F_{\nu, 0}}{1\times10^{23}~ \rm{Jy}} \right)10^{\left(\frac{M_{W3}}{-2.5}\right)}.
\end{equation}

The derived bolometric luminosities were first used to calculate Eddington ratios, $\lambda_{Edd}$, for the \textsc{bulgeless} sample (using the method outlined in Section~\ref{agnsample}), and then the black hole mass accretion rate, $\dot{m}$, using a simple matter to energy conversion:
\begin{equation}\label{eq:ltomdot}
L = \frac{E}{t} = \zeta\cdot \frac{mc^2}{t},
\end{equation}
where $\zeta =0.15$ \citep{elvis02}, is a lower limit\footnote{This estimate of the radiative effeciency factor, $\zeta$, is the lowest value necessary to ensure the X-ray background and the black hole energy density are consistent.} on the radiative efficiency factor (i.e. what fraction of the accreted mass can be turned into radiated energy). The black hole mass accretion rate, $\dot{m}$ is therefore calculated as:
\begin{equation}\label{eq:mdot}
\frac{m}{t} = \left(\frac{\dot{m}}{\rm{M}_{\odot}~\rm{yr}^{-1}}\right) = \left( \frac{1.58\times 10^{-26}}{\zeta} \right) \left( \frac{\rm{cm}~\rm{s}^{-1}}{c} \right)^2 \left( \frac{L_{\rm{bol}}}{\rm{erg}~\rm{s}^{-1}} \right).
\end{equation}


\subsubsection{Photometry}\label{sec:photo}

The AGN contribution to the magnitude of each galaxy, $m_{\rm{gal}}$, is estimated by subtracting the flux in the SDSS {\tt psfMag}, $m_{\rm{psf}}$, from the flux in {\notebsm {\tt modelMag}}, $m_{\rm{model}}$ in a given wave band, $b$ as follows:
\begin{equation}\label{galmag}
m_{\rm{b, gal}} = -2.5 \log_{10} \left[ 10^{\left(\frac{m_{\rm{b, model}}}{-2.5}\right)} - 10^{\left(\frac{m_{\rm{b, psf}}}{-2.5}\right)} \right],
\end{equation}
since the normalisation constants in the flux-magnitude conversion will be constant for different sized apertures in a given band. {\tt psfMag} is the best estimate of unresolved emission, while {\tt modelMag} is the optimal quantity for computing aperture-matched source colours\footnote{https://www.sdss3.org/dr10/algorithms/magnitudes.php}. The SDSS {\tt psfMag} is estimated by fitting a single PSF to the point source so that the entirety of the flux in the central PSF is attributed to the AGN. This assumption may therefore overestimate the contribution of the AGN to the central flux. However, given how bright the AGN are in the \textsc{bulgeless} sample (see Figure~\ref{fig:discdomdist}) and how extended the galaxies are (see Figures~\ref{fig:exampleimages}\&\ref{fig:INTimages}), this effect is assumed to be negligible in this case. A galaxy magnitude, $m_{b, gal}$, is estimated in both the SDSS $u$ and $r$ bands in order to determine the galaxy $u-r$ colour. 

A similar NUV galaxy magnitude can be calculated using the GALEX apertures, however matching these apertures to those provided by SDSS cannot be done with a large enough degree of accuracy to derive a reliable galaxy $NUV-u$ colour. This therefore means that \textsc{starpy} cannot return an accurate SFH for unobscured AGN host galaxies. However, the locations of the \textsc{bulgeless} sample on the optical colour-magnitude diagram can still be explored; this is studied in Section~\ref{sec:results} (see Figure~\ref{fig:cmdmdot}). 

%%%%%%%%%%%%%%%%%%%%%%%%%%%%%%%%%%%%%%%%%%%%%%
\subsubsection{Total stellar masses}\label{sec:galmass}
%%%%%%%%%%%%%%%%%%%%%%%%%%%%%%%%%%%%%%%%%%%%%%

Total stellar masses are calculated using the well-studied relation between stellar mass, absolute galaxy $r$-band magnitude, $M_{\rm{r, gal}}$, and $u-r$ galaxy colour \citep[corrected for galactic extinction;][using the galaxy magnitudes described in Section~\ref{sec:photo}]{schlegel98}, following the method of \citet[][see Section \ref{intro}]{Baldry06}. Uncertainties are propagated from the colour-magnitude relationship and due to the subtraction of the central AGN component. The average uncertainty on each measurement is $\sim0.3~\rm{dex}$. The distribution of the stellar masses calculated for the \textsc{bulgeless} sample is shown in the right panel of Figure~\ref{fig:discdomdist}. 
% not sure if this modelMag vs petroMag vs cModelMag will also be in e.g. Strauss et al. (2002)... must find someone to ask

\subsubsection{Bulge stellar masses}\label{sec:bulgemass}

Calculation of the bulge stellar mass for the \textsc{bulgeless} sample is more complicated than the total stellar mass calculation described in the previous section. The nuclear emission (as estimated via comparison of {\tt psfMag} to {\tt modelMag}) is generally between {\notebsm 20 to 200 per cent} of the entire galaxy-only emission. The presence of the luminous AGN therefore severely compromises the estimates of the bulge-to-total ratio, $(B/T)$, in the host galaxy provided by, e.g., the {\tt fracDeV} parameter reported in the SDSS catalogs. The {\tt fracDeV} parameter estimates that $\sim 80\%$ of the galaxies in this sample are pure \citet{devaucouleurs} bulges in the $r$-band, despite the fact that the sample was selected on the basis of clear visual signatures of dominant discs (see Figure \ref{fig:exampleimages}). None of the photometric parameters derived by the SDSS pipeline allow for the dual presence of an AGN and a host galaxy. Without such considerations the unresolved AGN light is likely to be attributed to the compact bulge component in a bulge-disc model fit \citep{simmons08,koss11} leading to an overestimate of the bulge stellar mass.


\begin{figure*}
\centering
\includegraphics[width=\textwidth]{agn/galfit_residuals.pdf}
\caption[Parametric fits and residuals for the 5 galaxies observed at the INT]{SDSS r-band images (left), with the 3 component parametric fits from GALFIT (middle) and the residuals (right; with the same scale as the original image) for the 5 galaxies observed with the IDS on the INT. Stable, but uncertain, bulge-to-total ratios were only recovered for the galaxies in the top two rows. In the bottom three rows the nuclear emission was too bright and image resolution too low to derive a reliable estimate of the bulge-to-total ratio.
}
\label{fig:galfit}
\end{figure*}

% wherever the first mention of a pseudo-bulge is, need to \citep{kormendy04}.
AGN-host decomposition based on 2-dimensional image fitting \citep[e.g.,][]{simard98,peng02,peng10} is more reliable \citep[e.g.][]{mclure99,urry00,mclure01,sanchez04,pierce07,gabor09,Simmons11,Simmons13,koss11}. However even in high-resolution \emph{Hubble Space Telescope (HST)} images \citep{simmons08} or SDSS imaging at $z \gtrsim 0.06$ \citep[][]{koss11,Simmons13} the recovered bulge-to-total ratio can be highly uncertain, particularly for disc-dominated galaxies with a very small bulge or pseudo-bulge \citep{kormendy04} component. While the AGN-host decompositions of the galaxies studied by \citet{Simmons13} recovered reliable bulge-to-total ratios for 11 of the 13 galaxies, their sources were at substantially lower redshift than the \textsc{bulgeless} sample (with the majority at $z < 0.08$), and their AGN significantly less luminous ($L_{bol} \lesssim 10^{44} \rm{erg}~\rm{s}^{-1}$, whereas in the \textsc{bulgeless} sample $L_{bol} \gtrsim 10^{44} \rm{erg}~\rm{s}^{-1}$, see Section~\ref{sec:eddratios}). 

Bulge-to-total fits were first attempted for the 5 galaxies in the \textsc{bulgeless} sample which were observed with the IDS on the INT using the \textsc{galfit} software \citep{peng02}; the results of which are shown in Figure~\ref{fig:galfit}. I used a S\'ersic light profile \citep{sersic68} to model bulge and disc components, defined by an effective radius, $R_e$, and light concentration index, $n$, as:
\begin{equation}\label{sersic}
I(R) = I_e \exp \left(  -b_n \left[  \left( \frac{R}{R_e}\right)^{1/n} -1 \right] \right),
\end{equation}
where $I_e$ is the intensity at the effective radius, $R_e$ and $b_n$ is a constant defined in relation to the S\'ersic index, $n$. Typical disc light profiles have $n\approxeq1$ and bulge profiles have $n\approxeq3$. Each of the galaxies observed with the INT were fitted with a disc, bulge and PSF component\footnote{Figure~\ref{fig:INTimages} reveals that at least one of these galaxies may contain a bar, however no bar component is fitted in this procedure. Incorporating a bar component into structural parameter fits has been shown that this is difficult even without the presence of an AGN \citep{kruk16}. Future studies, with higher resolution data (see Section~\ref{sec:hst}) will allow for the consideration of a bar component in structural fits.} (to account for the bright nuclear emission of the AGN). PSFs were extracted from the SDSS FITS images using the standard \texttt{read\_PSF} IDL code provided by the SDSS pipeline\footnote{\url{http://www.sdss.org/dr12/algorithms/read_psf/}}. Initial guesses of $n=2.5$ are used on the first pass of the \textsc{galfit} algorithm, which uses a $\chi^2$ minimisation method to determine the best fit S\'ersic index, effective radius, magnitude and position for each of the disc, bulge and PSF components. This first pass allows the positions of the components to be determined, which are then fixed on a second pass of the algorithm to ensure accurate magnitudes, radii and S\'ersic indices are then inferred. From these models, the \textsc{galfit} r band magnitudes of the bulge, $m_{r, \rm{bulge}}$, and disc, $m_{r, \rm{disc}}$, components were used to calculate the bulge-to-total ratio, $(B/T)_r$, as follows:
\begin{equation}\label{btratio}
(B/T)_r = \frac{10^{(\frac{m_{r, \rm{bulge}}}{-2.5})}}{\left[10^{(\frac{m_{r, \rm{bulge}}}{-2.5})} + 10^{(\frac{m_{r, \rm{disc}}}{-2.5})}\right]}.
\end{equation}

Highly uncertain bulge-to-total ratios were recovered for only {\notebsm 2} of the 5 galaxies (the top two rows in Figure~\ref{fig:galfit}). In the remaining 3 cases the nuclear emission was too bright and the image resolution too low for a reliable bulge-to-disc decomposition.  Detailed AGN host fits to the SDSS images in the rest of the \textsc{bulgeless} sample, which lie at similar redshifts, are therefore not likely to produce useful measurements of bulge masses. \emph{HST} imaging would enable these measurements, and although currently not available for the galaxies in this sample, observations are currently underway in Cycle 24 (Proposal ID: 14606, see Section~\ref{sec:hst}). 

Nevertheless, all we need in this study are conservative upper limits on the bulge masses of the \textsc{bulgeless} sample. Existing structural parameters from large-scale studies performing bulge-disc decompositions of SDSS galaxies can therefore constrain the maximum possible bulge contribution to the host galaxies. While such studies do not account for the presence of an AGN, their tendency to overestimate the bulge-to-total ratio as a result means that bulge masses derived from these quantities may be taken to be conservative upper limits.

\citet{simard11} have already fit multiple models to 1.12 million galaxies in the SDSS catalog to determine best-fit structural parameters for each galaxy. Their $r$-band bulge-to-total ratio {\notebsm of the best-fit model} is taken as an upper limit to the true bulge-to-total ratio of the \textsc{bulgeless} sample. To convert limits on bulge luminosities to limits on bulge masses, we assume the mass-to-light ratio of the bulge is equal to the mass-to-light ratio of the disc. This is a reasonable assumption for disc-dominated galaxies, where many of the ``bulge'' components, if present, are likely to be rotationally-supported pseudo-bulges \citep{kormendy04} with stellar populations similar to that of the disc {\notebsm \citep{graham01a}}.

The bulge-to-total ratio upper limits of the 89 galaxies in the \textsc{bulgeless} sample which were included in the \citet{simard11} study range from {\notebsm $0.13 \leq \left({\rm B/Tot}\right)_{r, \rm max} \leq 1.0$, with a mean value of 0.5}. Inspection of the morphologies of the galaxies shown in the images in Figure~\ref{fig:exampleimages} reveals how such a range in $\left({\rm B/Tot}\right)_{r, \rm max}$ is clearly an overestimate of the bulge contribution to these galaxies. Applying these bulge-to-total limits to the stellar masses derived in Section~\ref{sec:galmass} results in bulge mass upper limits of {\notebsm $3 \times 10^9 \mmsun < {\rm M_{bulge}} < 7 \times 10^{10} \mmsun $}. The distribution of bulge-to-total mass ratios in the \textsc{bulgeless} sample are shown in the middle panel of Figure \ref{fig:discdomdist}. The bulge-to-total mass ratios and the black hole masses (left panel Figure~\ref{fig:discdomdist}) of the \textsc{bulgeless} sample are smaller in general than the \textsc{qsocontrol} sample, however the stellar masses (right panel Figure~\ref{fig:discdomdist}) are found within the range of stellar masses estimated for the \textsc{qsocontrol} sample. 

All of these derived galaxy and black hole properties of the \textsc{bulgeless} sample will now be used to determine where these galaxies lie on typical galaxy-black hole scaling relations in Section~\ref{sec:intresults}.

\begin{figure*}
\centering
\includegraphics[width=\textwidth]{agn/diskdom_mbh_btot_stellar_mass_distributions.pdf}
\caption[Galaxy and black hole properties of the \textsc{bulgeless} sample in comparison to the \textsc{qsocontrol} sample]{Distributions of black hole mass (left), upper limits on the r-band bulge-to-total mass ratio from \citet[][middle]{simard11} and stellar mass (right) of the \textsc{bulgeless} sample (solid) in comparison to the \textsc{qsocontrol} sample (dashed).}
\label{fig:discdomdist}
\end{figure*}



%%%%%%%%%%%%%%%%%%%%%%%%%%%%%%%%%%%%%%%%%%%%%%
%
%  
\subsection{Results}\label{sec:intresults}
%
% 
%%%%%%%%%%%%%%%%%%%%%%%%%%%%%%%%%%%%%%%%%%%%%%



The total stellar mass and estimated bulge masses (see Section \ref{sec:galmass}) are plotted against the black hole masses for the \textsc{bulgeless} sample in Figures~\ref{fig:totvsbh} \& \ref{fig:bulgevsbh} respectively. I fit a multiple linear regression model to both of these relations using an inference method which encompasses the uncertainties on both $x$- and $y$-dimensions and the intrinsic scatter in the data. The full method is outlined in \citet{kelly07} and is publicly available as a \emph{Python} module \textsc{linmix}\footnote{\url{http://linmix.readthedocs.org/}}. A brief outline of the method is provided below. 

A multiple linear regression model assumes a simple linear relationship between two independent variables, $x$ and $y$, where both variables are unknown, with added noise, $\epsilon$, from an unknown unobserved random variable (e.g. intrinsic scatter, random errors). In the \textsc{limmix} package this is modelled with the following form:
\begin{equation}\label{2dlinemodel}
\eta = \alpha + \beta * x_i + \epsilon
\end{equation}
\begin{equation}\label{2dlinemodel2}
x = x_i + x_{err}
\end{equation}
\begin{equation}\label{2dlinemodel3}
y = \eta + y_{err}.
\end{equation}
Here $\alpha$ and $\beta$ are the regression coefficients to be inferred, $x_{err}$ is the error on the measured values $x_i$, and $y_{err}$ is the error on the measured values $\eta$. $\epsilon$ is assumed to be normally-distributed, centred around zero, with a variance $\sigma^2$. $x_{err}$ and $y_{err}$ are also assumed to be normally-distributed and centred around zero with variances $\sigma_x^2$ and $\sigma_y^2$, respectively and covariance $xy_{cov}$. This linear regression method can also incorporate the upper limits on the bulge mass measurements of the 89 SDSS galaxies measured by \citet[][see Section \ref{sec:galmass}]{simard11},  by treating them as `censored values' \citep[see Section 7.2 of][]{kelly07}, shown by the solid line in Figure~\ref{fig:bulgevsbh}.

Using \textsc{linmix}, I also fit to the observations of 30 early-type galaxies from \citet{haringrix04}. This is shown in Figure~\ref{fig:totvsbh} by the red shaded region for the total stellar mass against black hole mass in comparison to the linear regression fit to the \textsc{bulgeless} sample, shown by the grey shaded region. The two fitted relations are consistent with each other, despite the fact that the early-type galaxies from \citeauthor{haringrix04} have very different evolutionary histories to the galaxies in the \textsc{bulgeless} sample.  

Similarly in Figure~\ref{fig:bulgevsbh} the fit to the upper limits on the bulge stellar mass against the black hole mass of the \textsc{bulgeless} sample is shown by the grey shaded region in comparison to the fit to the early-type galaxies from \citet{haringrix04}. Since the \citeauthor{haringrix04} galaxies are early-types the same value for the total and bulge stellar masses are used across Figures~\ref{fig:totvsbh}\&\ref{fig:bulgevsbh}. The fits to the two samples are not consistent with each other with galaxies in the \textsc{bulgeless} sample, which contain either no bugle or a pseudo-bulge, preferentially lying above the established relationship between black hole and bulge stellar mass shown for the bulge dominated galaxies of \citet{haringrix04}. 

\begin{figure*}
\centering
\includegraphics[width=\textwidth]{agn/mass_bh_total_mass_fit_linmix_fit.pdf}
\caption[Black hole stellar mass relation for the \textsc{bulgeless} sample]{Total stellar mass against the black hole mass of the 101 galaxies, including those observed by SDSS (crosses), with the IDS on INT (blue squares) and detections from \citet[open cirlces]{Simmons13}. The best fit line to the data points and two-dimensional errors from linear regression is shown (solid line) with $\pm3\sigma$ (grey shaded). I also show the best fit found using this same method to the early-type galaxies of \citet{haringrix04} (dashed line) with $\pm3\sigma$ (red shaded) and the measured values shown by the red circles. Despite the fact that these galaxies are predominantly disc dominated they are found in the same region of parameter space as the bulge dominated systems used to derive the \citet{haringrix04} relationship (see discussion in Section \ref{sec:intdiscussion}).
}
\label{fig:totvsbh}
\end{figure*}

\begin{figure*}
\centering
\includegraphics[width=\textwidth]{agn/mass_bh_bulge_limits_INT_simmons13_measurements_linmix_fit.pdf}
\caption[Black hole bulge mass relation for the \textsc{bulgeless} sample]{The upper limits on the calculated stellar bulge masses are plotted against the black hole mass with the best fit to these upper limits and two-dimensional errors using linear regression methods (solid line) shown with $\pm3\sigma$ (grey shaded). The dashed line shows the fit if the upper limits are not treated as such. I also show the best fit found using this same method to the early-type galaxies of \citet{haringrix04} (dashed line) with $\pm3\sigma$ (red shaded) and the measured values shown by the red circles. Despite the fact that these galaxies are predominantly disc dominated they will are most likely to lie above the \citet{haringrix04} relationship found for bulge dominated systems (see discussion in Section \ref{sec:intdiscussion}).
}
\label{fig:bulgevsbh}
\end{figure*}

\begin{figure}
\centering
\includegraphics[width=\textwidth]{agn/mass_bh_mdot_with_all_errors_shen_11.pdf}
\caption[Black hole mass against mass accretion rate for the bulgeless AGN sample]{Black hole mass against mass accretion rate for the 101 \textsc{bulgeless} galaxies, including those observed by SDSS (crosses) and with the IDS on INT (squares). We also show detections from {\notebsm Simmons et al. (2013)} (open circles) and those from the redshift matched sample of {\notebsm Shen et al. (2011)}. For reference we show lines of example Eddington ratios of $\Lambda_{Edd}$ = 1 (dashed),  $\Lambda_{Edd}$ = 0.1 (dot-dashed) and $\Lambda_{Edd}$ = 0.01 (dotted).
}
\label{fig:mbhvsbol}
\end{figure}


I consider how the black hole mass relates to the black hole accretion rate (estimated using Equation~\ref{eq:mdot}) of the \textsc{bulgeless} sample, compared with the \textsc{qsocontrol} sample in Figure \ref{fig:mbhvsbol}. The galaxies of the \textsc{bulgeless} sample have both lower black hole masses and lower accretion rates in comparison to the \textsc{qsocontrol} sample and these are lower in proportion such that the Eddington ratios are very similar, as shown by the distributions in Figure \ref{fig:eddratioshen}. In fact, the Eddington ratios of the redshift matched \textsc{qsocontrol} sample are, on average, lower than that for the \textsc{bulgeless} sample. I performed a Kolmogorov–Smirnov (KS) test on these samples and found that I can reject the null hypothesis that the disc dominated galaxies Eddington ratios are drawn from the same distribution as the \textsc{qsocontrol} sample but not for the entire quasar sample of \citet{shen11}.

Within the \textsc{qsocontrol} sample, 108 galaxies were morphologically classified by the Galaxy Zoo 1 project \cite{lintott08, Lintott11}. Galaxy Zoo 1 did not ask whether a galaxy was a disc, merely whether a galaxy was spiralling clockwise or anti-clockwise. Therefore we can approximate the the likelihood of a galaxy being a disc through the combined spiral vote fraction, $p_{CS}$, where a higher combined spiral vote fraction indicates a higher likelihood of a galaxy being a disc. All of the 108 galaxies of the \textsc{qsocontrol} sample are found to have a debiased combined spiral vote fraction  of $p_{CS} < 0.5$ and a mean value of $\left<p_{CS} \right> = 0.17$.  The \textsc{qsocontrol} sample is therefore mainly comprised of bulge dominated galaxies unlike the \textsc{bulgeless} sample. %The slightly higher accretion rates are therefore occurring in the AGN of the galaxies in the \textsc{bulgeless} sample than in a bulge dominated \textsc{qsocontrol} sample. 

The colour-magnitude diagram for the \textsc{bulgeless} sample is also shown in Figure~\ref{fig:cmdmdot} in comparison to the SDSS sample and green valley definition from \citet{Baldry04}. The unobscured AGN host galaxies of the \textsc{bulgeless} sample are found across the entirety of this parameter space, however those with higher black hole mass accretion rates are found preferentially in the green valley and at brighter r-band magnitudes. 


\begin{figure}
\centering
\includegraphics[width=0.8\textwidth]{agn/edd_ratio_z_matched_shen_2011_compare.pdf}
\caption[Eddington ratio distribution of the bulgeless AGN sample]{Normalised distributions of logarithmic Eddington ratio for the sample of 101 disc dominated galaxies (solid line), compared with that for the redshift matched sample from Shen et al. (2011; dashed line) and the entire SDSS QSO sample (dot-dashed line). I also provide the p-values of a 2 sample KS test between the disc dominated sample and each of the quasar samples. We reject the null hypothesis that the two redshift matched samples are drawn from the same underlying population but cannot reject this hypothesis when comparing the fractional black hole accretion rates of the \textsc{bulgeless} samples and the entire SDSS QSO sample of \citet{shen11}.  
}
\label{fig:eddratioshen}
\end{figure}


\begin{figure}
\centering
\includegraphics[width=\textwidth]{agn/CMD_DISCDOM_coloured_accretion_rate.pdf}
\caption[Colour-magnitude diagram for the DISCDOM sample, coloured by black hole mass accretion rate]{Optical colour-magnitude diagram showing the positions of the \textsc{bulgeless} sample (circles) in comparison to the SDSS DR7 sample from \citet{Baldry04}. The galaxies of the \textsc{bulgeless} sample are coloured by their black hole mass accretion rate, $\dot{m}$ (see Equation \ref{eq:mdot}). The definition between the blue cloud and red sequence  from \citet{Baldry04} is shown by the solid line (as defined in Equation~\ref{eqgv}) with $\pm~1\sigma$ shown by the dashed lines. The \textsc{bulgeless} sample are found across the colour magnitude diagram, however those with higher mass accretion rates are found preferentially in the green valley and at brighter r-band magnitudes.}
\label{fig:cmdmdot}
\end{figure}


%%%%%%%%%%%%%%%%%%%%%%%%%%%%%%%%%%%%%%%%%%%%%%
%
%  
\subsection{Discussion}\label{sec:intdiscussion}
%
%
%%%%%%%%%%%%%%%%%%%%%%%%%%%%%%%%%%%%%%%%%%%%%%


The relatively large number of purely disc galaxies of the \textsc{bulgeless} sample hosting growing SMBHs provides a very powerful probe of the simultaneous evolution of galaxies and their black holes driven without typical bulge-forming mechanisms. Despite the rarity of the morphology of these \textsc{bulgeless} galaxies, they are found in common areas of black-hole-galaxy scaling relations, as seen in Figures~\ref{fig:totvsbh}, \ref{fig:bulgevsbh} \& \ref{fig:mbhvsbol}, frequented by typical AGN host galaxies of all morphologies (they also lie in the common regions of the local $M_{BH}-\lambda_{Edd}$ plane shown in Figure 1 of the review paper of \citealt{alexander12}). Significant black hole growth has occurred up to masses of $M_{BH}\sim10^8 - 10^9~\rm{M}_{\odot}$ in the \textsc{bulgeless} sample whilst the disc dominated nature of the galaxy has been preserved. Simulations have repeatedly shown that mergers with mass ratios larger than 1:10 (i.e. mergers where the mass of the satellite is greater than $10\%$ of the main galaxy's mass) will form a classical bulge \citep{walker96, hopkins11c, tonini16}; so how is this substantial mass growth possible in the absence of such a major merger dominated (and minor merger limited) formation history?

Using Equation~\ref{eq:mdot}, the black hole mass accretion rates are estimated to lie in the range $0.01 \leq \dot{m} \leq 0.59 ~\rm{M}_{\odot}~\rm{yr}^{-1}$. Simulations by \citet{crockett11}, and more recently by \citet{diteodoro14}, show that such accretion rates are completely achievable by cold accretion of minor satellites with mass ratios less than 1:10 (i.e. the mass of the satellite galaxy is less than $10\%$ of the main galaxy mass). However, there is also evidence that such accretion rates may also be possible via merger free, secular processes alone. One such secular process, is bar driven gas inflow into the central regions of galaxies. Simulations of barred galaxies have repeatedly shown that gas inflow rates due to a morphological bar range from $\sim0.1-\rm{few} ~\rm{M}_{\odot}~\rm{yr}^{-1}$ \citep{sakamoto96, maciejewski02, regan04, lin13} and may increase to up to $\sim 7 ~\rm{M}_{\odot}~\rm{yr}^{-1}$ \citep{friedli93} with increasing bar length, bar strength and axis ratio. 

These simulations however, struggle to show that the gas funnelled to the central regions of the galaxy is actually accreted into the central few parsecs, and instead often accumulates in a nuclear ring wherein it causes a starburst \citep{regan04}. Similarly, no observational correlation has yet been found between the presence of a bar and and that of an AGN either locally \citep{ho97, malkan98, erwin02, lee12,cisternas13} or out to $z\sim1$ \citep{cheung15}. I estimate a bar fraction, $f_{\rm{bar}}$ in the \textsc{bulgeless} sample (which is by no means complete), by visual inspection of the SDSS ugriz images (see Figure~\ref{fig:exampleimages}), of $f_{\rm{bar}} \sim 0.42$; a lower limit due to the edge on nature of some of the galaxies in the \textsc{bulgeless} sample. In agreement with the studies above, this is no higher than the local bar fraction observed in the general local galaxy population \citep{masters11a}.

Using these derived black hole mass accretion rates, the time required to grow the black holes in the \textsc{bulgeless} sample from a seed mass of $10^2 ~\rm{M}_{\odot}$ can also be derived. This calculation assumes that the black holes have grown at the currently observed bolometric luminosity, since the mass at which that luminosity was the Eddington luminosity and prior to this underwent Eddington limited growth. This means that the accretion rates calculated, $\dot{m}$, will be the maximum rates at which the black holes have grown over their lifetimes. This is a conservative assumption but gives an estimate of the total time these black holes would need to spend in actively growing phase if the calculated rates are typical of black holes residing in disc dominated host galaxies. The mean (median) time taken for the black holes of the \textsc{bulgeless} sample to grow from a seed black hole mass is $\sim 1.68 ~\rm{Gyr}$ ($\sim 0.37 ~\rm{Gyr}$). These times are considerably less than a Hubble time, with the SMBHs in the \textsc{bulgeless} sample needing to spend only $\sim 10\%$ of their lifetimes in a growing phase to give the black hole masses currently measured. This is in agreement with the fraction of time predicted by simulations and observed by others \citep{kauffmann03, hao05, hopkins06, fiore12, Simmons13}.

A time of $\sim 38 ~\rm{Gyr}$ was calculated for two of the most massive black holes in the \textsc{bulgeless} sample, which have very low current Eddington ratios (see Figure~\ref{fig:mbhvsbol}). Since this time frame is well beyond the current estimates for the age of the Universe \citep[$\sim13.8 ~\rm{Gyr}$;][]{planck16}, this value is clearly an overestimate of the time taken to grow these two black holes. Since AGN are believed to go through a duty cycle of varying accretion rates throughout their lifetimes \citep{martini01, yu02, schawinski15}, then we can assume that these two black holes were accreting at a higher rate at some point in their history. If these two black holes are assumed to have grown within the median (mean) time derived for the rest of the \textsc{bulgeless} sample, then the past accretion rate will have been on the order of, $\dot{m} \sim 7.95 ~(1.73) ~\rm{M}_{\odot}~\rm{yr}^{-1}$. As discussed earlier, similar gas inflow rates caused by the presence of a bar have been seen in simulations \citep{friedli93}; suggesting once again that despite having large masses, these black holes could in theory be grown by secular processes. Unfortunately, the two host galaxies are at too high a redshift to allow the detection of a morphological bar in the SDSS ugriz image. 

The black hole mass accretion rates are also shown in Figure~\ref{fig:cmdmdot}, wherein the locations of the \textsc{bulgeless} sample on the optical colour magnitude diagram are shaded by $\dot{m}$. Those galaxies hosting black holes with higher mass accretion rates are found preferentially in the green valley and at brighter r-band magnitudes. This once again supports the arguments of Section~\ref{sec:agnfeedback} that feedback from these AGN could cause galaxies to quench, and therefore transition through the green valley, due to a sustained period of high mass accretion.

Figures~\ref{fig:mbhvsbol} \& \ref{fig:eddratioshen} show how the Eddington ratios of the \textsc{bulgeless} sample are higher than those in the bulge dominated \textsc{qsocontrol} sample (which has mean combined spiral vote fraction from GZ1 of $\left<p_{CS} \right> = 0.17$), so that the null hypothesis that the two Eddington ratio distributions are drawn from the same parent sample can be rejected. However, the same null hypothesis cannot be rejected for the full non-redshift matched AGN sample of \citet{shen11} which spans a redshift range of $0.06 < z < 5.46$. The Eddington ratios of the \textsc{bulgeless} sample are therefore higher than bulge dominated systems in the same redshift range, and instead are consistent with black hole accretion rates occuring at earlier cosmic times. So, despite having merger free evolutionary histories, black hole growth in the \textsc{bulgeless} galaxies is occurring at a higher rate than in typical local AGN host galaxies.  

The black hole masses of the disc dominated host galaxies in the \textsc{bulgeless} sample are not expected to correlate in the same way to their stellar masses as those in bulge dominated galaxies, if different dynamical histories lead to different mechanisms for black hole growth. However, Figure~\ref{fig:totvsbh} shows how the black hole and total stellar masses of the \textsc{bulgeless} sample occupy the same region of parameter space as the bulge dominated elliptical galaxies used to derive the \citet{haringrix04} relationship. Similarly Figure~\ref{fig:bulgevsbh} shows how the black hole masses of the \textsc{bulgeless} sample (which contain either no bulge, or a possible small pseudo bulge) lie well above the \citet{haringrix04} relationship, particularly when the upper limits on \textsc{bulgeless} bulge masses are taken into account. In other words, given what we know about black hole growth mechanisms, the black holes in these disc dominated systems are $\sim1-2~\rm{dex}$ more massive than they should be, given the mass (or lack thereof) of their bulge component. 

This is in agreement with the results of \citet{Simmons13} who found a similar excess in the black hole masses of $\sim 1.5~\rm{dex}$ and $\sim 2~\rm{dex}$ for the two measured black hole masses in their sample of 13 pure disc galaxies. Both this result and the results shown in Figures~\ref{fig:totvsbh} \& \ref{fig:bulgevsbh} at first seem to be in contradiction with previous works which find that galaxies with pseudo-bulges have lower black hole masses than predicted by typical scaling relations \citep[see work by][]{greene08, hu09, jiang11a, mathur12, ho14}. However, all these studies are biased by their sample selection methods, first selecting based on black hole mass to produce a sample of low mass black holes ($M_{BH} < 10^6 ~\rm{M}_{\odot}$) within which they hoped to find bulgeless or pseudo-bulge morphologies. Now, with the larger \textsc{bulgeless} sample shown in Figure~\ref{fig:bulgevsbh}, we can see that the fitted relationship between their black hole masses and bulge mass upper limits (solid black line), intersects with the relationship derived for the bulge dominated \citet{haringrix04} sample at $M_{BH} \sim 10^{6.4} ~\rm{M}_{\odot}$. At this point, the relationship predicts that for disc dominated galaxies the black hole masses will indeed be less than those predicted for bulge dominated systems, as concluded by the studies referenced above. 

Splitting the AGN host population by morphology in this way however, leads to biased conclusions. As discussed in Chapter~\ref{chap:morph}, the strength of the \textsc{popstarpy} method lies partly due to the fact that no thresholds are applied to the GZ morphological vote fractions, allowing the dominance of intermediate quenching rates across the colour magnitude diagram to be revealed (see Section~\ref{sec:morphresults}). Similarly, if one does not ``\emph{discriminate}'' against morphology in the black hole mass-bulge mass plane and fit a linear regression model to galaxies in both the \textsc{bulgeless} (with proper consideration of upper limits) and \citet{haringrix04} samples, the result is consistent with a vertical line in the bulge-black hole mass plane. This suggests that there is perhaps no intrinsic correlation between black hole mass and stellar bulge mass across the full morphological spectrum of galaxies. 

This argument is supported by the agreement in Figure~\ref{fig:totvsbh} between the relationships derived in the total stellar mass-black hole mass plane for the \textsc{bulgeless} and bulge dominated \citet{haringrix04} samples. This agreement arises despite the two extremes in galaxy formation histories. This indicates that the mechanisms driving the dynamical and morphological structure of the galaxy may not be fundamental to the growth of the black hole. The black hole-galaxy relations observed across the $M_{BH}$-$\sigma$, $M_{BH}$-$M_{\rm{bulge}}$ and $M_{BH}$-$M_{*}$ planes, although demonstrating a correlation, have never implied a \emph{causation}. All of these parameters however, share mutual correlations to the overall gravitational potential of the dark matter halo of the galaxy \citep{booth10, volonteri11}, suggesting the true cause of the black-hole galaxy scaling relations is an outcome of hierarchical galaxy evolution \citep{jahnke11}, regardless of the merger history of the galaxy. 
 

\section{Conclusions}\label{sec:agnconclusion}

In Section \ref{sec:agnfeedback} I used morphological classifications from the Galaxy Zoo 2 project to determine the morphology-dependent SFHs of a population of $1,244$ Type 2 Seyfert AGN host galaxies, in comparison to an inactive galaxy population, via a Bayesian analysis of an exponentially declining SFH model. Using \textsc{popstarpy} I determined the population densities for the time and exponential rate that quenching occurs and find clear differences in the distributions, between \textsc{inactive} and \textsc{agn-host} galaxy populations and for host galaxies with AGN of different Eddington ratios. The main findings were:

\begin{enumerate}[(i)]
\item Quenching at early times is observed within the \textsc{inactive} population (see right panels of Figure~\ref{time}), where the population density is roughly constant until recent times where the distribution drops off at earlier times with increasing mass. This is evidence of downsizing within the \textsc{inactive} galaxy population, which is also seen in the \textsc{agn-host} smooth weighted population. This implies that AGN feedback is not responsible for the cessation of star formation within a proportion of these galaxies, as this quenching has occurred prior to the triggering of the current AGN.

\item Slow, early quenching is also observed in the disc weighted \textsc{agn-host} population (dashed lines bottom left panels of Figures~\ref{time} \& \ref{rate}) and so this SFH challenges the typical merger driven co-evolution of luminous black holes and their host galaxies.

\item Rapid quenching, possibly caused by the AGN itself through negative feedback, is the most dominant history within the low mass (left top panel Figure~\ref{rate}) and high Eddington ratio (bottom left panel of Figure~\ref{eddratiosplit}) \textsc{agn-host} population. This quenching history is particularly apparent for the smooth-weighted \textsc{agn-host} population, supporting the hypothesis that a merger, having caused a morphological transformation to a smooth galaxy, can also trigger an AGN, causing feedback and cessation of star formation on rapid timescales (initially proposed in Chapter~\ref{chap:morph}). Further work is required to determine if the AGN is indeed the cause of the quenching seen. 

\item The prevalence of star forming AGN host galaxies, combined with the dominance of rapid, recent quenching seen across the \textsc{agn-host} population allows us to consider that either: (i)  the AGN are the cause of the rapid quenching observed but only in gas-poor host galaxies where they can have a large impact, (ii) the AGN are a consequence of another quenching mechanism but can also be triggered by other means which do not cause quenching, or (iii) the SFR of a galaxy can recover post-quench and return to the star forming sequence after a few Gyr.

\end{enumerate}

In Section \ref{intbulgeless} I studied how black holes can grow in galaxies with merger free evolutionary histories, by investigating where AGN in disc dominated galaxies lie on typical black-hole galaxy scaling relations. Despite the fact that these disc dominated galaxies have different dynamical histories to bulge-dominated and elliptical shaped systems, they are found to lie in the same regions of parameter space are $\sim1-2~\rm{dex}$ more massive than they should be, given the mass (or lack thereof) of their bulge component. The main findings were:
\begin{enumerate}[(i)]
\item Significant black hole growth has occurred up to masses of $M_{BH}\sim10^8 - 10^9~\rm{M}_{\odot}$ in the \textsc{bulgeless} sample whilst the disc dominated nature of the galaxy has been preserved.

\item Eddington ratios of the \textsc{bulgeless} sample are higher than those in the bulge dominated \textsc{qsocontrol} sample; despite having merger free evolutionary histories, black hole growth in the \textsc{bulgeless} galaxies is occurring at a higher rate than in typical local AGN host galaxies. 

\item Those galaxies hosting black holes with higher mass accretion rates are found preferentially in the green valley and at brighter r-band magnitudes. This once again supports the arguments of Section~\ref{sec:agnfeedback} that feedback from these AGN could cause galaxies to quench, and therefore transition through the green valley, due to a sustained period of high mass accretion.

\item Figures~\ref{fig:totvsbh} \& \ref{fig:bulgevsbh} show how the galaxies of the \textsc{bulgeless} sample occupy the same region of the $M_*$-$M_{\rm{BH}}$ and $M_{\rm{bulge}}$-$M_{\rm{BH}}$ parameter spaces, respectively, as the bulge dominated \citet{haringrix04} sample. Given what we know about black hole growth mechanisms, the black holes in these disc dominated systems are $\sim1-2~\rm{dex}$ more massive than they should be, given the mass (or lack thereof) of their bulge component. This suggests that there is perhaps no intrinsic correlation between black hole mass and galaxy bulge mass across the full morphological spectrum of galaxies and that the true cause of the black-hole galaxy scaling relations may be due to mutual correlations to the overall gravitational potential of the dark matter halo of the galaxy.


\end{enumerate}

It is therefore clear that AGN can have a large impact on the evolution of their host galaxies, including causing quenching directly through AGN feedback across the entire population; which I have demonstrated here for the first time. 


\chapter{The influence of the group environment}\label{chap:env}

\section{Introduction}\label{sec:intro}
 
 So far, I have considered mergers, interactions, morphology and AGN as causes of quenching across the galaxy population. While it is clear that these mechanisms can strongly affect the SFHs of galaxies, the density of a galaxy's environment is thought to be the largest influence on the evolutionary path the galaxy can take. 
 
 The galaxy environment as a cause of quenching was proposed due to the correlation of morphology \citep{dressler80, smail97, poggianti99, postman05, Bamford09}, star formation rate (SFR) and the quenched galaxy fraction \citep{kauffmann03, Baldry06, peng12, darvish16} with environmental density. Star forming disc galaxies tend to be located in low-density environments with quiescent elliptical galaxies in more dense environments. Although these correlations were originally interpreted as causation, recent evidence from simulations suggests that the environment may not be the dominant quenching mechanism in the galaxy lifecycle \citep{ref, ref} with \citet{skibba09} suggesting that the morphology-density relation is mostly due to the colour-density relation as seen in \citet{pimbblet02}. Perhaps instead, the correlation of increased galaxy quenched fractions with environment is due to a superposition of the effects of mergers \& interactions and both mass \& morphology quenching. 
  
In order to isolate the cause of the density-morphology and density-SFR correlations, I need to observe how morphology and galaxy quenching timescales change in group environments with different properties in comparison to the field. Here, I use the group environment to tackle this problem, as this is a more typical environment for a galaxy than the relatively rare rich cluster environment \citep{carlberg04}. I construct a sample of both group and field galaxies and once again use \starpy\ to determine the quenching time and rate to describe a simple SFH for a galaxy given its photometry. However, dense environments are messy with many factors at work, whose effects are difficult to discern. These individual mechanisms effects would be washed out across the population density distributions produced by \textsc{popstarpy} and so I do not employ this method in this Chapter. 

Instead, I have simplified the analysis in order to separate the effects of the different quenching mechanisms contributing to the galaxy population in the group environment. I aim to determine the following: (i) How does the environment influence the detailed morphological structures of a galaxy?  (ii) Is quenching which is directly caused by the environment occurring in galaxy groups?

%In Section~\ref{sec:data} I describe the group catalog employed and highlight the results in Section~\ref{sec:results}. 
 
\section{Data and Methods}\label{sec:data}

\subsection{Group Identification}\label{sec:groups}

The selection of a robust cluster or group catalog is a thesis in itself, with many studies attempting this across the SDSS \citep{} and other large surveys \citep{}. Difficulties arise in removing projection effects, understanding the selection function used, covering large ranges in mass and redshift, and dealing with spectral fibre collisions \citep[see comprehensive review by][for in depth discussion]{postman02}. Various different methods have been employed to achieve robust group identification including clustering algorithms \citep[e.g.][]{nichol01, miller05}, galaxy colour modelling \citep{annis04}, adaptive filter halo modelling \citep{yang05, yang07} and friends-of-friends algorithms \citep{goto05, merchan05, berlind06}. 

Each group finding algorithm has to be tested for purity (how contaminated the groups are by non-members) and completeness (how often are true members excluded from a group). \citet{campbell15} compared the purity and completeness of two of the most frequently used group catalogs of \citet[][a friends-of-friends algorithm]{berlind06} and \citet[][a halo modelling algorithm]{yang07} and concluded that no sample could achieve perfect purity or completeness. Despite the different algorithms employed to identify group galaxies in the two catalogs, \citeauthor{campbell} found that the two catalogs are remarkably similar; however the \citeauthor{yang07} catalog has higher purity of satellites at lower halo masses (i.e. the low halo mass groups are less contaminated by non-members). For this reason the \citeauthor{yang07} is the most commonly used catalog across environment studies using the SDSS \citep{citationbomb}, however I find that when cross matched with the \textsc{gz2-galex} sample (with a $3``$ search radius) only $38$ galaxies (of $176,604$ galaxies with more than 2 galaxies in a group) are identified. This is most likely due to the necessity for GALEX NUV photometry. 

Instead I therefore use the \citet{berlind06} catalogue, which uses a friends-of-friends algorithm to identify group and cluster galaxies in the SDSS. This was cross matched to the \textsc{gz2-galex} sample and limited to $z < 0.1$ (to ensure GALEX completeness of the red sequence; see \citealt{wyder07, yesuf14}) to give $14,199$ group galaxies. Centrals were selected as the most massive galaxy in a group \citep[as in][]{yang07, yang09, pasquali10} with all other galaxies in a group designated as satellites and the number of galaxies in a galaxy's group, $N_{group}$ was recorded. 

The projected group centric radius, $R$, of all satellite galaxies was calculated from the projected separations of the co-ordinates of a satellite from its central; this was then converted to $\rm{kpc}$ from a consideration of the observed redshift of the central galaxy. In order to compare groups of different sizes, the virial radius is often used as a normalisation constant to this projected group centric radius. Here I use a proxy to the virial radius \citep[see][]{navarro95}, $R_{200}$, the radius within which the group mass overdensity is 200 times the critical density, $\rho_{\rm{crit}}(z)$, as defined by \citealt{finn05}:
\begin{equation}\label{eq:overdense}
200\rho_{\rm{crit}}(z) = \frac{M_{cl}}{\frac{4}{3}\pi R_{200}^3}
\end{equation}

where $M_{cl}$ is the mass of the group. \citeauthor{finn05} then use the $z$ dependance of the critical density and the virial mass to relate the line-of-sight velocity, $\sigma_x$, to the group mass so that $R_{200}$ becomes:
\begin{equation}\label{eq:r200}
R_{200} = 1.73 \left ( \frac{\sigma_x}{1000 \rm{km}~\rm{s}^{-1}} \right) \cdot \frac{1}{\sqrt{\Omega_{\Lambda} +\Omega_o(1+z)^3}} ~ h_{100}^{-1} ~\rm{Mpc}, 
\end{equation}

where $\sigma_x$ is the line-of-sight velocity dispersion of a group. This was calculated from the proper velocities of each galaxy, $i$ as defined in \cite{danese80}:
\begin{equation}\label{eq:propervel}
v_i = c \cdot \frac{z_i - z_{group}}{1 + z_{group}}.
\end{equation}
The line-of-sight velocity dispersion, $\sigma_x$, was then calculated for a group as the standard deviation of the velocity dispersions $\sqrt{(v_i - \left< v_i\right>)^2}$. These calculations resulted in a sample of $3,468$ centrals and $10,731$ satellites within a projected group centric radius range of $0 < R/R_{200} < 25$ and $z < 0.084$ which shall be referred to as the \textsc{gz2-berlind} sample. Note that for a galaxy (central or satellite) to be included in the \textsc{gz2-berlind} sample, the rest of its group does not, however the properties of that group are still retained by the included galaxy. 

Unlike in previous Chapters, here I will specifically focus on galaxies that are below the star forming sequence (SFS). I therefore select galaxies that are $1\sigma$ below the SFS, giving $4,629$ satellite and $2,314$ central galaxies which will collectively be referred to as the \textsc{gz2-group} sample. These galaxies are shown in the panels of Figure \ref{fig:sfrmass} and can be seen to lie below the SFS of star formation. 

\begin{figure}
\centering
\includegraphics[width=\textwidth]{environment/sfr_mass_quenched_centrals_satellites_gz2_group.pdf}
\caption[Stellar mass-SFR plane for the centrals and satellites of the \textsc{z2-group} sample]{The stellar mass-SFR plane showing central (left; red contours) and satellite (right; blue contours) in the \textsc{z2-group} sample. In both panels the entire SDSS sample from the MPA-JHU catalog is shown by the grey contours. The definition of the SFMS from \cite{peng10} at $\overline{z} = 0.053$ (solid line, the mean redshift of the \textsc{gz2-group} sample) with $\pm1\sigma$ (dashed lines) is shown.}
%KS Test between distributions?
\label{fig:sfrmass}
\end{figure}


I also compare the \textsc{gz2-berlind} and \textsc{gz2-group} samples with a measurement of the projected neighbour density, $\Sigma_N = N/4\pi d_N^2$, from \cite{Baldry06} where $d_N$ is the distance to the $N^{\rm{th}}$ nearest neighbour. In this work I use the estimates of \cite{bamford09} of a local galaxy density, $\Sigma$, determined by averaging $\log\Sigma_N$ for $N = 4$ and $N=5$. $90\%$ of the \textsc{gz2-berlind} sample have $\log\Sigma > -0.8$ (the threshold quoted by \citealt{Baldry06} below which field galaxies are found), suggesting high completeness of the sample. The distributions of $\log\Sigma$ for star forming and quenching/quenched centrals and satellites in the \textsc{gz2-berlind} sample are shown in Figure~\ref{fig:sigmadist}. Star forming galaxies tend to reside in less dense local environments than their quenching/quenched counterparts. The satellite galaxies as a whole also seem to occupy denser local environments than centrals, however on investigation this seems to arise because the satellites in the \textsc{gz2-berlind} sample reside in groups with larger $N_{group}$ than the centrals. This is once again likely due to the necessity for GALEX colours. 

\begin{figure}
\centering
\includegraphics[width=\textwidth]{environment/SIGMA_density_sf_q_cent_sat.pdf}
\caption[Local environment density distributions of central and satellite galaxies]{Local environment density, $\log\Sigma$, distributions of star forming (black) and quenching/quenched (red) central (left) and satellite (right) galaxies in the \textsc{gz2-group} sample.}
%KS Test between distributions?
\label{fig:sigmadist}
\end{figure}


\subsection{Field sample}\label{sec:field}

For all galaxies in the \textsc{gz2-galex} sample, I calculated the smallest projected group centric radii, $R/R_{200}$, from each of the central galaxies in the \citet{berlind06} catalog (regardless of whether a central was included in the \textsc{gz2-berlind} sample) and selected candidate field galaxies as those with (i) $R/R_{200} > 25$ and (ii) $\log\Sigma < -0.8$ \citep[the threshold on the local environment density which selects field galaxies as defined by][]{Baldry06}. I chose to use both of these environmental density measures ensure a pure sample of candidate field galaxies.

This sample of field galaxy candidates was then matched in redshift and stellar mass firstly to the central galaxies of the \textsc{gz2-group} sample to give $2,309$ field galaxies with $z < 0.084$ which will be referred to as the \textsc{gz2-cent-field} sample. In this work I shall focus on galaxies which are either quenching or quenched and are more than $1\sigma$ below the SFS. This encompasses $1,596$ field galaxies with $z < 0.084$ which will be referred to as the \textsc{gz2-cent-field-q} sample. It will be used as a control sample when investigating the trends with central galaxy properties of the inferred quenching parameters. The redshift distribution of the \textsc{gz2-cent-field-q} sample is shown in comparison to the distribution of central galaxies in the \textsc{gz2-group} sample in left panel of Figure~\ref{fig:zcompare}. %SDSS images of a random selection of galaxies from the \textsc{gz2-group} and \textsc{gz2-cent-field-q} samples are shown ordered by their GZ2 debiased vote fraction in Figure~\ref{fig:mosaic}. %KS test between samples?

Secondly, the field galaxy candidates were then matched in redshift and stellar mass to the satellite galaxies of the \textsc{gz2-group} sample to give $5, 004$ field galaxies with $z < 0.084$ which will be referred to as the \textsc{gz2-sat-field} sample. These galaxies in the \textsc{gz2-sat-field} sample will be used as a control when investigating the morphological trends of satellite galaxies with environment. Note that the sample is not restricted to being $1\sigma$ below the SFS in this case. The redshift distribution of the \textsc{gz2-sat-field} sample is shown in comparison to the distribution of satellite galaxies in the \textsc{gz-group} sample in the right panel of Figure~\ref{fig:zcompare}

We obtain SFRs and stellar velocity dispersions of galaxies for all of the field samples described above from the MPA-JHU catalogue \citep{kauffmann03, brinchmann04}. 

\begin{figure}
\centering{
\includegraphics[width=0.45\textwidth]{environment/redshift_cent_field.pdf}
\includegraphics[width=0.45\textwidth]{environment/redshift_sat_field.pdf}}
\caption{Redshift distributions of central (left) and satellite galaxies (right) in the \textsc{gz2-group} sample (black solid line) in comparison the redshift matched \textsc{gz2-cent-field-q} (left; blue dashed line) and \textsc{gz2-sat-field} samples (right; blue dashed line).}
%KS Test between distributions?
\label{fig:zcompare}
\end{figure}

\subsection{Morphological fractions}\label{sec:morphfrac}

I once again utilise the GZ2 vote fraction to quantity the morphology of galaxies in the \textsc{gz2-group} sample, in order to investigate the morphological trends with group radius. As in previous Chapters, I shall utilise $p_{\rm{disc}}$ and $p_{\rm{smooth}}$ but will also use $p_{\rm{bar}}$, $p_{\rm{bulge}}$ and $p_{\rm{merger}}$ to calculate the bar, bulge and merger fractions in the \textsc{gz2-group} sample respectively. 

Fractions are calculated considering the number of barred ($p_{\rm{bar}} > 0.5$; see \citealt{masters11a, cheung13}) and bulged ($p_{\rm{bulge}} > 0.5$) galaxies over the number of disc galaxies ($p_{\rm{disc}} > 0.43$, $p_{\rm{edge\_on, no}} > 0.715$, $N_{\rm{edge\_on, no}} > 20$; see Table~\ref{table:gz2thresholds}, originally printed in \citealt{GZ2}) in the \textsc{gz2-group} satellite sample. The merger fraction considers the number of merging galaxies ($p_{\rm{merger}} > 0.4$; see \citealt{darg10a}) over the number of galaxies in the \textsc{gz2-group} satellite sample. 

\subsection{Time since quenching}\label{sec:delta}

The SFHs of all galaxies in both the \textsc{gz2-group} and \textsc{gz2-field} samples were analysed using \starpy\; the output of \starpy\ is probabilistic in nature, providing the posterior probability distribution across the two-parameter space for an individual galaxy. Whereas in Chapters~\ref{chap:morph} \& \ref{chap:agn} the \textsc{popstarpy} method was used to combine and weight the individual distributions to give an overall distribution representing the population of galaxies, due to the complex nature of the group environment, in this Chapter I instead take the 50th percentile walker position of an individual posterior probability distribution to give the most likely $t_{q}$ and $\tau$ for each galaxy. \

This simplifies the output from \starpy  ~for each galaxy from a probability distribution to just values, with uncertainties which encode the information about the shape of the individual galaxy's SFH posterior probability distribution. In this Chapter I will look for trends in the time since quenching onset, $\Delta t$, for a given galaxy by calculating {\bf $\Delta t = t_\mathrm{obs} - t_{q}$}. I will observe how this quantity changes with group properties, including the halo mass, velocity dispersion, number of group members and relative velocity of a satellite galaxy. 


\section{Results}\label{sec:results}

\subsection{Mass dependance with radius}

Since morphological features have been shown to be dependent on the stellar mass of a galaxy \citep[e.g. the increase in the bar fraction with stellar mass][]{skibba12}, before investigating trends in the morphology with group radius in the \textsc{gz2-group} sample, the mass dependence on the group radius must be considered. This is shown in Figure~\ref{fig:massdep}. The average mass is roughly flat and consistent with the median field value with increasing group radius, until the most central group radius bin at $R \sim 0.01~R_{200}$. This trend is present for both morphologies, with early-type galaxies showing a larger increase in the average stellar mass. This trend with stellar mass should be kept in mind as a potential bias to the results presented in the following sections. 

\begin{figure}
\centering{
\includegraphics[width=0.45\textwidth]{environment/mass_trend_with_log_radius_compare_field.pdf}}
\caption[Average mass with group radius in the GZ2-GROUP sample]{The average stellar mass as a function of radius from the group centre. The shaded regions show the $\pm1\sigma$ in each bin of $R/R_{200}$. The average stellar mass of the \textsc{gz2-sat-field} sample is also shown (blue solid line) with $\pm1\sigma$ (blue dashed line).}
\label{fig:massdep}
\end{figure}

\subsection{Dependence of detailed morphological structure with environment}

\begin{figure}
\includegraphics[width=0.46\textwidth]{environment/p_disc_trend_with_log_radius_field_compare.pdf}
\includegraphics[width=0.46\textwidth]{environment/p_smooth_trend_with_log_radius_field_compare.pdf}
\caption{Mean GZ vote fraction for disc (top) and smooth (bottom) galaxies in the \textsc{gz2-group} sample binned in projected group centric radius, normalised by $R_{200}$, a proxy for the virial radius of a group. The shaded region shows $\pm1\sigma$ on the mean vote fraction. The mean vote fraction of the \textsc{gz2-sat-field} sample are also shown (blue solid lines) with $\pm1\sigma$ (blue dashed lines).}
\label{fig:morphradius}
\end{figure}

\begin{figure}
\centering{
\includegraphics[width=0.46\textwidth]{environment/bar_fraction_over_disc_trend_with_log_radius_sat_matched_field_cand.pdf}}
\caption{Bar fraction (number of barred disc galaxies over number of disc galaxies) in the \textsc{gz2-group} sample binned in projected group centric radius, normalised by $R_{200}$, a proxy for the virial radius of a group. The shaded region shows $\pm1\sigma$ on the bar fraction. The bar fraction of the \textsc{gz2-sat-field} sample is also shown (blue solid line) with $\pm1\sigma$ (blue dashed line).}
\label{fig:barradius}
\end{figure}

\begin{figure}
\centering{
\includegraphics[width=0.46\textwidth]{environment/merger_fraction_trend_with_log_radius_compare_sat_field_cand.pdf}}
\caption{Merger fraction in the \textsc{gz2-group} sample binned in projected group centric radius, normalised by $R_{200}$, a proxy for the virial radius of a group. The shaded region shows $\pm1\sigma$ on the merger fraction. The merger fraction of the \textsc{gz2-sat-field} sample is also shown (blue solid line) with $\pm1\sigma$ (blue dashed line).}
\label{fig:mergerradius}
\end{figure}


I perform an initial sanity check on th \textsc{gz2-group} sample by recreating the morphology-density relation \citep{dressler80} in Figure \ref{fig:morphradius} which shows the mean disc and smooth vote fractions as a function of group radius. The mean disc vote fraction decreases from the mean field value (blue line) under $1$ virial radius, in agreement with previous studies on the morphology-density relation \citep{dressler80, smail97, poggianti99, postman05, Bamford09}. The extensive morphological classifications provided by GZ2 now allow for the investigation of how more detailed morphological structure is affected by the group environment.  

Figure \ref{fig:barradius} therefore  shows how the bar fraction (number of barred disc galaxies over the number of disc galaxies) increases towards the centre of the group population, significantly over the field fraction (blue solid line). Figure \ref{fig:mergerradius} shows how the merger fraction does not significantly deviate from the field fraction (blue solid line) until within a virial radius. Similarly, the left panel of Figure \ref{fig:bulgeradius} shows how those galaxies identified as having no bulge or a just noticeable bulge are less common in the inner regions of the cluster (left panel), whereas the fraction of galaxies with obvious or dominant bulges increases with decreasing projected distance from the centre of the group.

%Figure \ref{fig:sfrradius} shows how the SFR of the \textsc{gz2-group} sample does indeed decline with decreasing group centric distance, significantly below the mean SFR of the \textsc{gz2-field} sample shown by the blue dashed line. This is in agreement with the results of \cite{gomez03} who observe a similar decline in SFR with group centric radius in SDSS clusters (see for example, Figure 6 in \citealt{gomez03}). This coincides with the morphological fraction changes seen in Figures~\ref{fig:morphradius}-{\ref{fig:merger radius} in support of the conclusions of \citet{smethurst15} that quenching is morphologically dependent. 


%\begin{figure}
%\includegraphics[width=0.46\textwidth]{environment/sfr_trend_with_log_radius_field_matched_blue_dashed_hlines_gomez_03_rv_not_r200.pdf}
%\caption{Median $H\alpha$ derived star formation rates of satellite galaxies in the \textsc{gz2-group} sample, binned in projected group centric radius, normalised by $R_{200}$, a proxy for the virial radius of a group.  The shaded region shows the SFRs encompassed by $50\%$ of the population in a given bin. The median SFR of the \textsc{gz2-sat-field} sample is shown (blue solid line) along with the 25th and 75th percentiles (blue dashed lines).}
%\label{fig:sfrradius}
%\end{figure}

\subsection{Quenching times in the group environment}

\begin{figure*}
\centering{
\includegraphics[width=0.85\textwidth]{environment/min_max_bulge_fraction_trend_with_log_radius_sat_field_cand.pdf}}
\caption{Fraction of galaxies with none/just noticeable bulge classifications (left) and with obvious/dominant bulge classifications (right) in the \textsc{gz2-group} sample binned in projected group centric radius, normalised by $R_{200}$, a proxy for the virial radius of a group. The shaded regions shows $\pm1\sigma$ on the bulge fractions. The bulge fractions of the \textsc{gz2-sat-field} sample are also shown (blue solid lines) with $\pm1\sigma$ (blue dashed lines).}
\label{fig:bulgeradius}
\end{figure*}

\begin{figure}
\centering{
\includegraphics[height=0.8\textheight]{environment/time_since_quenching_M*_Mh_mu.pdf}
\caption{The time since quenching onset ($\Delta t = t_{obs} - t_{q}$) binned in projected group centric radius, normalised by $R_{200}$, for satellite galaxies (crosses) split into bins of stellar mass (top), stellar mass of the corresponding central galaxy (middle; a proxy for halo mass of a group) and the stellar mass ratio ($\mu_* = M_*/M_{*,c}$, bottom). The corresponding values for central galaxies (squares, plotted at $0.01 R/R_{200}$) and galaxies in the \textsc{gz2-cent-field-q} sample (circles, plotted at $25 R/R_{200}$) are shown where calculable and connected by the dashed lines to help guide the eye. The shaded regions show the $\pm1\sigma$ on $\Delta t$ in each bin of $R/R_{200}$.}
\label{fig:timesinceradius}}
\end{figure}

\begin{figure}
\centering{
\includegraphics[height=0.8\textheight]{environment/time_since_quenching_Ngroup_delv_sigma.pdf}
\caption{The time since quenching onset ($\Delta t = t_{obs} - t_{q}$) binned in projected group centric radius, normalised by $R_{200}$, for satellite galaxies (crosses) split by the number of group members ($N_{group}$, top), absolute relative velocity of the satellite to its central galaxy ($|\Delta v|$, middle) and velocity dispersion ($\sigma_*$, bottom). The corresponding values for central galaxies (squares, plotted at $0.01 R/R_{200}$) and galaxies in the \textsc{gz2-cent-field-q} sample (circles, plotted at $25 R/R_{200}$) are shown where calculable and connected to the satellite values by the dashed lines to help guide the eye. The shaded regions show the $\pm1\sigma$ on $\Delta t$ in each bin of $R/R_{200}$.}
\label{fig:timesinceradiusvel}}
\end{figure}


With the output from \starpy~ I look at the trends in the time since quenching onset ($\Delta t = t_{obs} - t_{q}$, see Section \ref{sec:starpy}) with group radius for satellite galaxies and central galaxies in the \textsc{gz2-group} sample, compared with galaxies in the \textsc{gz2-field} sample. This is shown in Figures \ref{fig:timesinceradius} \& \ref{fig:timesinceradiusvel} with the \textsc{gz2-group} sample binned by various group properties:
\begin{itemize}

\item{Stellar mass, $M_*$: in the top panel of Figure \ref{fig:timesinceradius} galaxies in the \textsc{gz2-group} sample are split by their stellar mass and a clear trend for increasing time since quenching onset with increasing stellar mass for satellite, central and field galaxies can be seen. The central galaxies (shown by the square points at $\sim 0.01 R/R_{200}$) appear to have quenched more recently than the inner satellites (at $\sim0.1R/R_{200}$) 
of the same mass.}

\item{Halo mass: in the middle panel of Figure \ref{fig:timesinceradius} I use a proxy for halo mass by splitting the \textsc{gz2-group} sample by the stellar mass of the corresponding central galaxy of a group, $M_{c,*}$ and find a clear trend for increasing time since quenching onset with increasing stellar mass of the group central for satellite, central and field galaxies.}

\item{Mass ratio, $\mu_* = M_*/M_{*,c}$: the stellar mass ratio of the satellite to its central galaxy. In the bottom panel of Figure \ref{fig:timesinceradius} we show the time since quenching of the \textsc{gz2-group} split into bins of $\log_{10}\mu_*$. The change in $\Delta t $ with projected group centric radius occurs more steeply (particularly beyond $\sim$ a virial radius) for satellite galaxies with much smaller masses than their group central ($-2.0 < \log_{10}\mu_* < -1.0$, shown by the blue curve).}

\item{Number of group members, $N_{group}$: the top panel of Figure \ref{fig:timesinceradiusvel} shows that there is no trend with time since quenching onset with increasing $N_{group}$ for satellite galaxies. The central galaxies (shown by the square points at $\sim 0.01 R/R_{200}$) however, do show a trend for increasing time since quenching with $N_{group}$.}

\item{Relative velocity, $|\Delta v|$: in the bottom panel of Figure \ref{fig:timesinceradiusvel} I split the satellite galaxies of the \textsc{gz2-group} sample into bins of relative velocity to their central galaxies. There is no trend with time since onset of quenching with increasing relative velocity for galaxies in the group environment.}

\item{Stellar velocity dispersion, $\sigma_*$: the bottom panel of Figure \ref{fig:timesinceradiusvel} shows the time since quenching of the \textsc{gz2-group} sample split into bins of $\sigma_*$. The stellar velocity dispersion shows the largest trend in $\Delta t$ for satellite galaxies of all of the group properties studied, with galaxies with the smallest stellar velocity dispersions having quenched more recently, and vice versa. }
\end{itemize}

Across all the panels in Figures \ref{fig:timesinceradius} \& \ref{fig:timesinceradiusvel} a general trend for increasing time since quenching onset with group radius can be seen. As earlier, in Figures \ref{fig:morphradius}$-$\ref{fig:bulgeradius} significant differences from the field averages arise inside $\sim$ one virial radius. 


\section{Discussion}\label{sec:disc}

\subsection{The role of mergers in quenching in the group environment}\label{sec:rolemergerenv}

The merger classification in GZ2 has been shown to preferentially identify major mergers \citep{darg10a}; wheras, bulge formation in disc galaxies is often associated with (minor) merger driven evolutionary histories \citep{croton06, tonini16}.  Although we see evidence for an enhanced merger fraction in the inner regions of the group environment in Figure~\ref{fig:mergerradius}, the bulge fractions in Figure~\ref{fig:bulgeradius} vary much more significantly from the field value than the merger fraction. This suggests that minor mergers may be more dominant than major mergers in the group environment, particularly at $R/R_{200} > 0.5$. 

If mergers are an important evolutionary mechanism for satellite galaxies, as the morphological evidence in Figures~\ref{fig:mergerradius} \& \ref{fig:bulgeradius} suggests, we would expect to see a difference in the quenching histories of satellites residing in groups with a larger number of members. However, the top panel of Figure \ref{fig:timesinceradiusvel} shows there is no trend with time since quenching onset with increasing $N_{group}$ for the satellite galaxies. This suggests that mergers are not the dominant quenching mechanism for satellite galaxies.

For central galaxies however, this is not the case. The central galaxies (shown by the square points at $\sim 0.01 R/R_{200}$ in the top panel of Figure \ref{fig:timesinceradiusvel}), do show a trend for increasing time since quenching with increasing $N_{group}$. Therefore, we can postulate that the larger the number of group members, the more important mergers are for a central's evolutionary history. 

This idea is supported by the result shown in the top panel of Figure~\ref{fig:timesinceradius} when the \textsc{gz2-group} is split into bins of stellar mass. We can see that centrals of a given mass have quenched more recently than the inner satellites (at $\sim0.1R/R_{200}$) of a given mass. This suggests that an episode of more recent star formation may have occurred in the central galaxies but not in the inner satellites. This is once again suggestive of a merger dominated evolutionary history for central galaxies, as mergers are thought to cause an energetic burst of star formation which in turn quenches the remnant galaxy \citep{hopkins05, treister12, pontzen16}. If the central galaxies have quenched more recently than the inner satellites this could be a more recent burst of star formation caused by a merger under this regime. 


\subsection{The role of mass quenching in the group environment}\label{sec:rolemassenv}

The role of mass quenching in the group environment is apparent in the top panel of Figure~\ref{fig:timesinceradius} and the bottom panel of Figure~\ref{fig:timesinceradiusvel} where the \textsc{gz2-group} satellites are binned by stellar mass and central velocity dispersions respectively. We see a trend for increasing time since quenching with increasing stellar mass and velocity dispersion for centrals, satellites and field galaxies. This is suggestive of mass quenching occurring across the entire galaxy population irrespective of environmental density, supporting the work of \citet{Peng10, gabor10, peng12} and \citet{darvish16}.

\subsection{The role of morphological quenching in the group environment}\label{sec:rolemorphenv}

The results in Figure \ref{fig:barradius}, showing the increasing bar fraction towards the central group regions which is consistent with results that show that bars themselves may be the cause of morphological quenching through the funnelling of gas toward the central regions of galaxies \citep{athanassoula92b, sheth05,masters10c}. However we must consider whether the environment itself may play a role in triggering the disk instabilities which can produce a morphological bar. Indeed harassment and tidal interactions have been shown to both promote and inhibit bar formation dependent on the stellar mass \citep{noguchi88, moore96, skibba12}.  In this case, morphological quenching would be occurring but indirectly due to environmental quenching. 

\subsection{The role of the environment in quenching}\label{sec:roleenv}

Across all panels of Figures ~\ref{fig:timesinceradius}-\ref{fig:timesinceradiusvel} a trend for increasing time since quenching onset with decreasing projected group centric radius is present. This suggests that the environment does directly cause quenching throughout the infall time of a galaxy in a group. Those galaxies closer in, fell into the group earlier and as they did so they started to quench, giving rise to a larger inferred $\Delta t$.

The results in Figure \ref{fig:timesinceradius} suggest more massive halos therefore have a greater impact on the star formation histories of their satellites than less massive halos. The stellar mass of the central galaxy is used as a proxy for halo mass which is correlated with both (i) the gravitational potential of the group \citep{ref} and (ii) the temperature of the IGM \citep{ref}.

It is thought that higher mass halos have hotter inter galactic medium (IGM) temperatures \citep{ref} which can then have a greater impact impact on a galaxy through RPS of cold gas \citep{ref}. If RPS is indeed a dominant environmental quenching mechanism we should see a trend in $\Delta t$ with the speed of a satellite galaxy relative to the group central.  However in the middle panel of Figure \ref{fig:timesinceradiusvel} we see that this is not the case. 

This suggests that any environmental quenching mechanism responsible for the trends seen Figures~\ref{fig:timesinceradius} \& \ref{fig:timesinceradiusvel} are not correlated with satellite velocity. This therefore rules out RPS as the dominant environmental quenching mechanism, in support of the conclusions of simulations by\citet{emerick16, fillingham16} and observations by \citet{mcgee14}. 

The dominant environmental quenching mechanism in the group environment must therefore be correlated with the group potential. This suggests that satellite galaxies may be most affected by gravitationally driven environmental effects, such as harassment and galaxy-galaxy interactions or by thermal evaporation of the galaxy halo due to the hot IGM. 


\section{Conclusions}\label{sec:conc}

Mergers, mass quenching morphology quenching and environmental quenching are all prevalent in galaxy groups. The environment does play a role in quenching galaxies through a mechanism proportional to the halo mass of the group but not proportional to the relative speed of the satellite galaxy to its central galaxy. This suggests that ram pressure stripping is not the dominant environmental quenching mechanism. All of these mechanisms discussed coalesce to give rise to the distributions in galaxy properties we see across the Universe through their constant interplay across cosmic time. 


\chapter{Discussion}

In this Chapter I will discuss how my results compare to others I have not yet touched upon and discuss their broader implications (Section~\ref{sec:bigpic}). I will also discuss the implications my results have for the proper use of morphology in galaxy evolution studies (Section~\ref{sec:usemorph}). I will then discuss future work with \starpy\ (Section~\ref{sec:future}) and its application to IFU spectral data (Section~\ref{sec:IFU}). I will end with a reflection on the power of Hubble Space Telescope imaging in comparison to SDSS imaging. 

\section{The Big Picture}\label{sec:bigpic}

In Chapter~\ref{chap:morph} I discussed how the results suggested that quenching rates with $\tau < 1.5~\rm{Gyr}$ must be caused by mechanisms which can transform morphology from a disc to an elliptical. However this does not infer an immediate transition from disc dominated to bulge dominated. Work by the \textsc{ATLAS}$^{\rm{3D}}$ team \citep{cappellari11} showed that kinematic discs are found in the majority of the visually elliptical population \citep{emsellem11} with $\sim7$ times the number of \emph{fast rotators} (with kinematic discs) than \emph{slow rotators} \citep[with dispersion dominated kinematics see][]{cappellari07, emsellem07}. 

The authors of these works discuss how dry major mergers, which can destroy the disk dominated nature of a galaxy \citep{toomre72}, in rapid timescales may be able to produce fast rotators \citep{duc11, naab14}. I find across the red sequence population in Figure~\ref{red_s} that $12\%$ of the population density lies below $\tau <0.2 \rm{Gyr}$, in agreement with predictions that between $14-17\pm5\%$ of ellipticals are slow rotators \citep{emsellem11, stott16}, suggesting that these are the quenching rates which might give rise to a slow rotator. This therefore also provides an estimate for the percentage of the galaxy population which have undergone a dry major merger.  

Conversely fast rotators are theorised to be formed by the slow build up of a galaxy's bulge over time, until it eventually overwhelms the disc. This growth is thought to occur via wet major and minor mergers \citep{duc11} which can produce a bulge dominated quenched galaxy, which would be visually classified as an elliptical (or a smooth galaxy by GZ users). Previously, major mergers were considered the only mechanism which could achieve such a result, however I speculated in Chapter~\ref{chap:morph} that all mechanisms with $\tau < 1.5 ~\rm{Gyr}$ can cause a morphological change during quenching. This gradual, relatively slower evolutionary history may help to explain why studies have found that the current major merger rate is much lower than expected given the observed red sequence ($\sim 10\%$, \citealt{Lotz11}). The larger IFU studies of MaNGA, SAMI and CALIFA will allow for larger populations of slow and fast rotators to be identified so that the relative dominance of wet and dry mergers across the elliptical population can be determined more accurately. 

There is also a wealth of literature studying the rare \citep[$<1\%$;][]{Wong12, wild16} `missing link' post-starburst (PSB) galaxy phase. They are thought to have undergone an intense, unsustainable period of star formation in the recent past which has then rapidly quenched \citep{dressler83, abraham96b, poggianti99, goto03, goto05b, goto07}. Such a phase can add a significant fraction \citep[$\sim10\%$;][]{wild10} to the stellar mass of a galaxy and is thought to be linked with merger activity \citep{zabludoff96, blake04, goto05b, yang08, pawlik16}. \citet{Wild09} estimated that gas rich major merger induced starbursts, which simultaneously cause a morphological change, could account for $38\%$ of the growth of the red sequence at $z\sim0.7$. Figure~\ref{red_s} shows that $\sim40\%$ of the smooth weighted red population undergoes quenching at rates less than $0.7~\rm{Gyr}$, suggesting that these quenching rates could be associated with this post-starburst transition phase. However, this needs to be reconciled with the idea that wet major mergers are also thought to produce fast rotators. Future work studying the SFHs of PSB galaxies with \starpy\ applied to IFU data (see Section~\ref{sec:IFU}) may help to further constrain the estimated contribution to the growth to the red sequence by this rare evolutionary phase. 

A galaxy property which I have also overlooked in this work, but which is often investigated in quenching studies, is the stellar mass surface density of a galaxy; alternatively the ``concentration'' of a galaxy, which is found to correlate with SFR \citep{barro13b, whitaker16}. The build up of a galaxy's bulge is thought to be able to stabilise a disk against collapse and effectively stop it from forming stars. This is classed as a type of morphological quenching and is effective over a few $\rm{Gyr}$ \citep{Fang13} even if external gas is still fed to a galaxy. This slower quenching track of bulge dominated galaxies may help to explain the slow quenching rates observed across the red and green smooth weighted population densities seen in Figures~\ref{red_s} \& \ref{green_v}. The results in Chapter~\ref{chap:morph} suggest that slower quenching of smooth galaxies is occurring in up to $40\%$ of the smooth weighted population (see Figure~\ref{red_s}). By observing the population densities of the red galaxy population, the processes which both quench and grow the bulge simultaneously (such as a merger or interaction) and those which only grow the bulge and the SFR is consequently lowered slowly by morphological quenching can be separated. However, even in the former case, morphological quenching may help in either speeding up the process or in keeping the galaxy quenched post interaction. This is supported by the finding of \cite{abramson16} who found there is no threshold at which density triggered quenching occurs, but that denser systems typically redden faster than less dense galaxies. This suggests a symbiotic partnership between minor mergers and morphological quenching is needed to achieve true quiescence, similar to the partnership between starvation and stripping discussed in Chapter~\ref{chap:env}. 

This mutually beneficial partnership between two mechanisms is also reminiscent of the idea that without AGN feedback a major merger cannot fully quench a galaxy, as discussed in Chapters~\ref{chap:morph} \& \ref{chap:agn}. Figure~\ref{rate} shows galaxies in the \textsc{agn-host} population don't always quench at the rapid rates caused by major mergers, suggesting that a slow co-evolution of black hole and host galaxy can occur. Left alone, AGN are only efficient as a quenching mechanism in low mass galaxies where they can have a greater impact. In combination with a major merger however, they can have catastrophic effects \citep{conselice03, springel05b, hopkins08a}. These effects are therefore easily detectable, leading to the initial theories for the links between AGN and mergers \citep{merritt01, hopkins06b, hopkins08a, hopkins08b, peng07, jahnke11}. However, by studying the population as a whole with more robust statistics, the more subtle role of AGN across the population can be revealed. 

The results presented across Chapters~\ref{chap:morph}-\ref{chap:env} therefore reveal the interplay between all of the quenching mechanisms discussed in this thesis, including starvation and stripping, mergers \& AGN, disc instabilities \&  environment and minor mergers \& morphological quenching. All of these mechanisms are striving towards the same end goal of galaxy quiescence (with the occasional influx of gas thwarting their progress) but no single mechanism dominates over another, except in the most extreme environments or masses. While mass and morphological quenching will be far more likely to occur in the field, they still impact galaxies in the densest environments. Similarly, mergers will be far more likely to be a part of a galaxy's evolutionary history if it resides in dense environments and will often drown out the more subtle effects of slower quenching mechanisms which occurred before the merger. 

The dominance of each mechanism is therefore a matter of circumstance. Just as galaxy morphology is a spectrum of structure from the most disc dominated to the most spheroid dominated systems, so too are the quenching mechanisms which can cause this change. Mergers are a spectrum of mass ratios from micro mergers \citep{carlin16} through to major mergers, which have increasingly devastating impact upon the morphology and SFR of a galaxy. Morphological quenching mechanisms lie on a spectrum of mass and stellar mass surface density, whereas environmentally driven quenching mechanisms lie on a spectrum of increasing halo mass. All of these processes, depending on a galaxy's environment, are likely to affect a galaxy at some point in its lifetime, working in collaboration to reduce the SFR over a continuum of timescales. Rather than focusing on isolating the effects of a single  dominant mechanism, future galaxy evolution studies should attempt to understand this interplay of all possible quenching mechanisms over cosmic time. 

\section{The use of morphology in large surveys}\label{sec:usemorph}

As discussed in Section~\ref{sec:bigpic}, I consider morphology a continuous spectrum from disc dominated to bulge dominated systems which roughly reflects the continuous nature of galaxy evolution. This continuous nature is reflected by the parameters currently used to characterise the structure of a galaxy, including S\'ersic index \citep{sersic68}, Gini coefficient \citep{abraham03, lotz04}, asymmetry \citep{conselice00} and concentration index \citep{morgan58}. A problem arises however, when studies discretise these values by mapping them to the typical distinct Hubble classifications of morphology; either the data is mapped to T-types \citep{shimasaku01, brinchmann04, barro15} or merely split bimodally into late and early types, e.g. with S\'ersic index, $n \leq 2.5$ to identify discs \citep{ravindranath04, kelvin12, vika15}. Doing so reduces the complex internal structures which encode a galaxy's evolutionary history into two broad bins within which galaxies do not always share common features. This tendency to split populations bimodally is a common trope across the astrophysical community. Although this classification is physically motivated in some cases, such as Cephid variables and planet classification, in other instances this is not the case, e.g. Type 1 and 2 Seyferts, slow and fast rotators, long and short gamma ray bursts, supernova classifications and galaxy morphology, as I argue here. 

It is unsurprising that previous studies of galaxy evolution while split the galaxy population bimodally into early- and late-types galaxies have concluded there are two dominant evolutionary histories \citep[e.g.][]{schawinski14, casado15, belfiore16}. However, I have shown throughout this thesis that this is not the case, finding a continuous distribution of quenching rates across the galaxy population. I believe that this result is possible, in part due to the robust statistical method employed, but also due to the correct use of the full data set which is weighted by the continuous GZ vote fraction estimating the likelihood of either a disc or smooth galaxy. Ideally morphological parameters should be kept in their continuous forms to retain all the information about the galaxy's evolutionary history encoded in its structure \citep[e.g. see work by][]{peth16, savorgnan16, krywult16} For large surveys where a large amount of effort is placed in multi-component fitting \citep{haussler07, haussler11, haussler13, simard11, bruce14, vika15, johnston16} the use of morphology is particularly important as we must understand how to effectively utilise these fits when studying the dependence of morphology on a galaxy's quenching history. I believe that adapting such methods in future studies will allow the `high hanging fruit' science results to be picked out from both archival data and upcoming large surveys, such as LSST \citep{ivezic08}. 


\section{Future Work}\label{sec:future}

Due to the flexibility of the \starpy\ package I believe it will have a significant number of future applications. Firstly by investigating quenching using different wavebands as star formation indicators. For example, the $U-V$ and $V-J$ colours are used to separate star forming and quiescent galaxies on the UVJ diagram \citep{labbe05, wuyts07, williams09, brammer11, patel12} at higher redshift (out to $z\sim4$) in the COSMOS/UltraVISTA fields \citep[e.g. see work by][]{muzzin13}. Morphological classifications are also available for the COSMOS field with the recent release of the \textsc{gz:hubble} classifications in \cite{willett16}. Using \textsc{popstarpy} to investigate the SFHs inferred by these colours for galaxy populations weighted by morphology will help to further constrain the relative interplay of quenching mechanisms across the galaxy population with cosmic time. 

Secondly, \starpy\ could be adapted to consider many different possible SFHs for a galaxy and examine the Bayesian evidence to chose which is the most appropriate model to characterise the observed photometry of a galaxy. The current exponentially declining SFH (the so called ``$\tau$-model") used in \starpy\ is considered the simplest possible SFH one can assume and so more detail about the effects of different quenching mechanisms may be elucidated by increasing the complexity. For example, possible SFHs include a starburst model \citep{kauffmann03}, an extentded $\tau$-model \citep{simha14}, a Gaussian model \citep{feuillet16} or a log-normal SFR \citep{gladders13, abramson16}. 

Along with this expansion of the \starpy\ module itself, several avenues of data exploration are also still available using \starpy\ (including use with IFU data, see Section~\ref{sec:IFU}):
\begin{enumerate}[(i)]

\item A comparison of a large population of fast and slow rotators would aid in the understanding of the different quenching timescales of and interplay between wet and dry major mergers. 

\item A study of barred vs non-barred galaxies using $\{p_{\rm{bar}}, p_{\rm{no bar}}\}$ in place of $\{p_{\rm{disc}}, p_{\rm{smooth}}\}$ to weight the population densities derived with \textsc{popstarpy} may reveal the impact a bar can have on a galaxy's SFR by funnelling gas to central regions.

\item Studying the SFHs of low mass satellite galaxies with $M_* \leq 10^{8-9} ~M_{\odot}$, which are thought to have quenching histories dominated by ram pressure stripping \citep{hester06, fillingham16}, may help to constrain the quenching timescales for this mechanism. 

\item The effect of AGN feedback could be studied further by investigating the SFHs of unobscured Type 1 AGN (however this would require either a more accurate subtraction of the unobscured nuclear emission or a change in the bandpass input to \starpy\ to negate this issue) and those AGN identified by X-ray, radio and IR selection methods.  The results of \citep{ellison16} show that only radio and optically selected AGN have SFRs distributed below the SFS, whereas IR selected AGN have SFRs consistent with the SFS. This suggests that different selection methods may be biased towards either star forming or quenched galaxies. Reproducing the result seen in Chapter~\ref{sec:agnfeedback} with these AGN would corroborate the idea that quenching is actually occurring across the entire AGN population, and provide further support for the theory of AGN unification \citep{antonucci93, urry95}.

\end{enumerate}

\section{The use of \starpy\ with IFU data}\label{sec:IFU}

\begin{figure}
\centering{
\includegraphics[width=0.95\textwidth]{discussion/manga_data.png}}
\caption[Example MaNGA fibre bundle on a target galaxy with example emission data]{Example fibre bundle placed over a MaNGA target galaxy (left) and corresponding preliminary survey data showing the mapped velocity (middle) and flux (right) of $H\alpha$ gas emission as measured across the galaxy in each fibre. Such measurements can be utilised to calculate the star formation rate across the structure of a galaxy. Adapted from \cite{bundy15} Figure 14.}
\label{fig:manga}
\end{figure}

In Chapter \ref{chap:agn} I discussed how the current SDSS data (both photometric and spectroscopic) cannot determine whether the AGN is the cause or the consequence of the quenching seen across the AGN host population in Figures \ref{time} \& \ref{rate}. Using data from the MaNGA IFU survey \citep{bundy15} I hope to determine whether feedback from the AGN is truly the cause of this quenching. The MaNGA survey will provide 127 spectra across a single galaxy (see Figure~\ref{fig:manga}), allowing the SFH to be mapped as a function of radius. The acquisition of a spectrum in each of these apertures allows for the modification of \starpy\ to take spectral star formation indicators \citep[such as $H\alpha$][]{kennicutt94} to break the degeneracy provided by the photometric colours (see Figure~\ref{pred}), and also the easy removal of any unobscured AGN in the central regions. 

Any correlation of the inferred quenching parameters with radius of the galaxy will allow the determination of whether quenching is happening from the outside-in \citep[i.e. due to the environment, as in][]{pan15, clarke16, schaefer17} or inside-out, as in work with preliminary MaNGA survey data by \citet{belfiore16} and with CALIFA data by \citet{gonzalez16}. I will investigate how this preference for outside-in or inside-out quenching is correlated with the presence of an AGN and a galaxy's environment. This will no only help to answer the question of \emph{cause} vs. \emph{consequence} but also further constrain the timescales of quenching mechanisms possibly caused by AGN or environment. 

\section{Hubble Space Telescope follow up}\label{sec:hst}

In Chapter~\ref{chap:agn}, I discussed how accurate fits to the bulge-to-total ratio could not be made for galaxies of the \textsc{bulgeless} sample hosting AGN due to the resolution of ground based SDSS imaging. This led to the derivation of upper limits on the bulge masses of this sample. The Hubble Space Telescope (HST) however, provides (i) an extremely stable and well understood point spread function, enabling reliable separation of the AGN from the host galaxy, and (ii) high spatial resolution which is needed to distinguish between classical, merger driven bulges and pseudo bulges grown by secular processes. Observations with the HST of the galaxies in the \textsc{bulgeless} sample will enable extremely robust measures of bulge-to-total ratios for each host galaxy and allow the identification of truly secular systems with growing black holes. Such measurements will also allow the interplay between merger driven (building classical bulges) and merger free (growing pseudo-bulges or retaining a pure disc) co-evolutionary histories to be disentangled. 

\begin{figure*}
\centering{
\includegraphics[width=0.49\textwidth]{discussion/sdss_data.png}
\includegraphics[width=0.49\textwidth]{discussion/hst_data.jpg}}
\caption[Example HST image data in comparison to SDSS]{Example SDSS urgiz image (left) for J$192250.74$-$055259.15$, part of the \textsc{bulgeless} sample, described in Section~\ref{sec:selectAGN}, in comparison to space based imaging from the HST (right). The higher resolution HST image reveals finer structure than the SDSS image, including spiral arms, a ring and a possible bar along with the true disc dominated nature of the galaxy.}
\label{fig:hstdata}
\end{figure*}

Figure~\ref{fig:hstdata} demonstrates the difference in resolution provided by the HST with some of the first data received from our successful proposal (ID: 14606, Cycle 24) to observe galaxies in the \textsc{bulgeless} sample. The galaxy shown is J$192250.74$-$055259.15$ in comparison with the ground based SDSS ugriz image (left; also shown in Figure~\ref{fig:INTimages}). This galaxy has an estimated black hole mass of $M_{\rm{BH}} = 10^{7.4\pm0.1}~M_{\odot}$ but a bulge-to-total ratio could not be derived from the SDSS image. The HST image will allow for an accurate derivation of the bulge mass of this galaxy, allowing for a more concrete conclusion to be drawn on the controversial issue of secularly driven galaxy-black hole co-evolution. 


%\chapter{Conlusions}

Quenching is morphologically dependant. 

AGN may be responsible for some of this quenching.

The environment plays less of a role than typical mass quenching. 


%next line adds the Bibliography to the contents page
\addcontentsline{toc}{chapter}{Bibliography}
%uncomment next line to change bibliography name to references
%\renewcommand{\bibname}{References}
\bibliographystyle{mn2e}
\bibliography{refs}        %use a bibtex bibliography file refs.bib

\newpage
\vspace*{3cm}

\begin{center}

\emph{Mischief Managed.}
\end{center}


\end{document}


