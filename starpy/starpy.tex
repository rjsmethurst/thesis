\chapter{STARPY: A Bayesian analysis of a galaxy's SFH}

\emph{The work in the following chapter has been published in \citet{smethurst15}.}
\\

\section{Star Formation History Models}\label{qmod}

 \begin{figure}
\centering{
\includegraphics[height=0.75\textheight]{starpy/colours.pdf}}
\caption[Predicted colours and SFRs of quenching models]{Quenching timescale $\tau$ versus quenching onset time $t$ in all three panels for the quenched SFH models used in ~\starpy. Colour shadings show model predictions of the $u-r$ optical colour (top panel), $NUV-u$ colour (middle panel), and star formation rate in $M_\odot \rm{~yr}^{-1}$ (lower panel), at $t^{obs} = 12.8~\rm{Gyr}$, the mean observed redshift of the GZ2 sample (see Section \ref{qmod}). The combination of optical and NUV colours is a sensitive measure of the $\theta = [t_q, \tau]$ parameter space. Note that all models with $t > 12.8$ \rm{Gyr} are effectively un-quenched. The `kink' in the bottom panel is due to the assumption that the sSFR is constant prior to $t \sim 3~\rm{Gyr}$ ($z\sim 2.2$).}
\label{pred}
\end{figure}

The quenched star formation history (SFH) of a galaxy can be simply modelled as an exponentially declining star formation rate (SFR) across cosmic time ($0 \leq t ~\rm{[Gyr]} \leq 13.8$) as:
\begin{equation}\label{sfh}
SFR =
\begin{cases}
i_{sfr}(t_q) & \text{if } t < t_q \\
i_{sfr}(t_q) \times exp{\left( \frac{-(t-t_{q})}{\tau}\right)} & \text{if } t > t_q 
\end{cases}
\end{equation}
where $t_{q}$ is the onset time of quenching, $\tau$ is the timescale over which the quenching occurs and $i_{sfr}$ is an initial constant star formation rate dependent on $t_q$.  A smaller $\tau$ value corresponds to a rapid quench, whereas a larger $\tau$ value corresponds to a slower quench. 

We assume that all galaxies formed at a time $t=0~\rm{Gyr}$ with an initial burst of star formation. The mass of this initial burst is controlled by the value of the $i_{sfr}$ which is set as the average specific SFR (sSFR) at the time of quenching $t_q$.  \citet{peng10} defined a relation (their equation 1) between the average sSFR and redshift (cosmic time, $t$) by fitting to measurements of the mean sSFR of blue star forming galaxies from SDSS, zCOSMOS and literature values at increasing redshifts \citep{Elbaz07, Daddi07}:
\begin{equation}
sSFR(m,t) = 2.5 \left( \frac{m}{10^{10} M_{\odot}} \right)^{-0.1} \left(\frac{t}{3.5 ~\rm{Gyr}}\right)^{-2.2} \rm{Gyr}^{-1}.
\end{equation}
Beyond $z \sim 2$ the characteristic SFR flattens and is roughly constant back to $z\sim6$. The cause for this change is not well understood but can be seen across similar observational data \citep{peng10, gonzalez10, bethermin12}. Motivated by these observations, the relation defined in \citet{peng10} is taken up to a cosmic time of $t=3~\rm{Gyr}~(z \sim 2.3)$ and prior to this a constant average SFR is assumed (see Figure~\ref{sfr_mass_col}). At the point of quenching, $t_{q}$, the models are defined to have a SFR which lies on this relationship for the sSFR, for a galaxy with mass, $m = 10^{10.27} M_{\odot}$ (the mean mass of the GZ2 sample; see Section~\ref{results} and Figure~\ref{sfr_mass_col}).

\begin{figure*}
\centering{
\includegraphics[width=\textwidth]{starpy/sfr_mass_colour_diagram.pdf}}
\caption[SFH models in observational planes]{Left panel: SFR vs. $M_*$ for all 126,316 galaxies in our full sample (shaded contours), with model galaxy trajectories shown as coloured points/lines with each point representing a time step of $0.5~\rm{Gyr}$. The SFHs of the models are shown in the middle panel, where the SFR is initially constant before quenching at time $t$ and thereafter exponentially declining with a characteristic timescale $\tau$. We set the SFR at the point of quenching to be consistent with the typical SFR of a star-forming galaxy at the quenching time, $t$ (dashed line; \citealt{peng10}). The full range of models reproduces the observed colour-colour properties of the sample (right panel); for clarity the figures show only 4 of the possible models explored in this study. Note that some of the model tracks produce colours redder than the apparent peak of the red sequence in the GZ2 subsample; however this is not the \emph{true} peak of the red sequence due to the necessity for NUV colours from GALEX.}
\label{sfr_mass_col}
\end{figure*}
  
Under these assumptions the average SFR of our models will result in a lower value than the relation defined in \citet{peng10} at all cosmic times; each galaxy only resides on the `main sequence' at the point of quenching. However galaxies cannot remain on the `main sequence' from early to late times throughout their entire lifetimes given the unphysical stellar masses and SFRs this would result in at the current epoch in the local Universe \citep{bethermin12, Heinis14}. If we were to include prescriptions for no quenching, starbursts, mergers, AGN etc. into our models we would improve on our reproduction of the average SFR across cosmic time; however we chose to initially focus on the simplest model possible.

Once this evolutionary SFR is obtained, it is convolved with the \citet{BC03} population synthesis models to generate a model SED at each time step. The observed features of galaxy spectra can be modelled using simple stellar population techniques which sum the contributions of individual, coeval, equal-metallicity stars. The accuracy of these predictions depends on the completeness of the input stellar physics. Comprehensive knowledge is therefore required of (i) stellar evolutionary tracks and (ii) the initial mass function (IMF) to synthesise a stellar population accurately. 

These stellar population synthesis (SPS) models are an extremely well explored (and often debated) area of astrophysics \citep{Maraston05, Eminian08, CGW09, falkenberg09, Chen10, Kriek10, miner11, melbourne12}. In this investigation we chose to utilise the \citet{BC03} \emph{GALEXEV} SPS models, to allow a direct comparison with S14, along with a Chabrier \citep{chabrier03} IMF, across a large wavelength range ($0.0091 < ~\lambda~\rm{[\mu m]}~ < 160 $) with solar metallically (m62 in the \citet{BC03} models; hereafter BC03).


Fluxes from stars younger than $3~$Myr in the SPS model are suppressed to mimic the large optical depth of protostars embedded in dusty formation cloud (as in S14), then filter transmission curves are applied to the fluxes to obtain AB magnitudes and therefore colours.  For a particular galaxy at an observed redshift, $z$, we define the observed time, $t^{obs}$ for that galaxy using the standard cosmological conversion between redshift and time. We utilise the SFH models at this observed time for each individual galaxy to compare the predicted model and observed colours directly.


Figure~\ref{pred} shows these predicted optical and NUV colours at a time of $t^{obs} = 12.8 ~\rm{Gyr}$ (the average observed time of the Galaxy Zoo 2 sample, $z \sim 0.076$) provided by the exponential SFH model. These predicted colours will be referred to as $d_{c,p}(t_{q}, \tau, t^{obs})$, where c=\{opt,NUV\} and p = predicted. The SFR at a time of $t^{obs}=12.8~\rm{Gyr}$ is also shown in Figure~\ref{pred} to compare how this correlates with the predicted colours. The $u-r$ predicted colour shows an immediate correlation with the SFR, however the $NUV-u$ colour is more sensitive to the value of $\tau$ and so is ideal for tracing any recent star formation in a population . At small $\tau$ (rapid quenching timescales) the $NUV-u$ colour is insensitive to $t_{q}$, whereas at large $\tau$ (slow quenching timescales) the colour is very sensitive to $t_{q}$. Together the two colours are ideal for tracing the effects of $t_{q}$ and $\tau$ in a population. 

We stress here that this model is not a fully hydrodynamical simulation, it is a simple model built in order to test the understanding of the evolution of galaxy populations. These models are therefore not expected to accurately determine the SFH of every galaxy in the GZ2 sample, in particular galaxies which have not undergone any quenching. In this case the models described above can only attribute a constant star formation rate to these  unquenched galaxies. In reality, there are many possible forms of SFH that a galaxy can take, a few of which have been investigated in previous literature; starbursts \citep{Canalizo01}, a power law \citep{Glazebrook03}, single stellar populations \citep{Trager00, Sanchez06, Vazdekis10} and metallicity enrichment \citep{deLucia14}. Incorporating these different SFHs along with prescriptions for mergers and a reinvigoration of star formation post quench into our models is a possible future extension to this work once the results of this initial study are well enough understood to permit additional complexity to be added.

\section{Probabilistic Fitting Methods}\label{stats}

In order to achieve robust conclusions we conduct a Bayesian analysis \citep{Sivia, mackay03} of our SFH models in comparison to the observed GZ2 sample data. This approach requires consideration of all possible combinations of $\theta \equiv (t_{q}, \tau)$. Assuming that all galaxies formed at $t=0~\rm{Gyr}$ with an initial burst of star formation, we can assume that the `age' of each galaxy in the GZ2 sample is equivalent to an observed time, $t^{obs}_{k}$. We then use this  `age' to calculate the predicted model colours at this cosmic time for a given combination of $\theta$: $d_{c,p}(\theta_k, t^{obs}_{k})$ for both optical and NUV $(c={opt,NUV})$ colours. We can now directly compare our model colours with the observed GZ2 galaxy colours, so that for a single galaxy $k$ with optical ($u-r$) colour, $d_{opt, k}$ and NUV ($NUV-u$) colour, $d_{NUV,k}$, the likelihood $P(d_{k}|\theta_k, t^{obs}_{k})$ is:


\begin{equation}\label{like}
P(d_{k}|\theta_k, t^{obs}_{k}) = \frac{1}{\sqrt{2\pi\sigma_{opt, k}^2}}\frac{1}{\sqrt{2\pi\sigma_{NUV, k}^2}} \\ \exp{\left[ - \frac{(d_{opt, k} - d_{opt, p}(\theta_k, t_{k}^{obs}))^2}{\sigma_{opt, k}^2} \right]} \\ \exp{\left[ - \frac{(d_{NUV, k} - d_{NUV, p}(\theta_k, t_{k}^{obs}))^2}{\sigma_{NUV, k}^2} \right]}.
\end{equation}


We have assumed that $P(d_{opt}|\theta_k, t^{obs}_{k})$ and $P(d_{NUV}|\theta_k, t^{obs}_{k})$ are independent of each other and that the errors on the observed colours are also independent. To obtain the probability of each combination of $\theta$ values \underline{given} the GZ2 data: $P(\theta_k|d_k, t^{obs})$, i.e. how likely is a single SFH model given the observed colours of a single GZ2 galaxy, we utilise Bayes' theorem:
 \begin{equation}\label{big}
P(\theta_k|d_k, t^{obs}) = \frac{P(d_k|\theta_k, t^{obs})P(\theta_k)}{\int P(d_k |\theta_k, t^{obs})P(\theta_k) d\theta_k}.
\end{equation}
We assume a flat prior on the model parameters so that:
\begin{equation}\label{prior}
P(\theta_k) =
\begin{cases}
1 & \text{if } 0 \leq t_q ~\rm{[Gyr]}~ \leq 13.8 ~  \text{ and } ~ 0 \leq \tau  ~\rm{[Gyr]}~ \leq 4\\
0 & \text{otherwise.} \\
\end{cases}
\end{equation}

\begin{figure}
\centering{
\includegraphics[width=0.9\textwidth]
{starpy/triangle_t_tau_red_s_1237655504035185152_40000_14_16_06_08_14.pdf}
\caption[Example \starpy ~output]{Example output from \starpy ~for a galaxy within the red sequence. The contours show the positions of the `walkers' in the Markov Chain (which are analogous to the areas of high probability) for the quenching models described by $\theta = [t_q, \tau]$ and the histograms show the 1D projection along each axis. Solid (dashed) blue lines show the best fit model (with $\pm 1\sigma$) to the galaxy data. The postage stamp image from SDSS is shown in the top right along with the debiased vote fractions for smooth ($p_s$) and disc ($p_d$) from Galaxy Zoo 2.} }
\label{one_example}
\end{figure}

As the denominator of Equation~\ref{big} is a normalisation factor, comparison between likelihoods for two different SFH models (i.e., two different combinations of $\theta_k = [t_q, \tau]$) is equivalent to a comparison of the numerators. Calculation of $P(\theta_k|d_k, t^{obs})$  for any $\theta$ is possible given galaxy data from the GZ2 sample. Markov Chain Monte Carlo (MCMC; \citealt{mackay03, emcee13, GW10}) provides a robust comparison of the likelihoods between $\theta$ values; here we choose \emph{emcee},\footnote{\url{emcee13.iel.fm/emcee/}} a Python implementation of an affine invariant ensemble sampler by \cite{emcee13}.

This method allows for a more efficient exploration of the parameter space by avoiding those areas with low likelihood. A large number of `walkers' are started at an initial position where the likelihood is calculated; from there they individually `jump' to a new area of parameter space. If the likelihood in this new area is greater (less) than the original position then the `walkers' accept (reject) this change in position. Any new position then influences the direction of the  `jumps' of other walkers. This is repeated for the defined number of steps after an initial `burn-in' phase. \emph{emcee} returns the positions of these `walkers', which are analogous to the regions of high probability in the model parameter space. The model outlined above has been coded using the \emph{Python} programming language into a package named \starpy ~which has been made freely available to download\footnote{\url{github.com/zooniverse/starpy}}. An example output from this Python package for a single galaxy from the GZ2 sample in the red sequence is shown in Figure~\ref{one_example}. The contours show the positions of the `walkers' in the Markov Chain which are analogous to the areas of high probability.

\begin{figure}
\centering{
\includegraphics[width=\textwidth]{starpy/mosaic_test.pdf}}
\caption[Testing \starpy]{Results from \starpy ~for an array of synthesised galaxies with known, i.e. \underline{true}, $t_q$ and $\tau$ values (marked by the red lines) using the complete function to calculate the predicted colour of a proposed set of $\theta$ values in each MCMC iteration, assuming an error on the calculated known colours of $\sigma_{u-r} = 0.124$ and $\sigma_{NUV-u} = 0.215$, the average errors on the GZ sample colours. In each case \starpy ~succeeds in locating the true parameter values within the degeneracies of the star formation history model.}
\label{test_mosaic}
\end{figure}

In order to test that \starpy ~can find the correct quenching model for a given observed colour, 25 synthesised galaxies were created with known SFHs (i.e. known values of $\theta$) from which optical and NUV colours were generated. These were input into \starpy ~ to ensure that the known values of $\theta$ were reproduced, within error, for each of the 25 synthesised galaxies. Figure~\ref{test_mosaic} shows the results for each of these 25 synthesised galaxies, with the known values of $\theta$ shown by the red lines. In some cases this red line does not coincide with the peak of the distribution shown in the histograms for one parameter, however in all cases the intersection of the red lines is within the sample contours. 

We find peaks in the histograms across all areas of the parameter space in both dimensions of $[t, \tau]$, ensuring that any conclusions drawn from these distributions are due to a superposition of extended probability distributions, as opposed to a bimodal distribution of probability distributions across all galaxies.


\begin{figure*}
\centering{
\includegraphics[width=0.49\textwidth]{starpy/corner_test_starfpy_full_sfh_function_0.pdf}
\includegraphics[width=0.49\textwidth]{starpy/corner_test_starfpy_lookup_0.pdf}}
\caption[Comparing complete and look-up table versions of \starpy]{Left panel: Results from \starpy ~for \underline{true} $t_q$ and $\tau$ values (red lines) using the complete function to calculate the predicted colour of a proposed set of $\theta$ values in each MCMC iteration. The median walker position (the 50th percentile of the Bayesian probability distribution) is shown by the solid blue line with the dashed lines encompassing $68\% (\pm 1\sigma)$ of the samples (the 16th and 84th percentile positions). The time taken to run for a single galaxy using this method is approximately 2 hours. Right panel: Results from \starpy ~for \underline{true} $t_q$ and $\tau$ values using a look up table generated from the complete function to calculate the predicted colour of a proposed set of $\theta$ values in each MCMC iteration. The time taken to run for a single galaxy using this method is approximately 2 minutes.}
\label{lookup}
\end{figure*}

\begin{table}
\centering{
\caption{Median walker positions (the 50th percentile; as shown by the blue solid lines in Figure~\ref{lookup}) found by \starpy ~ for a single galaxy, using the complete star formation history function and a look up table to speed up the run time. The errors quoted define the region in which $68\%$ of the samples are located, shown by the dashed blue lines in Figure~\ref{lookup}. The known true values are also quoted, as shown by the red lines in Figure~\ref{lookup}. All values are quoted to three significant figures.}
\begin{tabular*}{0.65\textwidth}{r @{\extracolsep{\fill}}ccc}
\multicolumn{1}{l}{} & \multicolumn{3}{c}{}                                          \\ \hline
                     & $t_q$                       & $\tau$                       &  \\ \hline
True                 & $4.37$                        & $2.12$                         &  \\
Complete             & $3.893 \pm^{3.014}_{2.622}$ & $2.215 \pm^{0.395}_{0.652}$ &  \\
Look up table        & $3.850 \pm^{2.988}_{2.619}$ & $2.218 \pm^{0.399}_{0.649}$ & \\ \hline
\end{tabular*}}
\label{median_lu}
\end{table}

Considering the size of the sample in this investigation of $126,316$ galaxies total, a three dimensional look up table (in observed time, quenching time and quenching rate) was generated using the star formation history function in \starpy ~to speed up the run time. 

We wish to consider the model parameters for the populations of galaxies across the colour magnitude diagram for both smooth and disc galaxies, therefore we run the \starpy ~package on each galaxy in the GZ2 sample. This was extremely time consuming; for each combination of $\theta$ values which \emph{emcee} proposes, a new SFH must be built, prior to convolving it with the BC03 SPS models at the observed age and then predicted colours calculated from the resultant SED. For a single galaxy this takes up to 2 hours on a typical desktop machine for long Markov Chains. A look-up table was therefore generated at $50 ~t^{obs}$, for $100 ~t_{quench}$ and $100 ~\tau$ values; this was then interpolated over for a given observed galaxy's age and proposed $\theta$ values at each step in the Markov Chain. This ensures that a single galaxy takes approximately 2 minutes to run on a typical desktop machine. Figure~\ref{lookup} shows an example of how using the look up table in place of the full function does not affect the results to a significant level. Table~\ref{median_lu} quotes the median walker positions (the 50th percentile of the Bayesian probability distribution) along with their $\pm 1\sigma$ ranges for both methods in comparison to the true values specified to test \starpy. The uncertainties incorporated into the quoted values by using the look up table are therefore minimal with a maximum $\Delta = 0.043$.

Using this lookup table, each of the $126,316$ total galaxies in the GZ2 sample was run through \starpy ~on multiple cores of a computer cluster to obtain the Markov Chain positions (analogous to $P(\theta_k|d_k)$) for each galaxy, $k$ (see Figure~\ref{one_example}). In each case the Markov Chain consisted of $100$ `walkers' which took $400$ steps in the `burn-in' phase and $400$ steps thereafter, at which point the MCMC acceptance fraction was checked to be within the range $0.25 < f_{acc} < 0.5$ (which was true in all cases). Due to the Bayesian nature of this method, a statistical test on the results is not possible; the output is probabilistic in nature across the entirety of the parameter space.

These individual galaxy positions are then combined to visualise the areas of high probability in the model parameter space across a given population (e.g. the green valley). 

\begin{table*}
\centering{
\caption{Number of galaxies in each population which had walker positions discarded due to low probability in order to exclude those galaxies from the analysis which were poorly fit by this quenching model.}
\begin{tabular*}{0.95\textwidth}{p{4cm} @{\extracolsep{\fill}} ccc}
                                          & \textbf{Red Sequence}                                   & \textbf{Green Valley}                                  & \textbf{Blue Cloud}                                      \\ \hline
All walkers discarded                     & \begin{tabular}[c]{@{}c@{}}1420\\ (7.00\%)\end{tabular} & \begin{tabular}[c]{@{}c@{}}437\\ (2.41\%)\end{tabular} & \begin{tabular}[c]{@{}c@{}}3109\\ (5.37\%)\end{tabular}  \\
More than half walker positions discarded & \begin{tabular}[c]{@{}c@{}}2010\\ (9.92\%)\end{tabular} & \begin{tabular}[c]{@{}c@{}}779\\ (4.30\%)\end{tabular} & \begin{tabular}[c]{@{}c@{}}6669\\ (11.52\%)\end{tabular} \\ \hline
\end{tabular*}}
\label{discardnum}
\end{table*}

We discard walker positions returned by \starpy~ with a corresponding probability of $P(\theta_k|d_k) < 0.2$ in order to exclude galaxies which are not well fit by the quenching model; for example blue cloud galaxies which are still star forming will be poorly fit by a quenching model (see Section~\ref{qmod}). This raises the issue of whether we exclude a significant fraction of our galaxy sample and whether those galaxies reside in a specific location of the colour-magnitude. The fraction of galaxies which had all or more than half of their walker positions discarded due to low probability are shown in Table \ref{discardnum}. Using this constraint, $2.4\%$, $7.0\%$ and $5.4\%$ of green, red and blue galaxies respectively had \emph{all} of their walker positions discarded. 

This is not a significant fraction of either population, therefore this shows that the \starpy~ module is effective in fitting the majority of galaxies and that this method of discarding walker positions ensures that poorly fit galaxies are removed from the analysis of the results. Figure \ref{discarded} shows that these galaxies with discarded walker positions are also scattered across the optical-NUV colour-colour diagram and therefore \starpy ~is also effective in fitting galaxies across this entire plane. 

\begin{figure}
\includegraphics[width=0.9\textwidth]{starpy/discarded_galaxy_colour_colour.pdf}
\caption[Colours of discarded galaxies]{Contours show the full GZ2 subsample optical-NUV colour-colour diagram. The points show the positions of the galaxies which had all (top panels) or more than half (bottom panel) of their walker positions discarded due to their low probability for the red sequence (left), green valley (middle) and blue cloud (right).}
\label{discarded}
\end{figure}



The Markov Chain positions are then binned and weighted by their corresponding logarithmic posterior probability $\log [P(\theta_k|d_k)]$, provided by the \emph{emcee} package, to further emphasise the features and differences between each population in the visualisation. The GZ2 data also provides uniquely powerful continuous measurements of a galaxy's morphology, therefore we utilise the user vote fractions to obtain separate model parameter distributions for both smooth and disc galaxies. This is obtained by also weighting by the morphology vote fraction when the binned positions are summed. We stress that this portion of the methodology is a non-Bayesian visualisation of the combined individual galaxy results for each population.

For example, the galaxy shown in Figure~\ref{one_example} would contribute almost evenly to both the smooth and disc parameters due to the GZ2 vote fractions. Since galaxies with similar vote fractions contain both a bulge and disc component, this method is effective in incorporating intermediate galaxies which are thought to be crucial to the morphological changes between early- and late-type galaxies. It was the consideration of these intermediate galaxies which was excluded from the investigation by S14.

\subsection{Alternative Hierarchical Bayesian approach}\label{althyper}

The approach presented above relies upon a visualisation of the SFHs across each population, with no inference involved beyond the use of \textsc{starpy} to derive the individual galaxy SFHs. The preferred approach to this problem would be to use a hierarchical Bayesian method to determine the `hyper-parameters' that describe the distribution of the parent population $\theta' = [t_q', \tau']$ that each individual galaxy's SFH is drawn from. 

We want the posterior PDF for $\vec{\theta}'$ to describe a galaxy population:
\begin{equation}\label{hyper}
P(\vec{\theta}'|\vec{d}) = \frac{P(\vec{d}|\vec{\theta}')P(\vec{\theta}')}{P(\vec{d})}, 
\end{equation}
where $\vec{d}$ represents all of the optical and NUV colour data in a population $\{\vec{d}_k\}$. For one galaxy, $k$, the marginalised likelihood is:
\begin{equation}\label{one}
P(d_k|\vec{\theta}') = \iint \! P(d_k|t_k, \tau_k) P(t_k, \tau_k|\vec{\theta}') \ \mathrm{d}t_k ~ \mathrm{d}\tau_k
\end{equation}
and for all galaxies, $N$, therefore: 
\begin{equation}
P(\vec{d}|\vec{\theta}') = \prod_k^N P(d_k|\vec{\theta}').
\end{equation}

Using \textsc{starpy}~ for individual galaxies we sample from the `interim' posterior $P(t_k, \tau_k|d_k)$ which we can relate to $P(d_k|t_k, \tau_k)$  so that:

\begin{equation}\label{marg}
P(d_k|\vec{\theta}') = \iint  \! P(t_k, \tau_k|d_k) . P(d_k) . \frac{P(t_k, \tau_k|\vec{\theta}')}{P(t_k, \tau_k)} \ \mathrm{d}t_k ~ \mathrm{d}\tau_k.
\end{equation}
In order to calculate this we draw $N_s$ random samples, $r$, from the interim posterior, $P(t_k, \tau_k|d_k)$ so that Equation \ref{marg} can be expressed as a sum over a number of random samples, $N_s$ (as with the calculation of an expected mean):
\begin{equation}\label{imp}
P(d_k|\vec{\theta}') = \frac{P(d_k)}{N_s} \sum_r^{N_s} \frac{P(t_{k,r}, \tau_{k,r}|\vec{\theta}')}{P(t_k, \tau_k)},
\end{equation}
for the $r^{th}$ sample of $N_s$ total samples taken from one galaxy's, $k$,  interim posterior PDF. This fraction is known as the `importance weight', $w_r$, in importance sampling. 

However, we also have two morphological vote fractions that we can weight by to determine separate hyper-parameters, $\vec{\theta}' = [\vec{\theta}'_d, \vec{\theta}'_s]$, for both disc, $d$, and smooth, $s$, galaxies. Therefore:

\begin{equation}\label{morphimp}
w_r = \frac{P(t_{k,r}, \tau_{k,r}|\vec{\theta}')}{P(t_k, \tau_k)} =  \frac{p_{d,k} P(t_{k,r}, \tau_{k,r}|\vec{\theta}'_d) + p_{s,k} P(t_{k,r}, \tau_{k,r}|\vec{\theta}'_s)}{P(t_k, \tau_k)}
\end{equation} 

If we substitute equation \ref{imp} into equation \ref{hyper} we find that the $P(d_k)$ terms cancel and we are left with:
\begin{equation}
P(\vec{\theta}'|\vec{d}) = P(\vec{\theta}')~\prod_k^N \frac{1}{N_{s,k}} \sum_r^{N_s} w_r ,
\end{equation}
where $P(\vec{\theta}')$ is the assumed prior on the hyper-parameters, which is assumed to be uniform.

This approach is heavily dependent on what shape is assumed for the hyper-distribution; a decision which is not trivial. It is often common for this function to take the form of a multi-component Gaussian mixture model \citep{mackay03, lahav00}. For example a two component Gaussian mixture model in $[t, \tau]$ space is described by eight hyper-parameters for a single morphology, $\vec{\theta}' = [\mu_{t,1}, \sigma_{t,1}, \mu_{\tau,1}, \sigma_{\tau,1}, \mu_{t,2}, \sigma_{t,2}, \mu_{\tau,2}, \sigma_{\tau,2}]$. Here we also assume no covariance between hyper-parameters for simplicity. 

We used this assumption of a two component Gaussian mixture model, to infer the population parameters for both the \textsc{agn-host} and \textsc{inactive} populations and the results are shown in Figure~\ref{method3}. These results were produced by drawing $N_s = 100$ random samples from each galaxy, $k$, in each mass bin. We plot the distributions for a given morphology by taking the median value of the posterior distribution for each of the 8 parameters describing the two component Gaussian mixture. We can see in Figure~\ref{method3} that this hierarchical method produces similar distributions for the \textsc{agn-host} and \textsc{inactive} samples. This finding is not expected given the differences between the two samples in colour-colour space seen in Figure~\ref{colcol}. 

%\begin{figure}
%\includegraphics[width=0.48\textwidth]{figc1a.pdf}
%\includegraphics[width=0.48\textwidth]{figc1b.pdf}
%\caption[8pt]{Hierarchical-posterior PDF of the quenching time ($t_q'$, top) and rate ($\tau'$, bottom) population parameters, normalised so that the areas under the curves are equal. \textsc{agn-host} (left) and \textsc{inactive} (right) galaxies are split into low (top), medium (middle) and high (bottom) mass, weighted for smooth (red dashed) and disc (blue solid) galaxies. A low (high) value of $t_q'$ corresponds to the early (recent) Universe. A small (large) value of $\tau'$ corresponds to a rapid (slow) quench.}
%\label{method3}
%\end{figure}

In order to test whether this assumption of a multi-component Gaussian mixture model is appropriate, we sampled the inferred hierarchical distributions to produce replica datasets in optical-NUV colour space. These are shown here in Figure~\ref{replica}  in comparison to the observed colour-colour distributions of the \textsc{agn-host} and \textsc{inactive} samples. For all masses and morphologies the replicated $u-r$ and $NUV-u$ colours do not accurately match the observed data. 

%\begin{figure}
%\begin{centering}
%\includegraphics[width=0.42\textwidth]{figc2a.pdf}
%\includegraphics[width=0.42\textwidth]{figc2b.pdf}
%\caption[8pt]{Optical-NUV colour-colour diagrams for the \textsc{agn-host} (top) and \textsc{inactive} (bottom) galaxies shown by the black contours, split into low mass (top), medium mass (middle) and high mass (bottom) galaxies weighted by $p_d$ (left) and $p_s$ (right). Kernel smoothing has been applied to the overlaid replica datasets, which are created by sampling from the \textbf{inferred 2 component Gaussian mixture model hierarchical parent distributions}, shown here in Figure~\ref{method3}. Gaussian random noise is also added to the inferred colours, with a mean and standard deviation of the errors on the observed colours of the respective sample. Contours are shown for samples taken from the disc (blue) and smooth weighted (red) inferred hierarchical distributions.
%}
%\label{replica}
%\end{centering}
%\end{figure}
%
%\begin{figure}
%\begin{centering}
%\includegraphics[width=0.42\textwidth]{figc3a.pdf}
%\includegraphics[width=0.42\textwidth]{figc3b.pdf}
%\caption[8pt]{Optical-NUV colour-colour diagrams for the \textsc{agn-host} (top) and \textsc{inactive} (bottom) galaxies shown by the black contours, split into low mass (top), medium mass (middle) and high mass (bottom) galaxies weighted by $p_d$ (left) and $p_s$ (right). Kernel smoothing has been applied to the overlaid replica datasets, which are created by sampling from the \textbf{unweighted summed visualisation distributions} presented in Figures~\ref{time} and \ref{rate}. Gaussian random noise is also added to the inferred colours, with a mean and standard deviation of the errors on the observed colours of the respective sample. Contours are shown for samples taken from the disc (blue) and smooth weighted (red) summed visualisation distributions.
%}
%\label{replicapop}
%\end{centering}
%\end{figure}

We also varied the value of $N_s$ and found that increasing the number of samples drawn did not improve this fit for either the \textsc{agn-host} or \textsc{inactive} populations. Similarly increasing the number of components in the Gaussian mixture model did not immediately improve the accuracy of the fit.  We therefore concluded that this functional form of the population distribution was unsatisfactory. %An extensive exploration of a wide variety of functional forms is necessary to ensure the correct conclusions are drawn from the data. Such an investigation is beyond the scope of this paper. 

The approach described in section \ref{stats} and applied throughout the rest of this work was motivated by the investigation increasing the number of samples, $N_s$ drawn from the posterior of each galaxy, k, until the point where all the samples were drawn. Instead of attempting to infer parameters to describe this distribution, as above, we presented the distribution itself.  The distributions produced by this visualisation method, shown in Figures~\ref{time} and \ref{rate}, reveal the complexity that the parent distribution must describe which, as we concluded earlier, cannot be effectively modelled.

We also tested whether this method is reasonable by producing replica datasets in optical-NUV colour space, as before, by drawing $1000$ $[t, \tau]$ values from the unweighted summed distributions presented in Figures~\ref{time} and \ref{rate}. These replica datasets are shown here in Figure~\ref{replicapop} in comparison to the observed colour-colour distributions of the \textsc{agn-host} and \textsc{inactive} samples. Comparing these replica colours in Figure~\ref{replicapop}, with those produced by drawing from the inferred hierarchical distributions, shown in Figure~\ref{replica}, we can see that they produce a more accurate match to the observed data for the majority of masses and morphologies. 

We therefore use this visualisation method to display the parent population distribution, rather than quoting inferred values to describe it.


