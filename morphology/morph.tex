\chapter{The morphological dependance of quenching}\label{morph}

\emph{The work in the following chapter has been published in \citet{smethurst15}.}


By studying the galaxies which have just left the `main sequence' of star formation (see top panel of Figure \ref{sfr_mass_col}), the nature of the quenching mechanisms which cause this departure can be probed. By investigating the \emph{amount} of quenching that has occurred in the blue cloud, green valley and red sequence; and by comparing that amount across the three populations, constraints can be applied to the many possible quenching mechanisms outlined in Chapter \ref{intro}. 

I have been motivated by a recent result suggesting there are two contrasting evolutionary pathways through the green valley for different morphological types (\citealt{schawinski14}, hereafter S14). S14 used the same exponentially declining star formation model, as described in Section~\ref{qmod}, to obtain predicted optical and NUV colours for four possible SFHs through the green valley; two with fast quenching rates ($\tau = [0.001, 0.25]$ $\rm{Gyr}$) and two with slower quenching rates ($\tau = [1, 2.5]$ $\rm{Gyr}$). These predicted colours were then compared to observed colours of early- and late-type green valley galaxy colours on an optical-NUV colour-colour diagram. They concluded from this diagram that late-type galaxies quench with a slower rate and form a nearly static disc population in the green valley, whereas early-type galaxies quench with very rapid rates, transitioning through the green valley and onto the red sequence in $\sim 1$~Gyr \citep{Wong12}. 

Although this result of morphologically dependent quenching is intriguing, the work of S14 is hindered for the following reasons: (i) the incompleteness of the galaxy sample; only definitively early- ($p_s \geq 0.8$) and late-type ($p_s \leq 0.8$) galaxies were studied, whereas galaxies of intermediate morphology were excluded, and (ii) the lack of statistics to support the conclusions. Here I use the same toy model but implement \starpy ~in order to statistically study the star formation histories of galaxies of all morphologies across the colour magnitude diagram.


\section{Defining the Green Valley}\label{defGV}

To define which of the $126, 316$ galaxies of the \textsc{gz2-galex} sample are in the green valley, I looked to previous definitions in the literature defining the separation between the red sequence and blue cloud. For example, \citet{Baldry04} traced this bimodality with a large sample of $66,846$ local SDSS galaxies ($0.004 < z < 0.08$) by fitting double-peaked Gaussians to the colour magnitude diagram. Their relation between the $u-r$ colour, $C'_{ur}$, and r-band magnitude, $M_r$, to define the colour cut between the blue and red galaxy populations is defined in their Equation 11 as:
\begin{equation}\label{eqgv}
C'_{ur}(M_{r}) = 2.06 - 0.244 \tanh \left( \frac{M_r + 20.07}{1.09}\right).
\end{equation}

Due to the necessity for NUV photometry in this study, matching to GALEX removed typical `red and dead' galaxies from the \textsc{gz2-galex} sample. This is apparent in the optical $u-r$ colour histograms shown in the right panels of Figure \ref{fig:cmgvsplit}; the \textsc{gz2-galex} sample is split in bins of absolute r-band magnitude and for each bin the position of the green valley at that $M_r$, as defined by \citet{Baldry04} is shown. For the \textsc{gz2-galex} sample at brighter r-band magnitudes (i.e. larger mass), this definition of the green valley seems to intersect with the observed peak at red colours. 

\begin{figure}
\centering{
\includegraphics[width=0.49\textwidth]{morphology/sdss_hist_slice.pdf}
\includegraphics[width=0.49\textwidth]{morphology/galzoo_hist_slice.pdf}}
\caption[Optical $u-r$ colour histograms in absolute r-band magnitude slices of the \textsc{gz2-galex} and Baldry et al. (2004) complete SDSS samples]{Optical $u-r$ colour histograms, sliced in absolute r-band magnitude for a complete SDSS sample (MPA-JHU catalog; left) and for the \textsc{gz2-galex} sample (right). In each panel the definition between the blue cloud and the red sequence from \citet{Baldry04} is shown by the dashed line (as defined in Equation~\ref{eqgv}); the solid lines show $\pm 1\sigma$ either side of this definition.}
\label{fig:cmgvsplit}
\end{figure}

However, for the larger SDSS sample (from the MPA-JHU catalog; \citealt[][left panels of Figure~\ref{fig:cmgvsplit}]{kauffmann03, brinchmann04}) the \citet{Baldry04} green valley definition does not intersect with the peak at red colours, as this sample is complete, containing the high mass typical `red and dead' galaxies of the red sequence. It would therefore not be appropriate to define the green valley by a visual fit to the colour magnitude diagram for this study (this method was used in S14 and adopting it here would have allowed for a direct comparison to this previous work) as this would cause green valley galaxies to be misclassified as red sequence.

I therefore adopt the \citet{Baldry04} green valley definition for this study, which is shown in Figure~\ref{fig:CMGV} by the dashed line in comparison to both the \textsc{gz2-galex} sample (left) and the SDSS data used for the fit by \cite[][right]{Baldry04}. Any galaxy within $\pm 1\sigma$ of this relationship, shown by the solid lines in Figure~\ref{fig:CMGV}, is therefore considered a green valley galaxy. 

\begin{figure*}
\centering{
\includegraphics[width=\textwidth]{morphology/col_mag_GV_Baldry_data.pdf}}
\caption[Colour-magnitude diagram showing the location of the Baldry et al. (2004) green valley definition]{Colour-magnitude diagram for the \textsc{gz2-galex} sample (left) and the SDSS sample from \citet[][right]{Baldry04}. In both panels the definition between the blue cloud and the red sequence from \citet{Baldry04} is shown by the dashed line, as defined in Equation~\ref{eqgv}. The solid lines show $\pm 1\sigma$ either side of this definition; any galaxy within the boundary of these two solid lines is considered a green valley galaxy. The lack of red sequence galaxies due to the necessity for NUV GALEX colours skews the apparent location of the green valley in the \textsc{gz2-galex} sample, therefore a literature definition of the green valley is used to ensure galaxies are correctly classified.}
\label{fig:CMGV}
\end{figure*}

However, although the galaxies identified as residing on the red sequence within the \textsc{gz2-galex} sample have NUV detections, this does not mean they are not representative of a typical red sequence `red and dead' galaxy. \cite{ko13} show that in a sample of quiescent red-sequence galaxies without $\mathrm{H}\alpha$ emission (i.e. without spectral indication of recent star formation), $26\%$ show NUV excess emission and that the fraction with recent star formation is $39\%$. Of the $48\%$ of \textsc{gz2-galex} galaxies classified as below the main sequence using the definition from \citet[][see Section~\ref{qmod}]{peng10}, $44\%$ of these galaxies lie on the red sequence ($94\%$ of all the red sequence galaxies in the \textsc{gz2-galex} sample; see Table ~\ref{table:qsubs}). I am therefore confident that the \textsc{gz2-galex} sample will include galaxies from across the entirety of the colour magnitude diagram. 

The decomposition of the \textsc{gz2-galex} sample into red sequence, green valley and blue cloud galaxies is shown in Tables~\ref{table:subs} and \ref{table:qsubs} along with further division by galaxy type and SFR (where available for the \textsc{gz2-galex} sample from the MPA-JHU catalog) respectively. The tables also list the definitions I adopt henceforth for early-type ($p_s~ \geq~0.8$), late-type ($p_d~ \geq~0.8$), smooth-like ($p_s~ >~0.5$), disc-like ($p_d~ >~0.5$), quenched ($\rm{SFR}$ $ < P - 5\sigma$), quenching ($P - 5\sigma < \rm{SFR}$ $< P - \sigma$) and star forming  ($\rm{SFR}$ $> P -\sigma$) galaxies, where $P$ is the SFR as defined by \citet{peng10} for a given stellar mass and observed time (see Equation \ref{eq:peng}). 

\begin{table}
\caption{Table showing the decomposition of the \textsc{gz2-galex} sample by galaxy type into the subsets of the colour-magnitude diagram.}
\begin{tabular*}{\textwidth}{l @{\extracolsep{\fill}}cccc}
\hline
\begin{tabular}[c]{@{}c@{}} {\color{white} -} \\ {\color{white} -}  \end{tabular} & All                                                      & Red Sequence                                              & Green Valley                                              & Blue Cloud \\  \hline 
Smooth-like ($p_s > 0.5$)        & \begin{tabular}[c]{@{}c@{}}42453\\ (33.6\%)\end{tabular} & \begin{tabular}[c]{@{}c@{}}17424\\ (61.9\%)\end{tabular}  & \begin{tabular}[c]{@{}c@{}}10687\\ (44.6\%)\end{tabular}   & \begin{tabular}[c]{@{}c@{}}14342\\ (19.3\%)\end{tabular}  \\ 
Disc-like ($p_d > 0.5$)          & \begin{tabular}[c]{@{}c@{}}83863\\ (80.7\%)\end{tabular} & \begin{tabular}[c]{@{}c@{}}10722\\ (38.1\%)\end{tabular}   & \begin{tabular}[c]{@{}c@{}}13257\\ (55.4\%)\end{tabular}  & \begin{tabular}[c]{@{}c@{}}59884\\ (47.4\%)\end{tabular}  \\
Early-type ($p_s \geq 0.8$) & \begin{tabular}[c]{@{}c@{}}10517\\ (8.3\%)\end{tabular}  & \begin{tabular}[c]{@{}c@{}}5337\\ (18.9\%)\end{tabular}    & \begin{tabular}[c]{@{}c@{}}2496\\ (10.4\%)\end{tabular}    & \begin{tabular}[c]{@{}c@{}}2684\\ (3.6\%)\end{tabular}    \\
Late-type ($p_s \geq 0.8$)  & \begin{tabular}[c]{@{}c@{}}51470\\ (40.9\%)\end{tabular} & \begin{tabular}[c]{@{}c@{}}4493\\ (15.9\%)\end{tabular}    & \begin{tabular}[c]{@{}c@{}}6817\\ (28.5\%)\end{tabular}    & \begin{tabular}[c]{@{}c@{}}40430\\ (54.4\%)\end{tabular}  \\ \hline
\textbf{Total}                       & \begin{tabular}[c]{@{}c@{}}\textbf{126316} \\ (100.0\%)\end{tabular}                                                & \begin{tabular}[c]{@{}c@{}}28146 \\ (22.3\%)\end{tabular} & \begin{tabular}[c]{@{}c@{}}23944 \\ (18.9\%)\end{tabular} & \begin{tabular}[c]{@{}c@{}}74226 \\ (58.7\%)\end{tabular} \\\hline
\end{tabular*}
\label{table:subs}
\end{table}


\begin{table}
\caption{Table showing the decomposition of the \textsc{gz2-galex} sample by their star formation rate in the subsets of the colour-magnitude diagram.}
\begin{tabular*}{\textwidth}{l @{\extracolsep{\fill}}cccc}
\hline
\begin{tabular}[c]{@{}c@{}} {\color{white} -} \\ {\color{white} -}  \end{tabular} 		& All                                                      						& Red Sequence                                              			& Green Valley                                             			 & Blue Cloud \\  \hline 
\begin{tabular}[l]{@{}l@{}}Quenched\\ ($\rm{SFR} < P - 5\sigma$) \end{tabular}	& \begin{tabular}[c]{@{}c@{}}24278\\ (19.7\%)\end{tabular} 			& \begin{tabular}[c]{@{}c@{}}17018\\ (60.9\%)\end{tabular}    & \begin{tabular}[c]{@{}c@{}}6440\\ (27.5\%)\end{tabular}    & \begin{tabular}[c]{@{}c@{}}820\\ (1.1\%)\end{tabular}  \\ 
\begin{tabular}[l]{@{}l@{}}Quenching\\ ($P - 5\sigma < \rm{SFR} < P - \sigma$) \end{tabular}	 & \begin{tabular}[c]{@{}c@{}}34743\\ (28.2\%)\end{tabular}			 & \begin{tabular}[c]{@{}c@{}}9277\\ (33.1\%)\end{tabular}    & \begin{tabular}[c]{@{}c@{}}12181\\ (51.9\%)\end{tabular}    & \begin{tabular}[c]{@{}c@{}}13285\\ (18.6\%)\end{tabular}  \\ 
\begin{tabular}[l]{@{}l@{}}Star Forming  \\ ($\rm{SFR} > P -\sigma$) \end{tabular} & \begin{tabular}[c]{@{}c@{}}63957\\ (52.0\%)\end{tabular} 			& \begin{tabular}[c]{@{}c@{}}1665 \\ (5.9\%)\end{tabular}    & \begin{tabular}[c]{@{}c@{}}4828\\ (20.6\%)\end{tabular}    & \begin{tabular}[c]{@{}c@{}}57464\\ (80.3\%)\end{tabular}  \\ \hline
\textbf{Total}                       		& \begin{tabular}[c]{@{}c@{}}\textbf{122,978} \\ (100.0\%)\end{tabular} & \begin{tabular}[c]{@{}c@{}}27960 \\ (22.7\%)\end{tabular} & \begin{tabular}[c]{@{}c@{}}23449 \\ (19.1\%)\end{tabular} & \begin{tabular}[c]{@{}c@{}}71569 \\ (58.2\%)\end{tabular} \\\hline
\end{tabular*}
\label{table:qsubs}
\end{table}

Figure~\ref{sfr_mass_sub} shows the SFR against the stellar mass for the \textsc{gz2-galex} sample (where available from the MPA-JHU catalog) by splitting it into blue cloud, green valley, red sequence, late- and early-type populations. This figure confirms that the green valley galaxies in the \textsc{gz2-galex} sample are indeed a population which have either left, or begun to leave, the star forming `main sequence' or have some residual star formation still occurring.

\begin{figure*}
\centering{
\includegraphics[width=\textwidth]{morphology/sfr_mass_subsets.pdf}}
\caption[SFR-stellar mass plane split by morphology and colour]{Star formation rate against stellar mass for the different populations of galaxies (top row, left to right: all galaxies, late-type galaxies, early-type galaxies; bottom row, left to right: blue cloud, green valley and red sequence galaxies) and how they contribute to the star forming sequence (from \citet{peng10}, shown by the solid blue line with 0.3 dex scatter by the dashed lines). Based on positions in these diagrams, the green valley does appear to be a transitional population between the blue cloud and the red sequence. Detailed analysis of star formation histories can elucidate the nature of the different populations' pathways through the green valley.}
\label{sfr_mass_sub}
\end{figure*}


\section{Results}\label{sec:morphresults}

The population density distributions for both smooth and disc weighted populations in the red sample, green valley and blue cloud are shown in Figures~\ref{red_s},~\ref{green_v} \&~\ref{blue_c} respectively. The full two dimensional distributions are shown in each case, along with a histogram showing the one dimensional projection for each parameter, $[t, \tau]$. The percentages shown in Figures~\ref{red_s},~\ref{green_v} \&~\ref{blue_c} are calculated as the fractions of the population densities located in each region of parameter space for a given population. 

Since the sample contains such a large number of galaxies, a peak in the population densities will be caused by a large number of galaxies with peaks in their individual posterior distribution at that location in parameter space. This will overwhelm contributions to this area of the population density from galaxies where this region of parameter space is not dominant in their individual posterior distributions. Therefore these fractions can be interpreted as broadly equivalent to the percentage of galaxies in a given population undergoing quenching within the stated timescale range. Although this is not quantitatively exact, it is nevertheless a useful framework for interpreting the population densities.

Figure~\ref{fig:bestfit} shows the distribution of the median walker positions (the 50th percentile of the posterior probability distribution) of each individual galaxy, split into red, green and blue disc-like ($p_d > 0.5$) and smooth-like ($p_s > 0.5$; in order to incorporate the full \textsc{gz2-galex} sample and still investigate the morphological dependance of the results) populations. Unlike in the \textsc{popstarpy} method (see Section \ref{popstarpy}) these positions were calculated without discarding any walker positions due to low probability and without weighting by the GZ2 morphological vote fractions; therefore may be more intuitive to understand than Figures~\ref{red_s},~\ref{green_v} \&~\ref{blue_c}.

Although the quenching timescales are continuous in nature, in this Section I refer to rapid, intermediate and slow quenching timescales which correspond to ranges of  $\tau ~\rm{[Gyr]} < 1.0$, $1.0 < \tau ~\rm{[Gyr]} < 2.0$ and $\tau ~\rm{[Gyr]} > 2.0$ respectively for ease of discussion.



\subsection{The Red Sequence}\label{rs}

\begin{figure*}
\centering{
\includegraphics[width=0.55\textwidth]{morphology/red_smooth.pdf}\\
\includegraphics[width=0.55\textwidth]{morphology/red_disc.pdf}}
\caption[Population densities of red smooth and disc galaxies]{Contour plots showing the population densities for red galaxies of the \textsc{gz2-galex} sample, weighted by the morphological vote fractions from GZ2 to give both bulge (top) and disc (bottom) dominated distributions. The histograms show the projection into one dimension for each parameter. The dashed lines show the separation between rapid ($\tau ~\rm{[Gyr]} < 1.0$), intermediate ($1.0 < \tau ~\rm{[Gyr]} < 2.0$) and slow ($\tau ~\rm{[Gyr]} > 2.0$) quenching timescales with the fraction of the combined posterior probability distribution in each region shown (see Section~\ref{stats}).}
\label{red_s}
\end{figure*}

The top panel of Figure~\ref{red_s} reveals that the red smooth weighted population density is dominated $(49.5\%$; see Figure~\ref{red_s}) by rapid quenching timescales across all cosmic time resulting in a very low current SFR. At early quenching times (high redshift), the population density is dominated by slow and intermediate timescales (top panel of Figure~\ref{red_s}). Perhaps this is the influence of intermediate galaxies (with $p_s \sim p_d \sim 0.5$), hence why similar high density areas exist for both the smooth and disc weighted populations in the top and bottom panels of Figure~\ref{red_s}. This is especially apparent considering there are far more of these intermediate galaxies than those that are definitively early- or late-types (see Table~\ref{table:subs}). 

The bottom panel of Figure~\ref{red_s} reveals a bimodal disc weighted population density between rapid $(31.3\%)$ and slow $(44.1\%)$ quenching timescales. The \emph{very} slow ($\tau > 3.0 ~\rm{Gyr}$) quenching timescales present in these red disc population densities (which are not seen in either the green valley or blue cloud, see Figures~\ref{green_v} and~\ref{blue_c}) suggest that these galaxies have only just reached the red sequence after a very slow evolution across the colour-magnitude diagram. Considering their limited number and the requirement for NUV emission, it is likely that these galaxies are currently on the edge of the red sequence having recently (and finally) moved out of the green valley. 

Despite this dominance of slow quenching timescales in the red disc weighted population densities, rapid quenching timescales also contribute a significant fraction ($31.3\%$). This is similar to the red smooth weighted population, albeit with a marginally lower fraction of population density. 

\citet{tojeiro13} used the VErsatile SPectral Analyses spectral fitting code (VESPA; \citealt{tojeiro07}) and found that red late-type spirals show 17 times more recent star formation than red elliptical galaxies. The results in Figure \ref{red_s} can be tested against this finding  by comparing the SFRs predicted by the inferred SFH model of both the smooth and disc weighted population densities (these SFRs are shown in Figure \ref{pred}). For the peak at early times and slow quenching rate in the red disc weighted population density, this SFH model still has some residual star formation occurring with a SFR$~\sim0.105 M_{\odot}yr^{-1}$. Whereas the peak at recent times and rapid quenching rates in the red smooth weighted population density, this SFH model has a resultant SFR$~\sim0.0075 M_{\odot}yr^{-1}$. This is approximately 14 times less than the residual SFR still occurring in the red disc weighted population; within error, this is in agreement with the findings of \citet{tojeiro13}. 

These results for the red galaxies have many implications for green valley galaxies, as all of these systems must have passed through the green valley on their way to the red sequence.


\subsection{Green Valley Galaxies}\label{gv}

\begin{figure*}
\centering{
\includegraphics[width=0.55\textwidth]{morphology/green_smooth.pdf}\\
\includegraphics[width=0.55\textwidth]{morphology/green_disc.pdf}}
\caption[Population densities of green smooth and disc galaxies]{Contour plots showing the population densities for green valley galaxies in the \textsc{gz2-galex} sample weighted by the morphological vote fractions from GZ2 to give both bulge (top) and disc (bottom) dominated distributions. The histograms show the projection into one dimension for each parameter. The dashed lines show the separation between rapid ($\tau ~\rm{[Gyr]} < 1.0$), intermediate ($1.0 < \tau ~\rm{[Gyr]} < 2.0$) and slow ($\tau ~\rm{[Gyr]} > 2.0$) quenching timescales with the fraction of the combined posterior probability distribution in each region shown (see Section~\ref{stats}).}
\label{green_v}
\end{figure*}

Figure \ref{green_v} shows how the smooth weighted green valley population densities are dominated by both intermediate quenching rates ($40.6\%$) and slow quenching at rates early times ($z > 1$; $40.7\%$). The fraction of the population density at rapid quenching rates in the smooth weighted population has dropped by over a half compared to the red sequence smooth weighted population. However this will be influenced by the observability of galaxies in the green valley undergoing such a rapid quench. To quantify this, I tested the time spent in the green valley across the $[t, \tau]$ parameter space, which is shown in Figure~\ref{fig:timeingv}. The galaxies with such a rapid decline in star formation rate spend very little time in the green valley and so will be detected at a lower fraction than those galaxies transitioning more slowly with intermediate quenching rates.  This explains the observed number of intermediate morphology galaxies (see Table \ref{table:subs}) which are present in the green valley (assuming a morphological change occurs during the quench) and explains the dominance of rapid quenching rates in the red smooth and disc weighted population densities (see Figure \ref{red_s}).

\begin{figure*}
\centering{
\includegraphics[width=0.9\textwidth]{morphology/green_valley_time.pdf}}
\caption[Time spent in the green valley across parameter space]{Plot showing the time spent in the green valley across the SFH model parameter space by the current epoch. This affects the observability of those galaxies which have quenched rapidly and recently and have passed too quickly through the green valley to be detected. The white region denotes those models with colours that do not enter the green valley by the present cosmic time.}
\label{fig:timeingv}
\end{figure*}

Conversely, the green valley disc weighted population densities are now completely dominated by slow quenching rates ($47.4\%$) with a slightly smaller fraction of intermediate quenching rates detected ($37.6\%$; see Figure~\ref{green_v}).

If the population densities of Figure~\ref{green_v} and Figure~\ref{red_s} are compared, quenching is detected at later cosmic times (lower redshift) in the green valley than in the red sequence for both morphological types. This therefore suggests that both morphologies are tracing the evolution of the red sequence, confirming that the green valley is indeed a transitional population between blue cloud and red sequence regardless of morphology. 

Given enough time ($t\sim4 - 5~\rm{Gyr}$), the current green valley disc galaxies will therefore eventually transition through to the red sequence (the right panel of Figure~\ref{sfr_mass_col} shows galaxies with $\tau > 1.0~\rm{Gyr}$ do not approach the red sequence within $3~\rm{Gyr}$ post quench). This is most likely the origin of the `red spirals', attributed to the \emph{very} slow quenching rates discussed in Section~\ref{rs} (and see bottom panel of Figure~\ref{red_s}). This is in contradiction to the conclusions of S14 who state that the green valley disc population is static in nature. 

Considering this result that the green valley is a transitional population, the ratio of smooth:disc galaxies that is currently observed in the green valley is expected to evolve into the ratio observed in the red sequence (assuming that the decreased number of galaxies detected in the red sequence due to matching to GALEX is independent of morphology). Table~\ref{table:subs} shows the ratio of smooth-like : disc-like galaxies in the red sequence is $62:38$, whereas in the green valley this ratio is $45:55$. Making the very simple assumptions that this ratio does not change with redshift and that quenching is the only mechanism which causes a morphological transformation, then $31.2\%$ of the disc-like galaxies in the green valley would have to undergo a morphological change to a smooth-like galaxy. 

Inspecting the the disc weighted green valley population density (bottom panel of Figure~\ref{green_v}) reveals that $29.4\%$ of the distribution occupies the $\tau < 1.5 ~\rm{Gyr}$ parameter space. Since this is a similar fraction to the number of green valley disc-like galaxies which would have to undergo a morphological change to a smooth-like galaxy to match the ratio of smooth:disc galaxies in the red sequence, this suggests that quenching mechanisms with $\tau < 1.5 ~\rm{Gyr}$ are capable of destroying the disc-dominated structure of galaxies. 

However this is most likely an overestimate of the timescale that can cause a morphological change because of the observability of those galaxies which undergo such a rapid quench. \citet{Martin07} showed that after considering the time spent in the green valley, the fraction of galaxies inferred to be undergoing a rapid quench would quadruple. If this result is applied to the disc weighted green valley population density (bottom panel Figure~\ref{green_v}) and the distribution was then renormalised, the resulting population density would be very similar in shape to the one found for the disc weighted red sequence population (bottom panel Figure~\ref{red_s}). This provides more evidence that the green valley is tracing the evolution of the red sequence and is therefore a transitional population.

All of this evidence suggests that there are not just two contrasting evolutionary pathways through the green valley for different morphological types as concluded by S14. The intermediate quenching rates reside in the space between the extremes sampled by the optical-NUV colour-colour diagrams of S14. The inclusion of the intermediate galaxies in this investigation and the use of a statistical method, elucidates a continuum of quenching timescales, with all galaxies transitioning through the green valley to the red sequence during quenching regardless of morphology. 

Instead of concluding that \emph{`the green valley is a red herring'} as in S14, I would therefore conclude that the \emph{`grass is always redder on the other side'}.


\subsection{Blue Cloud Galaxies}\label{bc}

\begin{figure*}
\centering{
\includegraphics[width=0.55\textwidth]{morphology/blue_smooth.pdf}\\
\includegraphics[width=0.55\textwidth]{morphology/blue_disc.pdf}}
\caption[Population densities of blue smooth and disc galaxies]{Contour plots showing the population densities blue cloud galaxies in the \textsc{gz2-galex} sample, weighted by the morphological vote fractions from GZ2 to give both bulge (top) and disc (bottom) dominated distributions. The histograms show the projection into one dimension for each parameter. The dashed lines show the separation between rapid ($\tau ~\rm{[Gyr]} < 1.0$), intermediate ($1.0 < \tau ~\rm{[Gyr]} < 2.0$) and slow ($\tau ~\rm{[Gyr]} > 2.0$) quenching timescales with the fraction of the combined posterior probability distribution in each region shown (see Section~\ref{stats}). Positions with probabilities less than 0.2 are discarded as poorly fit models, therefore unsurprisingly blue cloud galaxies are not well described by a quenching star formation model. }
\label{blue_c}
\end{figure*}

\begin{figure*}
\includegraphics[width=0.95\textwidth]{morphology/contour_t_tau_mcmc_bestfit.pdf}
\caption[Best fit contours for red, green and blue clean galaxies]{Contours showing the positions in the $[t, \tau]$ parameter space of the median walker position (the 50th percentile; as shown by the intersection of the solid blue lines in Figure~\ref{one_example}) for each galaxy for all (top), disc ($p_d > 0.5$; middle), and smooth ($p_s > 0.5$; bottom) red sequence, green valley and blue cloud galaxies in the left, middle and bottom panels respectively. The error bars on each panel shows the average $68\%$ confidence on the median positions (calculated from the 16th and 84th percentile, as shown by the blue dashed lines in Figure~\ref{one_example}). These positions were calculated without discarding any walker positions due to low probability and without weighting by vote fractions, therefore this plot may be more intuitive than Figures~\ref{red_s},~\ref{green_v} \&~\ref{blue_c}. The differences between the smooth and disc populations and between the red, green and blue populations remain clearly apparent.}
\label{fig:bestfit}
\end{figure*}

Since the blue cloud is considered to be made of star forming galaxies \starpy~ is expected to have some difficulty inferring any quenching model to describe them, as confirmed by Figure~\ref{blue_c}. The attempt to characterise a star forming galaxy with a quenched SFH model leads \starpy~ to attribute the extremely blue colours of the majority of these galaxies with a constant SFR until recent times with a fast quench at the observed redshift (i.e. the colour has not had enough time to change from blue post-quench). 

This is particularly apparent for the blue disc weighted population. Perhaps even galaxies which are currently quenching slowly across the blue cloud cannot be well fit by the quenching models implemented, as they still have high SFRs despite some quenching. By definition although a galaxy is undergoing quenching, star formation can still be occurring in a galaxy, just at a slower rate than at earlier times, described by $\tau$.

A very small fraction of the blue smooth weighted population density is found at slower quenching rates which began prior to $z \sim 0.5 $. These populations have been blue for a considerable period of time, slowly using up their gas for star formation by the Kennicutt$-$Schmidt law \citep{Schmidt59, Kennicutt97}. However the dominant fraction of the blue smooth weighted population density occurs at rapid quenching rates at recent times. This therefore provides some support to the theories for blue ellipticals as either merger-driven ($\sim76\%$; like those identified as recently quenched ellipticals with properties consistent with a merger origin by \citealt{McIntosh14}) or gas inflow-driven reinvigorated star formation that is now slowly decreasing ($\sim24\%$; such as the population of blue spheroidal galaxies studied by \citealt{Kaviraj13}).

The blue cloud is therefore primarily composed of both star forming galaxies of all morphologies and a smooth population which are undergoing a rapid quench, presumably after a previous event triggered star formation and turned them blue.


\section{Discussion}\label{morph:discussion}

In the previous section I presented the results of using \textsc{popstarpy} to derive the distribution of quenching histories for galaxies across the colour magnitude diagram. I  found differences between the SFHs of smooth- and disc-weighted populations of the red sequence, green valley and blue cloud. These results are summarised in Table~\ref{table:morphresults}. In this section I will speculate on the following question: what are the possible mechanisms driving these differences? 

\begin{table}[]
\centering
\caption{Table summarising the main results in Section~\ref{results} by describing the distributions derived for each population across the colour-magnitude diagram.}
\label{table:morphresults}
\resizebox{\textwidth}{!}{%
\begin{tabular}{rccc}
\hline
                                     & Red Sequence                                                                                 & Green Valley                                                                            & Blue Cloud                                                                       \\ \hline \hline
\multicolumn{1}{r|}{Smooth-weighted} & \begin{tabular}[c]{@{}c@{}}Dominated by \\ rapid \\ quenching rates\end{tabular}             & \begin{tabular}[c]{@{}c@{}}Dominated by \\ intermediate \\ quenching rates\end{tabular} & \begin{tabular}[c]{@{}c@{}}Dominated by \\ rapid \\ quenching rates\end{tabular} \\ \hline
\multicolumn{1}{r|}{Disc-weighted}   & \begin{tabular}[c]{@{}c@{}}Bimodal between \\ slow and rapid \\ quenching rates\end{tabular} & \begin{tabular}[c]{@{}c@{}}Dominated by \\ slow \\ quenching rates\end{tabular}         & -                                                                                \\ \hline
\end{tabular}%
}
\end{table}

\subsection{Rapid Quenching Rates}\label{rapid}

Rapid quenching is found at larger fractions in the smooth weighted population densities than the disc weighted. Red galaxies also have larger fractions of raid quenching rates than green valley population densities. However the observability of a galaxy may decline with increasing quenching rate. Rapid mechanisms may be more common in the green valley than found in Figure \ref{green_v}, however this observability should not depend on morphology. The conclusion that rapid quenching mechanisms are detected more for smooth rather than disc populations still holds. 

This suggests that rapid quenching mechanisms can cause a change in morphology from a disc- to a smooth dominated galaxy as it quickly traverses the colour-magnitude diagram to the red sequence. This is supported by the number of disc galaxies that would need to undergo a morphological change in order for the disc : smooth ratio of galaxies in the green valley to match that of the red sequence (see Section~\ref{gv}). From this indirect evidence I suggest that this observed rapid quenching mechanism is caused by major mergers. However, since a significant fraction of the red disc weighted population density is found at rapid quenching rates, this suggests that the quenching may have occurred more rapidly than the morphological change in such a merger.

Inspection of the individual galaxies dominating this area of the smooth weighted red population density reveals that this dominance of rapid quenching not arise due to \emph{currently} merging galaxies. Although efforts were made to remove currently merging galaxies from the \textsc{gz2-galex} sample (using the GZ2 morphological vote fractions, see Section \ref{class}), this check was still performed to ensure no merging pairs were missed by the GZ users. Instead, the area is dominated by typical smooth galaxies with both red optical and NUV colours that \starpy~ attributes to rapid quenching at early times. Although a prescription for modelling a merger in the SFH is not included in this work the after-effects can still be detected (see Section \ref{future} for future work planned with \starpy).

In order to achieve such rapid quenching rates ($\tau \lesssim 0.5$) in a simulation of a major merger, \citet*{springel05} showed that feedback from black hole activity is necessary. As discussed in Section \ref{intro}, powerful quasar outflows are thought to be able to remove much of the gas from the inner regions of the galaxy, terminating star formation on extremely short timescales. \citet{Bell06}, using data from the COMBO-17 redshift survey ($0.4 < z < 0.8$), estimate a merger timescale of $\sim 0.4~\rm{Gyr}$ for the merger to go from being classified as a close galaxy pair to morphologically disturbed. \citet*{springel05b} consequently find using hydrodynamical simulations that after $\sim1~\rm{Gyr}$ the merger remnant has reddened to $u-r \sim 2.0$. 

These simulation results are in agreement with the simple exponential quenching models used here which show (Figure~\ref{sfr_mass_col}) that the models with a SFH with $\tau < 0.4~\rm{Gyr}$ (under the simple assumption that the merger timescale will be $\sim$ quenching rate) have reached the red sequence, with $u-r ~\gtrsim 2.2$, within $\sim1~\rm{Gyr}$. This could explain the fraction of the red disc weighted population with very rapid quenching rates. These galaxies may have undergone a major merger recently but are still undergoing a morphological change from disc, to disturbed, to an eventual smooth galaxy (see also \citealt{vdW09}). Similarly they may have retained their disc structure in a merger, such as in recent simulations by \citet{pontzen16}. This possible connection between AGN feedback and rapid quenching timescales is explored further in Chapter~\ref{agnfeedback}. 

These aid quenching rates, while found across all populations do not fully characterise the entire red sequence or green valley populations. Dry major mergers therefore do not fully account for the formation of any galaxy population, supporting the observational conclusions made by \citet{Bell07,Bundy07, kaviraj14a} and simulations by \citet{Genel08}. 

\subsection{Intermediate Quenching Rates}\label{int}

Intermediate quenching timescales are found to be equally prevalent across both smooth and disc weighted populations across cosmic time, and are particularly dominant in the green valley. This suggests that this intermediate quenching route must therefore be possible with mechanisms that both preserve and transform morphology. It is this result of another route through the green valley that is in contradiction with the findings of S14. 

In the simulations of \citet*{springel05b}, which do not include feedback from black holes, merger remnants can sustain low levels of star formation for several Gyrs if even the a small fraction of gas is not consumed in the merger triggered starburst (either because the mass ratio is not large enough or from the lack of strong black hole activity). The remnants from these simulations take $\sim5.5~\rm{Gyr}$ to reach red optical colours of $u-r \sim 2.1$. The SFH models in this work with intermediate quenching rates of $1.0 \lesssim ~\tau~\rm{[Gyr]} ~\lesssim 2.0$ take approximately $2.5-5.5~\rm{Gyr}$ to reach these red colours. Similarly the recent simulations of \citet{pontzen16} show that without AGN feedback the SFR of a disc galaxy can recover post merger and remain a star forming disc. 

I propose that the intermediate quenching timescales are caused by gas rich major mergers, major mergers without black hole feedback, minor mergers and galaxy interactions. This is supported by the findings of \citet{Lotz11}, who find in simulations that increasing the baryonic gas fraction in a merger (mass ratios $\sim 1:1-1:4$) the observability timescale of the merger increases from $\sim0.2~\rm{Gyr}$ (with little gas, as above for major mergers causing rapid quenching timescales) up to $\sim1.5~\rm{Gyr}$ (with large gas fractions). 

\citet{lotz08b} also show that the remnants of simulated equal mass gas rich disc mergers (wet disc mergers) are observable for $\gtrsim1~\rm{Gyr}$ post merger and state that they appear ``disc-like and dusty" in the simulations, which is consistent with an ``early-type spiral morphology".  Such galaxies are often observed to have spiral features with a dominant bulge, suggesting that such galaxies may divide the votes of the GZ2 users, producing vote fractions of $p_s \sim p_d \sim 0.5$. Such a vote fraction may also arise because the galaxy is at a large distance or because it is an S0 galaxy whose morphology can be interpreted by different GZ2 users in different ways. \citet{GZ2} find that S0 galaxies expertly classified by \citet{nair10} are more commonly classified as ellipticals by GZ2 users, but have a significant tail to high disc vote fractions. This may be why the intermediate quenching timescales are equally dominant for both smooth and disc populations in Figures~\ref{red_s} and~\ref{green_v}. 

 Observationally, \citet{Darg10a} showed an increase in the spiral to elliptical ratio for merging galaxies ($0.005 < z < 0.1$) by a factor of two compared to the typical galaxy population. They attribute this to the much longer timescales during which mergers of spirals are observable compared to mergers with elliptical galaxies. This confirms the hypothesis that the longer quenching timescales of $\tau < 1.5 ~\rm{Gyr}$ found for the disc weighted green valley population density will eventually lead to a morphological change (see Section \ref{gv}) possibly caused by a wet major or minor merger. Similarly, \citet{Casteels13} observe that galaxies ($0.01 < z < 0.09$) which are interacting often retain their spiral structures and that a spiral galaxy which has been classified as having `loose winding arms' by the GZ2 users are often entering the early stages of mergers and interactions.

$40.6\%$ of the population density for smooth weighted green valley galaxies is found in the intermediate quenching rate regime (see Figure~\ref{green_v}). This is in agreement with work done by \citet{kaviraj14a, kaviraj14b} who by studying SDSS photometry ($z<0.07$) state that approximately half of the star formation in galaxies is driven by minor mergers at $0.5 < z < 0.7$ therefore exhausting available gas for star formation and consequently causing a gradual decline in the star formation rate. 

Assuming that rapid and intermediate quenching rates are caused by major or minor mergers, the fraction of population density with $\tau \leq 2~\rm{Gyr}$ can provide an estimate for the percentage of galaxies with a merger dominated evolutionary history.  This is calculated as $73.9\%$ and $59.3\%$, for the smooth weighted red sequence and green valley populations (top panels Figures~\ref{red_s} and~\ref{green_v}) respectively. This supports earlier work by \cite{kaviraj11} who, using multi wavelength photometry of galaxies in COSMOS \citep{Scoville07}, found that $70\%$ of early-type galaxies appear morphologically disturbed, suggesting either a minor or major merger in their history. Note that the star formation model used here is a basic one and has no prescription for reignition of star formation post-quench which can also cause morphological disturbance of a galaxy, like those detected by \cite{kaviraj11} and seen in simulations by \cite{pontzen16}.

\citet{Darg10a} show in their Figure 6 that that less than a merger ratio of $1:10$ (up to $\sim 1:100$), green is the dominant average galaxy colour of visually identified merging pairs in GZ. These pairs are also dominated by spiral-spiral mergers as opposed to elliptical-elliptical and elliptical-spiral mergers. This supports the hypothesis that these intermediate timescales are caused in part by minor mergers due to their dominance in the disc weighted green valley population densities. 

Any external event which can cause either a burst of star formation (depleting the gas available) or directly strip a galaxy of its gas, for example galaxy harassment, interactions, ram pressure stripping, strangulation and interactions internal to clusters, should also cause quenching with an intermediate rate. Such mechanisms would be the dominant cause of quenching in dense environments; considering that the majority of galaxies reside in groups or clusters (\citealt{Coil08} find that green valley galaxies are just as clustered as red sequence galaxies). It is not surprising therefore that the majority of the \textsc{gz2-galex} galaxies ($\sim50\%$) are considered intermediate in morphology (i.e., $p_d \sim p_s \sim 0.5, $see Table~\ref{table:subs}) and therefore may be undergoing or have undergone such an interaction. This obvious dependancy of the quenching parameters with the galaxy environment will be investigated further in Chapter~\ref{chap:env}.


\subsection{Slow Quenching Rates}\label{slow}
Although intermediate and rapid quenching timescales are the dominant mechanisms across the colour-magnitude diagram, together they cannot completely account for the quenching of disc galaxies. S14 concluded that slow quenching timescales were the most dominant mechanism for disc galaxies. However I show that: (i) intermediate quenching timescales are equally important in the green valley and (ii) rapid quenching timescales are equally important in the red sequence. There is also a significantly lower fraction in the smooth weighted population densitiesof slow quenching rates; suggesting that the evolution (or indeed creation) of typical smooth galaxies is dominated by processes external to the galaxy. The exceptions are galaxies in the blue cloud where a small fraction of the smooth weighted population density is found at slow quenching rates, which could be due to a reinvigoration of star formation which is slowly depleting the gas available (via the Kennicutt$-$Schmidt law).

\citet{Bamford09} using GZ1 vote fractions of galaxies in the SDSS, found a significant fraction of field galaxies are high stellar mass red spiral galaxies. As these galaxies are isolated from the effects of interactions from other galaxies, the slow quenching mechanisms present in the red smooth weighted population densities are most likely due to secular processes (i.e. mechanisms internal to the galaxy, in the absence of sudden accretion or merger events; \citealt{kormendy04, Sheth12}). Bar formation in a disc galaxy is such a mechanism, whereby gas is funnelled to the centre of the galaxy by the bar over long timescales where it is used for star formation \citep{masters12a, saintonge12, Cheung13}, consequently forming a `pseudo-bulge' \citep{Kormendy10, Simmons13}.

Table~\ref{table:subs} shows that $3.9\%$ of the sample are red sequence late-type galaxies, i.e. red late-type spirals. This is, within uncertainties, in agreement with the findings of \citet{masters10c}, who find $\sim6\%$ of late-type spirals are red when defined by a cut in the $g-r$ optical colour (rather than with $u-r$ as used in this investigation) and are at the `blue end of the red sequence'. 

If these slow quenching timescales are due to secular evolution processes, this is to be expected since these processes do not change the disc dominated nature of a galaxy. 

\section{Conclusions}\label{morph:conc}

I have used morphological classifications from the Galaxy Zoo 2 project to determine the morphology-dependent star formation histories of galaxies via a Bayesian analysis of an exponentially declining star formation quenching model. The most likely parameters were determined for the quenching onset time, $t_q$ and quenching timescale $\tau$ in this model for galaxies across the blue cloud, green valley and red sequence to trace the morphological dependance of galactic evolution across the colour-magnitude diagram. The green valley is indeed found to be a transitional population for all morphological types (in agreement with \citet{schawinski14}), however this transition proceeds slowly for the majority of disc dominated galaxies and occurs rapidly for the majority of smooth dominated galaxies in the red sequence. However, in addition to \citet{schawinski14}, this Bayesian approach has revealed a more nuanced result, specifically that the prevailing mechanism across all morphologies and populations is quenching with intermediate timescales. The main findings are summarised as follows:
\begin{enumerate}[(i)]
\item The subset of red sequence galaxies with NUV emission studied in this investigation are found to have similar preferences for quenching timescales compared to green valley galaxies, but the quenching occurs at earlier quenching times (i.e. higher redshift) regardless of morphology (see Figures~\ref{red_s} and~\ref{green_v}). Therefore the quenching mechanisms currently occurring in the green valley were also active in creating the `blue end of of the red sequence' at earlier times; confirming that the green valley is indeed a transitional population, regardless of morphology.

\item The typical red galaxy with NUV emission studied, is elliptical in morphology and has undergone a rapid to intermediate quench at some point in cosmic time, resulting in a very low current SFR (see Section~\ref{rs}.

\item The green valley as it is currently observed is dominated by very slowly evolving disc dominated galaxies along with intermediate- and smooth dominated galaxies which pass across it with intermediate timescales within $\sim 1.0-1.5~\rm{Gyr}$ (see Section~\ref{gv}).

\item There are many different mechanisms responsible for quenching, all causing a galaxy to progress through the green valley, which are dependant on galaxy type, with the smooth and disc dominated galaxies each having different dominant star formation histories across the colour-magnitude diagram. These timescales can be roughly split into three main regimes; rapid ($\tau < 1.0~$Gyr), intermediate ($1.0 < \tau~$[Gyr]~$< 2.0$) and slow ($\tau > 2.0~$ Gyr) quenching.

\item Blue cloud galaxies are not well fit by a quenching model of star formation due to the continuous high star formation rates occurring (see Figure~\ref{blue_c}).

\item Rapid quenching timescales are detected with a lower probability for green valley galaxies than the red sequence galaxies studied. I speculate that this quenching mechanism is caused by major mergers with black hole feedback, which are able to expel the remaining gas not initially exhausted in the merger-induced starburst and which can cause a change in morphology from disc- to bulge-dominated. The colour-change timescales from previous simulations of such events agree with the derived timescales (see Section~\ref{rapid}). These rapid timescales are instrumental in forming red galaxies, however galaxies at the current epoch passing through the green valley do so at more intermediate timescales (see Figure~\ref{green_v}).

\item Intermediate quenching timescales ($1.0 < ~\tau~\rm{[Gyr]}~ < 2.0 $) are found with constant density across red and green galaxies for both smooth- and disc-weighted populations, the timescales for which agree with observed and simulated minor merger timescales (see Section~\ref{int}). I hypothesise that such timescales can be caused by a number of external processes, including gas rich major mergers, mergers without black hole feedback, galaxy harassment, interactions and ram pressure stripping. The timescales and observed morphologies from previous studies agree with the results, including that this is the dominant mechanisms for intermediate galaxies such as early-type spiral galaxies with spiral features but a dominant bulge, which split the GZ2 vote fractions (see Section~\ref{int}). 

\item Slow quenching timescales are the most dominant mechanism in the disc galaxy populations across the colour-magnitude diagram. Disc galaxies are often found in the field, therefore I hypothesise that such slow quenching timescales are caused by secular evolution and processes internal to the galaxy (see Section \ref{bc}). A small amount of slow quenching timescales is also detected for blue elliptical galaxies which is attributed to a reinvigoration of star formation, the peak of which has passed and has started to decline by slowly depleting the gas available (see Section~\ref{bc}). 
\end{enumerate}
